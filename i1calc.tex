% $Id$

\chapter{Logische Kalküle}
\label{chap:calc}

Nachdem die bisherigen Kapitel den Bogen von der praktischen Konstruktion
einfacher Programme bis zur
objektorientierten Programmierung geschlagen haben, beleuchtet dieses Kapitel
einige weitere theoretische Grundlagen der
Programmierung.

Für die Informatik ist der Begriff des \emph{logischen Kalküls\index{Kalkül}\index{logischer Kalkül}} von zentraler Bedeutung.
Ein Kalkül dient dazu, auf formale Art und Weise wahre Aussagen abzuleiten, ohne daß
es dabei nötig wird, den Sinn der Aussagen zu begreifen.   
Logische Kalküle bilden außerdem eine mathematische Grundlage, um die
Bedeutung von Programmen präzise zu erklären.  Der $\lambda$-Kalkül,
um den es im nächsten Kapitel geht, ist ein Beispiel.

\section{Wahrheit und Beweisbarkeit}
\label{sec:truth-provability}

\index{Wahrheit}\index{Beweisbarkeit}Im Zentrum
dieses Kapitels steht die vertraute \textit{Aussagenlogik},\index{Aussagenlogik} in
Anhang~\ref{sec:aussagenlogik} kurz zusammengefaßt.  Die "<elementaren
Bestandteile"> von Aussagen sehen zum Beispiel so aus:
%
\begin{enumerate}
\item Das Gras ist grün.
\item Der Himmel ist grün.
\item Axl ist grün im Gesicht.
\end{enumerate}
%
Die erste dieser Aussagen ist wahr, die zweite falsch, und die
Wahrheit der dritten Aussage hängt von den Umständen ab.  Tatsächlich
hängen auch die beiden anderen Aussagen von Umständen ab: Sie gelten zwar in
unserer Welt (von Spitzfindigkeiten einmal abgesehen), aber es sind
Welten\index{Welt} (auch \textit{Modelle\index{Modell}} genannt)
vorstellbar, in denen das Gras blau und der Himmel grün ist.

Neben den elementaren
Aussagen\index{Aussage}\index{Aussage!elementar}\index{elementare Aussage}
gibt es auch zusammengesetzte Aussagen\index{Aussage!zusammengesetzt}\index{zusammengesetzte Aussage}:
%
\begin{enumerate}
\item Das Gras ist grün und der Himmel ist grün.
\item Das Gras ist grün und das Gras ist nicht grün.
\item Das Gras ist grün oder das Gras ist nicht grün.
\item Wenn Axl grün im Gesicht ist, so ist er nicht blau im Gesicht.
\end{enumerate}
%
Die Wahrheit der ersten Aussage hängt von der Wahrheit ihrer
Teilaussagen ab.  Die Wahrheit der zweiten und dritten Aussage ist
allerdings völlig unabhängig von der Wahrheit der Teilaussagen: die
zweite ist immer falsch, die dritte immer wahr.  Die vierte Aussage
ist in manchen Welten nicht wahr: Axls Gesicht könnte grün mit
blauen Punkten sein.  Sie wird höchstens dadurch wahr, daß zusätzlich postuliert
wird, daß beide Teilaussagen nicht gleichzeitig wahr sein können,
also daß, wenn Axl blau im Gesicht ist, er nicht grün im Gesicht ist,
und wenn er grün im Gesicht ist, er dann nicht blau im Gesicht ist.

Es gibt also mindestens zwei Varianten der Wahrheit:
%
\begin{itemize}
\item Eine Aussage ist wahr in einer bestimmten Welt.
\item Eine Aussage ist wahr in allen Welten.
\end{itemize}
%
Feststellen läßt sich die Wahrheit durch Einsetzen von
Wahrheitswerten~-- wahr oder falsch~--
für die elementaren Aussagen wie in
Anhang~\ref{sec:aussagenlogik}.  In der
Aussagenlogik\index{Aussagenlogik} wird also eine Welt durch die Menge der in ihr wahren
elementaren Aussagen bestimmt.  Diese legen damit auch die Bedeutung jeder
zusammengesetzten Aussage fest.

Um die
zweite Art Wahrheit festzustellen, ist es notwendig, alle möglichen
Belegungen von Wahrheitswerten durchzuprobieren.  
Das Durchprobier-Verfahren ist häufig mühsam oder gar völlig unmöglich.  
Allerdings ist es möglich, die zweite Sorte Wahrheit völlig
unabhängig von irgendwelchen Wahrheitsbelegungen festzustellen; dazu
sind die logischen Kalküle da.  Ein logischer Kalkül dient der
Formulierung eines \textit{Beweises\index{Beweis}} für eine Formel.  Ist ein Beweis
gefunden, so ist die Wahrheit in allen Welten garantiert, sofern der Kalkül
selbst korrekt ist.  Damit gibt
es in der Logik neben der Wahrheit auch noch den Begriff der
\textit{Beweisbarkeit}.  Entsprechend gibt es zwei Zweige der Logik:
die \textit{Modelltheorie\index{Modelltheorie}} und die \textit{Beweistheorie\index{Beweistheorie}}.

\section{Ein Kalkül für die Aussagenlogik}
\label{sec:calculus-logic}

In diesem Abschnitt geht es um einen einfachen Kalkül für die
Aussagenlogik.  Generell besteht ein logischer Kalkül aus
\begin{enumerate}
\item einer Menge von \emph{Axiomen}, d.~h.\ grundlegenden \index{Axiom}
  Aussagen, deren Wahrheit unbestreitbar ist, und
\item einer Menge von \textit{Regeln},\index{Regel}  mit denen neue wahre
  Aussagen aus anderen Aussagen abgeleitet werden können, wenn deren Wahrheit
  bereits feststeht.
\end{enumerate}
%
Ein \emph{Beweis} einer Aussage in einem bestimmten Kalkül besteht in der
Angabe einer endlichen Folge von Regelanwendungen, welche diese Aussage auf die
Axiome des Kalküls zurückführen.  Zunächst gehört zu einem Kalkül
\index{Kalkül} eine Sprache, in der sich die zu beweisenden Aussagen
formulieren lassen.

Für die Definition der Sprache $\mathcal{L}_1$ der Aussagenlogik
wird der Begriff der Termmenge aus Kapitel~\ref{cha:indu} herangezogen:
%
\begin{definition} Sei $\mathcal{V}=\{P_1,P_2,\ldots,
  P_N\}$ eine endliche Menge der \textit{aussagenlogischen
    Variablen\index{Variable!aussagenlogisch}\index{aussagenlogische Variable}}, mit $N>0$, $\mathcal{C} = \{ \top, \bot \}$ die Menge der
  \index{*@$\top$}\index{*@$\bot$}
  \textit{aussagenlogischen
    Konstanten\index{Konstante!aussagenlogisch}\index{aussagenlogische
      Konstante}}.  Die Elemente von
  $\mathcal{C}\cup\mathcal{V}$ werden auch \emph{Literale\index{Literal}} genannt.
  \index{Literal}  Die Sprache
  $\mathcal{L}_1$\index{L1@$\mathcal{L}_1$} der
  \textit{aussagenlogischen
    Formeln\index{Formel!aussagenlogisch}\index{aussagenlogische
      Formel}} ist durch folgende Grammatik definiert:
  \begin{grammar}
    \meta{$\mathcal{L}_1$} \: $\top$ \| $\bot$
    \> \| \meta{$\mathcal{V}$} \| $\neg$ \meta{$\mathcal{V}$}
    \> \| \meta{$\mathcal{L}_1$} $\wedge$ \meta{$\mathcal{L}_1$}
    \> \| \meta{$\mathcal{L}_1$} $\vee$ \meta{$\mathcal{L}_1$}
  \end{grammar}

\end{definition}
%

Logiker schreiben aussagenlogische Formeln in
der üblichen Infixnotation und lassen dabei Klammern weg, sofern
Mißverständnisse ausgeschlossen sind.
%
Als Konvention bezeichnen die Buchstaben $L, M, N$  Literale 
und $A, B, C, D, F, G$ Formeln.

Die möglichen Formeln der Sprache $\mathcal{L}_1$ sind gegenüber der
in Anhang~\ref{sec:aussagenlogik} beschriebenen Aussagenlogik eingeschränkt:
Negationen dürfen nur direkt vor Literalen stehen, d.h.\ Ausdrücke wie
$\neg(L\vee M)$ sind in $\mathcal{L}_1$ nicht enthalten.  Das bedeutet 
keine ernsthafte Einschränkung, denn alle Ausdrücke lassen sich unter Erhaltung 
ihres Wahrheitswerts
mit Hilfe der DeMorgan'schen Gesetze (siehe Anhang~\ref{sec:aussagenlogik})
so umformulieren, daß sie in $\mathcal{L}_1$ enthalten
sind (im Beispiel: $\neg L\wedge \neg M$; siehe auch
Aufgabe~\ref{ex:negation}). 

In \textit{Sequenzenkalkülen\index{Sequenzenkalkül}},
lassen sich Beweise besonders einfach finden.
Sequenzenkalküle beschäftigen sich statt mit einzelnen Formeln
allgemeiner mit Sequenzen daraus.
Eine \textit{Sequenz\index{Sequenz}} ist eine endliche Multimenge
(siehe Anhang~\ref{sec:multisets}) von Formeln in
$\mathcal{L}_1$, die durch Kommata getrennt
hintereinandergeschrieben werden, also z. B.\
\[P_1\wedge\neg P_2,\; P_2,\; P_2\vee\neg P_1\] 
Eine einzelne Formel wird mit der entsprechenden einelementigen Sequenz
identifiziert.  
Sequenzen werden mit $\Gamma, \Pi, \Delta$ etc.\ bezeichnet.  


\begin{definition}[Kalkül SC$_1$]\index{Kalkül}\index{Kalkül!SC$_1$}\index{SC1@SC$_1$}
Der Kalkül SC$_1$ ist gegeben durch die Sprache $\mathcal{L}_1$ und die
Axiome 
%
\begin{equation}
\top, \Gamma
\label{S1}
\end{equation}
\begin{equation}
\neg P_i, P_i, \Gamma
\label{S2}
\end{equation}
für alle Sequenzen $\Gamma$ und alle $i\in\{1,\ldots,N\}$, außerdem für alle
Sequenzen $\Gamma$ die Regeln
%
\begin{equation}
\begin{prooftree}
  A, B, \Gamma \justifies A\vee B, \Gamma
\end{prooftree}
\label{Sd}
\end{equation}
\begin{equation}
  \begin{prooftree}
    A, \Gamma \quad B, \Gamma \justifies A\wedge B, \Gamma
  \end{prooftree}
\label{Sc}
\end{equation}
%
\end{definition}

Mit Kalkülen wird rein syntaktisch, d. h. also ohne Bezug auf irgendeine
Semantik operiert.   Trotzdem sollen mit Hilfe eines Kalküls semantisch
korrekte Resultate entstehen.  Deshalb steht auch hinter SC$_1$ eine
semantische Vorstellung:
$\top$ (sprich: "<top">) und $\bot$ (sprich: "<bottom">) sind Symbole, die
für die Wahrheitswerte "<wahr"> bzw.\ "<falsch"> stehen.
Eine Sequenz wird als Disjunktion ihrer Formeln
betrachtet, d.h.\ die Sequenz $A, B, C$ steht für "<$A$ oder $B$
oder $C$">.  Die Axiome formalisieren folgende Einsichten:
(\ref{S1}) "<wahr oder $\Gamma$"> ist für alle $\Gamma$ wahr,
ebenso (\ref{S2}) für jede Aussage $P$ die Aussage "<$P$ oder nicht
$P$">.    

Die Regeln sind folgendermaßen zu lesen:
%
Über dem "<Bruchstrich">, der in diesem Zusammenhang \textit{Inferenzstrich\index{Inferenzstrich}}
genannt wird, befinden sich \emph{Voraussetzungen} oder \textit{Prämissen\index{Prämisse}}.  In
einem Beweis sind dies entweder Axiome oder "<schon bewiesene"> Sequenzen.
Unter dem Inferenzstrich befindet sich die \textit{Schlußfolgerung\index{Schlußfolgerung}} oder
\emph{Konsequenz\index{Konsequenz}} der Regel, die Sequenz, die bewiesen wird.  Für
$A$ und $B$ können beliebige Formeln eingesetzt werden.

Auch die Regeln formalisieren fundamentale Einsichten:
Ist "<$A$ oder $B$
oder $\Gamma$"> bewiesen (\ref{Sd}), so auch "<$A\vee B$ oder $\Gamma$">.  Für
einen Beweis von $A\wedge B$ muß es separate Beweise sowohl für $A$
als auch für $B$ geben
(\ref{Sc}). 

\begin{definition}[Beweis in SC$_1$] Ein \emph{Beweis} für eine Sequenz
  $\Gamma$ in SC$_1$ ist ein Baum, an dessen Wurzel $\Gamma$ und an dessen
  Blättern Axiome stehen und an dessen inneren Knoten Regeln angewendet werden.
  Eine Sequenz $\Gamma$ heißt \emph{beweisbar} in SC$_1$ genau dann, wenn es
  einen Beweis für $\Gamma$ gibt.  Dafür wird auch das folgende Zeichen
  verwendet: \index{Beweis}\index{Beweisbarkeit}\index{*@$\vdash$}
\begin{displaymath}
  \vdash \Gamma\index{*@$\vdash$}
\end{displaymath}
\end{definition}

Hier ist ein Beweis für die Formel
$(P_1\wedge P_2)\vee(\neg P_1\vee \neg P_2)$:
%
\begin{displaymath}
  \begin{prooftree}
    \[
    \[
    P_1, \neg P_1, \neg P_2~(\ref{S2})
    \qquad
    P_2, \neg P_1, \neg P_2~(\ref{S2})
    \justifies
    P_1\wedge P_2,\neg P_1,\neg P_2
    \using (\ref{Sc}) 
    \]
    \justifies
    P_1\wedge P_2,\neg P_1\vee\neg P_2
    \using (\ref{Sd})
    \]
    \justifies
    (P_1\wedge P_2)\vee(\neg P_1\vee\neg P_2)
    \using (\ref{Sd})
  \end{prooftree}
\end{displaymath}
%
Die verwendeten Axiome sowie Schlüsse sind durch ihre Nummern
gekennzeichnet.  Bei der Anwendung von (\ref{S2}) in der rechten Hälfte wird
deutlich, daß bei Sequenzen die Reihenfolge keine Rolle spielt.

Die Konstruktion eines Beweises geht
von der zu beweisenden Formel bzw.\ Sequenz aus und
versucht, diese durch Anwendung von Regeln auf die Axiome zurückzuführen.  Aus
einem "<Beweisziel"> entsteht somit nach und nach eine ganze Folge von
"<Unterzielen">, die zu erfüllen sind.  Es werden also bei der Konstruktion eines
Beweises die Inferenzregeln von unten nach
oben angewendet.

\section{Modelle für die Aussagenlogik}
\label{sec:models}

Die im vorangegangenen Abschnitt eingeführte Beweisbarkeit ist ein
syntaktisches Konzept: Hier wird mit Formeln gearbeitet, ohne daß deren
Bedeutung dabei eine Rolle spielt.  Der in
Abschnitt~\ref{sec:truth-provability} bereits angesprochene Ansatz der
Modelltheorie setzt dagegen voraus, daß jeder
Formel in der Sprache $\mathcal{L}_1$ eine Bedeutung gegeben wird:  

\begin{definition}[Modell der Aussagenlogik]
  Sei $\Gamma=A_1,\ldots,A_n$ eine Sequenz von aussagenlogischen Formeln $A_i\in
  \mathcal{L}_1$ über einem Variablenalphabet
  $\mathcal{V}$, $B$ die boolesche Algebra mit $\top_B=\mathrm{W}$ und $\bot_B=\mathrm{F}$, $f:\mathcal{V}\rightarrow 
  B$ eine Variablenbelegung.  Dann heißt $f$ ein \emph{Modell\index{Modell}} der
  Aussagenlogik und kann zu einem
  Homomorphismus $\hat{f}$ forgesetzt werden.

  FIXME: Homomorphismus

  Die Semantik einer einzelnen Formel $A_i$ ist dann gegeben
  durch $\hat{f}(A_i)$.  $\hat{f}$ wird auf $\Gamma$
  fortgesetzt durch
  \[ \widehat{\mbox{$f$}^*}(A_1,\dots,A_n) = \hat{f}(A_1)\vee\dots\vee\hat{f}(A_n)\]
\end{definition}

Mit diesen Grundlagen läßt sich nun folgendes definieren:

\begin{definition}
  Eine Sequenz $\Gamma$ heißt \emph{wahr\index{wahr} im Modell} $f$, geschrieben
  als
%
\(
  \models_f \Gamma\ ,\index{*@$\models_f$}
\)
genau dann, wenn gilt
\(\widehat{\mbox{$f$}^*}(\Gamma) = \mathrm{W}\).
%
$\Gamma$ heißt \emph{allgemeingültig\index{allgemeingültig}} oder eine \emph{Tautologie\index{Tautologie}}, in Zeichen
%
\(
  \models \Gamma\ ,\index{*@$\models$}
\)
%
genau dann, wenn $\Gamma$ in allen Modellen wahr ist.

\end{definition}

\section{Korrektheit, Konsistenz und Vollständigkeit}
\label{sec:kkv}

Interessant ist nun die Frage, wie die intuitive, semantische Vorstellung
von Wahrheit in der Modelltheorie und der syntaktische Begriff der
Beweisbarkeit zusammenhängen.  Idealerweise sollen diese natürlich
zusammenfallen, und im SC$_1$ ist dies der Fall.  
Die folgende 
Definition charakterisiert diese Beziehnungen:

\begin{definition}
%
Ein logischer Kalkül, der eine Negationsoperation $\neg$ hat, heißt 
\begin{itemize}
\item \textit{konsistent\index{konsistent}},
  wenn es keine Formel $A$ gibt mit $\vdash A$ und 
  $\vdash \neg A$,
\item \textit{korrekt\index{korrekt}}, wenn gilt:
  \(
    \textrm{falls } \vdash A \textrm{, so } \models A
    \),
\item  \textit{vollständig\index{vollständig}}, wenn gilt:
  \(
    \textrm{falls } \models A \textrm{, so } \vdash A
    \).
\end{itemize}
\end{definition}
                                

Logische Kalküle, die nicht konsistent sind, können auch nicht
korrekt sein, denn die Formel $A\wedge\neg A$ wäre in einem inkonsistenten
Kalkül beweisbar, aber sie kann in keinem Modell außer dem trivialen
("<einpunktigen">) Modell wahr sein, welches überhaupt nur einen einzigen Wert
beinhaltet, der $\top$ ebenso wie $\bot$ darstellt.
%
\begin{satz}
  SC$_1$ ist konsistent, korrekt und vollständig.
\end{satz}

\begin{beweis}
  \begin{description}
  \item[Korrektheit] Sei $f : V\rightarrow \{ \mathrm{W}, \mathrm{F}\}$ eine
    Variablenbelegung.  

    Für Axiom~\ref{S1} gilt 
    $\widehat{\mbox{$f$}^*}(\top,\Gamma)=\mathrm{W}\vee\widehat{\mbox{$f$}^*}(\Gamma)=\mathrm{W}$ und für Axiom~\ref{S2} gilt $\widehat{\mbox{$f$}^*}(P_i,\neg P_i,\Gamma)=f(P_i)\vee\neg f(P_i)\vee\widehat{\mbox{$f$}^*}(\Gamma)=\mathrm{W}$.
    
    Gelte nun $\widehat{\mbox{$f$}^*}(A, B, \Gamma)=\mathrm{W}$.  Dann muß ${\hat{f}}(A)=\mathrm{W}$,
    ${\hat{f}}(B)=\mathrm{W}$ oder
    $\widehat{\mbox{$f$}^*}(\Gamma)=\mathrm{W}$ gelten.
    Es gilt in jedem Fall
    ${\hat{f}}(A\vee B)=\mathrm{W}$ oder $\widehat{\mbox{$f$}^*}(\Gamma)=\mathrm{W}$.
    Regel (\ref{Sd}) erhält also die Wahrheit.
    
    Gelte nun $\widehat{\mbox{$f$}^*}(A, \Gamma)=\mathrm{W}$ und $\widehat{\mbox{$f$}^*}(B, \Gamma)=\mathrm{W}$.
    Wenn nun $\widehat{\mbox{$f$}^*}(\Gamma)=\mathrm{W}$ nicht gilt, so muß ${\hat{f}}(A)=\mathrm{W}$ und
    ${\hat{f}}(B)=\mathrm{W}$ gelten.  Dann gilt aber auch ${\hat{f}}(A\wedge B)=\mathrm{W}$.
    Auch die Regel (\ref{Sc}) erhält also die Wahrheit.
    
    In einem Beweisbaum haben deshalb zunächst alle Axiome~-- also alle
    Blätter~-- den Wahrheitswert $\mathrm{W}$.  Wahrheit vererbt sich in den Regeln
    von oben nach unten; durch strukturelle Induktion folgt also, daß auch
    die Sequenz ganz unten im Beweisbaum den Wahrheitswert $\mathrm{W}$ haben muß.
  \item[Vollständigkeit] Sei $\Gamma$ eine Sequenz mit
    $\widehat{\mbox{$f$}^*}(\Gamma)=\mathrm{W}$ für alle Variablenbelegungen $f$.
    Dann läßt sich mit $\Gamma$ als Wurzel ein 
    Beweisbaum konstruieren, indem systematisch alle Formeln durch Anwendung
    der Regeln (\ref{Sc}) und (\ref{Sd}) zerlegt werden, bis nur noch Literale
    übrig sind.
    Analog zum Beweis für die Korrektheit wird Wahrheit
    auch von unten nach oben vererbt, wie sich anhand der
    Regeln sehen läßt.
    
    Die Blätter des Baums haben also unter allen Variablenbelegungen den
    Wahrheitswert $\mathrm{W}$.  Angenommen, ein solches Blatt $\Delta$ hätte weder die
    Form (\ref{S1}) noch die Form (\ref{S2}).  Dann besteht es nur aus $\bot$
    und Literalen der Form $P_i$ und Literalen der Form $\neg P_j$, wobei
    für ein $i$ die Literale $P_i$ und $\neg P_i$ nicht gleichzeitig auftreten.
    ($\bot$ allein ist offensichtlich nicht möglich.)  Dann läßt sich aber eine
    Belegung $f'$ konstruieren mit $f'(P_i) = \mathrm{F}$ für
    $P_i\in \Delta$ und $f'(P_j) = \mathrm{W}$ für
    $\neg P_j\in \Delta$.  Für deren Fortsetzung $\widehat{\mbox{$f'$}^*}$ gilt dann
    $\widehat{\mbox{$f'$}^*}(\Delta)=\mathrm{F}$ im Widerspruch zur Voraussetzung.
  \item[Konsistenz] Sei $A$ eine Formel mit $\vdash A$ und $\vdash
    \neg A$.  Aus der Korrektheit folgt dann, daß für jede
    Belegung $f$ gilt ${\hat{f}}(A) = \mathrm{W} =
    {\hat{f}}(\neg A)$.  Das ist aber unmöglich, da 
    \[{\hat{f}}(\neg A)
    = \left\{\begin{array}{ll}
        \mathrm{F} & \mbox{\rm falls\ }{\hat{f}}(A)=\mathrm{W}\\
        \mathrm{W} & \mbox{\rm sonst}
      \end{array}\right.\]
    \end{description}
\end{beweis}
  

\section{Der Reduktionskalkül RC$_1$}
\label{sec:rc1}

SC$_1$ ist ein sogenannter \textit{Inferenzkalkül\index{Inferenzkalkül}}: Beweise sind Bäume, in
denen Anwendungen von Regeln durch Inferenzstriche getrennt sind.  Es gibt
jedoch noch andere Methoden, logische Kalküle aufzubauen.  Die sogenannten
\textit{Reduktionskalküle\index{Reduktionskalkül}} benutzen das Prinzip der algebraischen
Vereinfachung: Eine logische Formel wird schrittweise durch die Anwendung von
Gleichungen vereinfacht, wobei gelegentlich Teilterme durch andere, 
\emph{äquivalente} Terme ersetzt werden.  %Wir haben dies in einem
%vergangenen Abschnitt (Beispiel~\ref{bsp:cons}) bereits als "<Rechnen in einem
%Gleichungssystem"> informell ausgenutzt.
(Das Substitutionsmodell aus Abschnitt~\ref{sec:substitution-model}
ist damit auch eine spezielle Art Reduktionskalkül.)

Der Kalkül RC$_1$\index{RC1@RC$_1$} ist, wie SC$_1$, eine Formalisierung der Aussagenlogik und
benutzt die gleiche Sprache, $\mathcal{L}_1$.  Kommt am Ende $\top$ heraus, ist
die Formel eine Tautologie.  Hier sind die Regeln von RC$_1$:
%
\begin{definition}[Regeln von RC$_1$]\index{*@$\triangleright$}
\begin{eqnarray}
    \top\vee A\triangleright \top &\qquad& A\vee\top \triangleright\top\label{eq:e1}\\%\tag{e1}\\
    \top\wedge A\triangleright A &\qquad& A\wedge\top\triangleright A\label{eq:e2}%\tag{e2}
\end{eqnarray}
\begin{equation}
  L\vee A_1\vee\ldots \vee A_n\vee\neg L\vee B\:\triangleright\: \top\label{eq:t}%\tag{t}
\end{equation}
\begin{equation}
  (A\vee B)\vee C\:\triangleright\: A\vee(B\vee C)\label{eq:a}%\tag{a}
\end{equation}
\begin{eqnarray}
  (A\wedge B)\vee C&\triangleright& (A\vee C)\wedge (B\vee C)\label{eq:d1}\\%\tag{d1} \\
  C\vee(A\wedge B)&\triangleright& (C\vee A)\wedge (C\vee B)\label{eq:d2}%\tag{d2}
\end{eqnarray}
\end{definition}
%
Eine logische Formel läßt sich
mit Hilfe einer Reduktionsregel dann vereinfachen, wenn sie der Form
der linken Seite der Regel entspricht.  (Formal gesagt muß die Formel
auf die linke Seite einer Regel \textit{passen\index{paßt auf}}.)  
Sie heißt dann ein \emph{Redex\index{Redex}} (aus
\emph{reducible expression} abgekürzt) und wird durch die
Entsprechung der rechten Regelseite ersetzt.  Hier ein Beispiel:
%
\begin{displaymath}
  E_1 \deq{} (P_1\wedge P_2)\vee (\neg
    P_1\vee \neg P_2)
\end{displaymath}
%
Diese Formel hat die Form der linken Seite von Regel~(\ref{eq:d1}) und läßt
sich folgendermaßen mit $A=P_1$,
$B=P_2$ und $C=(\neg P_1\vee\neg{P_2})$
reduzieren:
%
\begin{displaymath}
  E_2 \deq{} (P_1\vee(\neg P_1\vee\neg{P_2}))\wedge 
  (P_2\vee(\neg P_1\vee\neg{P_2}))
\end{displaymath}
%
Auf
diese Art und Weise bildet $\triangleright$ eine Relation auf
$\mathcal{L}_1$.
Es ist leicht zu beweisen~-- z.B.\ in SC$_1$~-- daß bei den
Reduktionsregeln jeweils linke und rechte Seite äquivalent sind.

Leider läßt sich $E_2$ nicht weiter reduzieren~-- sie paßt auf
keine linke Regelseite.  Das ist insbesondere deswegen bedauerlich, weil es sich
um eine Tautologie handelt.  Jedoch läßt sich $E_2$ als $E_2 = E_2^{(1)} \wedge
E_2^{(2)}$ schreiben und beide \textit{Teilformeln} $E_2^{(1)}$ und $E_2^{(2)}$
passen auf die Regel~(\ref{eq:t}).  Nach Anwendung von Regel~(\ref{eq:t}) auf die beiden
Teilformeln kommt folgende Formel heraus:
%
  \begin{displaymath}
    E_3 \deq{} \top \wedge \top
  \end{displaymath}
%
  Diese läßt sich wiederum mit Regel~(\ref{eq:e2}) zu $\top$ reduzieren,
  und fertig ist der Beweis.

Zu $\triangleright$ gehört eine abgeleitete
Relation $\blacktriangleright$, die auch auf
Subtermen von Formeln arbeiten kann, also auch Redexe im Inneren einer Formel
finden kann:
%
\begin{definition}[Erweiterung von $\triangleright$ auf Subterme]
  $\blacktriangleright$\index{*@$\blacktriangleright$} ist die \textit{Erweiterung von $\triangleright$ auf
    Subterme}: Sei $A$ eine Formel aus $\mathcal{L}_1$, in der eine andere
  Formel $B$ an einer bestimmten Stelle vorkommt.  Sei $C$ eine Formel, für die
  $B\triangleright C$ gilt.  Wenn
  $A\blacktriangleright D$ gilt, entsteht die Formel $D$ dadurch, daß in $A$
  das Vorkommen von $B$ durch $C$ ersetzt wird.  
\end{definition}
%
Abbildung~\ref{fig:reduction} zeigt die Situation: Bei der Anwendung
der Reduktionsregel $\blacktriangleright$ auf einen Term bleibt der
umschließende, weiße Teil des Terms unverändert, nur der Rest wird
gemäß $\triangleright$ reduziert und ersetzt.

\begin{figure}[tb]
  \begin{center}
    {
      \psfrag{W}{$\blacktriangleright$}
      \psfrag{P}{$\triangleright$}
      \includegraphics[width=0.6\textwidth]{reduction.eps}
      }
    \caption{Reduktion auf Subtermen}
    \label{fig:reduction}
  \end{center}
\end{figure}


Nun läßt sich Beweisbarkeit in RC$_1$ definieren:
%
\begin{definition}[Beweisbarkeit in RC$_1$]
  Eine Formel $A$ ist \textit{in RC$_1$ beweisbar} (geschrieben
  $\mathrm{RC}_1 \vdash A$ oder einfach nur $\vdash A$), wenn
  $A\blacktriangleright^\ast\top$ gilt.\index{*@$\vdash$}
\end{definition}
%
Dabei ist $\blacktriangleright^\ast$ der
transitiv-reflexive Abschluß von $\blacktriangleright$.  (Siehe dazu
  Definition~\ref{def:relation-closure} in
  Anhang~\ref{sec:relationen}.)

Diese drei Zutaten machen einen Reduktionskalkül aus:
Reduktionsregeln, Erweiterung auf Subterme und reflexiv-transitiver
Abschluß.

\begin{satz}
  RC$_1$ ist korrekt, konsistent und vollständig.
\end{satz}

\begin{beweis}

Für $A\blacktriangleright B$ sind $A$ und $B$ immer
äquivalent.  Daraus folgen Konsistenz und Korrektheit.

Der Beweis für die Vollständigkeit ist etwas aufwendiger und wird
darum nur skizziert.  Prinzipiell schwierig ist er allerdings nicht:

Die Reduktionsregeln~\ref{eq:a}, \ref{eq:d1} und \ref{eq:d2} für sich
gesehen überführen eine Formel in die folgende Form:
%
\begin{eqnarray*}
  &&(L_1^{(1)}\vee (L_2^{(1)} \vee(\ldots \vee L_{k_1}^{(1)})\ldots))\\
  \wedge&&\ldots\\
  \wedge&&\ldots\\
\wedge&&
  (L_1^{(n)}\vee (L_2^{(n)} \vee(\ldots \vee L_{k_n}^{(n)})\ldots))
\end{eqnarray*}
%
(Diese Form heißt auch \textit{konjunktive Normalform\index{Normalform!konjunktiv}\index{konjunktive Normalform}}.)  Eine Formel
dieser Form ist genau dann wahr, wenn alle Teilformeln
%
\begin{displaymath}
(L_1^{(j)}\vee (L_2^{(j)} \vee(\ldots \vee L_{k_j}^{(j)})\ldots))
\end{displaymath}
%
wahr sind.  Eine solche Teilformel ist genau dann wahr, wenn sie
entweder $\top$ enthält oder zwei Literale $L_l^{(j)}$ und $\neg L_m^{(j)}$
mit $L_l^{(j)} = L_m^{(j)}$.  Genau diese beiden Fälle werden
aber von den Regeln~(\ref{eq:e1}) und (\ref{eq:t}) abgedeckt, so daß sich
diese Teilformeln zu $\top$ reduzieren lassen.  Regel~(\ref{eq:e2})
besorgt dann den Rest.
\end{beweis}
                                
%\section*{Anmerkungen}
%Im allgemeinen ist Beweisbarkeit ein schwächeres Konzept als Allgemeingültigkeit, sobald
%einigermaßen leistungsfähige mathematische Systeme studiert werden.

\section*{Aufgaben}

\begin{aufgabe}
  Zeige mit Hilfe der Wahrheitstafeln aus
  Anhang~\ref{sec:aussagenlogik}, wie sich aussagenlogische
  Formeln, die Implikation $\Rightarrow$ enthalten, in äquivalente
  Formeln ohne $\Rightarrow$ übersetzen lassen.
\end{aufgabe}

\begin{aufgabe}
  \textit{Gant\=os Axt\index{Gantos Axt@Gant\=os Axt}} ist ein Zen-Koan folgenden Inhalts:
  %
  \begin{quote}
    Eines Tages sagte Tokusan zu seinem Schüler Gant\=o: "<Ich habe
    zwei Mönche, die schon seit vielen Jahren hier sind.  Geh hin und
    prüfe sie."> Gant\=o nahm eine Axt und begab sich zu der Hütte, in
    der die zwei Mönche meditierten.  Er hob die Axt und sprach:
    "<Wenn ihr ein Wort sagt, so werde ich euch die Köpfe abhauen, und
    wenn ihr kein Wort sagt, werde ich euch ebenfalls die Köpfe
    abhauen.">
  \end{quote}
  %
  Werden den Mönchen die Köpfe abgehauen?  Beweise deine Antwort in
  SC$_1$!
\end{aufgabe}

\begin{aufgabe}
  Beweise folgende Formeln in SC$_1$!

    \begin{displaymath}
      \begin{array}{c}
      (P_3\Rightarrow(P_1\wedge P_2))\Leftrightarrow (P_3\Rightarrow P_1)\wedge
      (P_3\Rightarrow P_2)
      \\[1.5ex]
      ((P_1 \Leftrightarrow P_2)\wedge(P_3\Leftrightarrow
      P_4))\Rightarrow((P_1\wedge P_3)\Leftrightarrow (P_2\wedge P_4))
    \end{array}
  \end{displaymath}
\end{aufgabe}

\begin{aufgabe}\label{ex:negation}
 Eine aussagenlogische Formel mit Variablen $\neg
  F$, wobei $F$ selbst kein $\neg$ enthält, ist äquivalent zu $\eta(F)$, wobei $\eta(F)$ aus $F$ dadurch
  entsteht, daß alle aussagenlogischen Konstanten und Variablen umgedreht werden
  ($W\mapsto F, F\mapsto W$, $\neg X\mapsto X$, $X \mapsto \neg X$)
  und jeweils $\vee$ durch  $\wedge$ und umgekehrt ersetzt wird.  Zum
  Beispiel ist also $\neg(A\wedge(B\vee \neg C)) \equiv \neg A\vee (\neg B\wedge C)$.

  \begin{itemize}
  \item Schreibe eine induktive Definition für $\eta$.
  \item Beweise die Behauptung mittels struktureller Induktion und den
    DeMorgan'schen Gesetzen (Anhang~\ref{sec:aussagenlogik}).
    \item Beweise in SC$_1$, daß $A,\eta(A)$ für
    jede Formel $A$ beweisbar ist.
    Benutze dazu strukturelle Induktion über
    $A$!
  \end{itemize}
\end{aufgabe}

\begin{aufgabe}\label{ex:sc1-scheme}
  Programmiere Abstraktionen für den Umgang mit
  $\mathcal{L}_1$-Formeln!
  \begin{enumerate}
  \item Schreibe eine Datendefinition für Formeln und dazu passende
    Record-Definitionen.
  \item Schreibe eine Funktion \texttt{write-L1}, welche
    $\mathcal{L}_1$-Formeln in lesbarer und vollständig geklammerter
    Form ausdruckt!  Dabei soll \texttt{T} für $\top$, \texttt{B} für
    $\bot$, \texttt{!} für die
    Negation, \texttt{\&} für die Konjunktion, und \verb&|& für die
    Disjunktion stehen:
\begin{alltt}
(write-L1
 (make-conjunction (make-negation (make-variable 1))
                   (make-disjunction (make-constant #t)
                                     (make-constant #f)))))
\prints{} (!P1&(T|B))
\end{alltt}
  \end{enumerate}
\end{aufgabe}

\begin{aufgabe}
  Basierend auf den Abstraktionen für
  $\mathcal{L}_1$-Formeln von Aufgabe~\ref{ex:sc1-scheme}, schreibe ein Programm, das
  für Formeln herausfindet, ob sie in SC$_1$ beweisbar sind oder
  nicht!

  \begin{enumerate}
  \item Schreibe ein Prädikat \texttt{literal?}, das feststellt, ob
    eine Formel die Form $\top$, $\bot$, $P_i$ oder
    $\neg P_i$ hat.
  \item Benutze für die Repräsentation von Sequenzen Listen von
    Formeln. Schreibe ein Prädikat \texttt{axiom?}, das feststellt, ob
    eine gegebene Sequenz ein Axiom ist, also ob $\top$ in der Liste
    vorkommt, oder zu einer Variable die entsprechende negierte
    Variante.

    Hinweis: Eine Hilfsfunktion \texttt{variable-member? }ist
    hilfreich, die  ein Vorzeichen und den
    Namen einer Variablen sowie eine Sequenz akzeptiert, und feststellt,
    ob die entsprechende Variable in der Liste vorkommt.
\begin{alltt}
(define f1 (make-conjunction (make-constant #t) (make-variable 1)))
(define f2 (make-conjunction
             (make-disjunction (make-negation (make-variable 1))
                               (make-variable 2))
             (make-variable 3)))
(define f3 (make-variable 1))
(define f4 (make-negation (make-variable 1)))
(define s1 (list f1 f2 f3))
(variable-member? #t 1 s1)
\evalsto{} #t
(variable-member? #t 2 s1)
\evalsto{} #f
(variable-member? #f 1 s1)
\evalsto{} #f
(axiom? s1)
\evalsto{} #f
(define s2 (list f1 f2 f3 f4))
(axiom? s2)
\evalsto{} #t
\end{alltt}
    

  \item Schreibe eine Funktion \texttt{reorder-sequence}, welche eine
    Sequenz so umsortiert, daß ein Nichtliteral vorn steht, oder
    \texttt{\#f} zurückgibt, falls das nicht möglich ist.
    
    Anleitung: Benutze dazu eine endrekursive mit \texttt{letrec}
    gebundene interne Hilfsfunktion, welche die Sequenz absucht und
    dabei die Listenelemente vor dem Nichtliteral 
    aufsammelt.
\begin{alltt}
(define s3 (list f3 f1 f4 f2))
(for-each (lambda (formula)
              (write-L1 formula)
              (write-newline))
            s3)
\prints{} P1
\prints{} (T&P1)
\prints{} (!P1)
\prints{} (((!P1)|P2)&P3)
(define s4 (reorder-sequence s3))
(for-each (lambda (formula)
              (write-L1 formula)
              (write-newline))
            s4)
\prints{} (T&P1)
\prints{} P1
\prints{} (!P1)
\prints{} (((!P1)|P2)&P3)
\end{alltt}

  \item Schreibe nun eine Funktion \texttt{tautology?}, welche eine
    Sequenz als Parameter akzeptiert, und feststellt, ob sie beweisbar
    ist.  Diese Funktion sollte die Sequenz entweder als Axiom
    identifizieren, oder durch Zerlegung eines Nichtliterals (durch
    \texttt{reorder"=sequence} nach vorn sortiert) eine oder mehrere
    neue zu beweisende Sequenzen erzeugen, auf denen dann
    weiterbewiesen wird.
\end{enumerate}
\end{aufgabe}

\begin{aufgabe}
  Beweise, daß folgende Axiome bzw.\ Regeln zu
  SC$_1$ hinzugefügt werden können, ohne daß der Kalkül mächtiger
  wird:
  %
    \begin{displaymath}
      \begin{prooftree}
        A\wedge B,\Gamma \justifies A,\Gamma
      \end{prooftree}
      \qquad
      \begin{prooftree}
        A\wedge B,\Gamma \justifies B,\Gamma
      \end{prooftree}
      \qquad
      \begin{prooftree}
        A\vee B,\Gamma \justifies A, B,\Gamma
      \end{prooftree}
  \end{displaymath}
  %
  Betrachte dazu einen fiktiven Kalkül SC$_1'$, der diese zusätzlichen
  Regeln enthält.  Gib nun eine Anleitung an, wie sich ein Beweisbaum
  in SC$_1'$ in einen Beweisbaum in SC$_1$ übersetzen läßt, so daß die
  zusätzlichen Regeln dort nicht mehr vorkommen.
\end{aufgabe}

\begin{aufgabe}
  Betrachte MPC$_0$\index{MPC0@MPC$_0$}, einen Inferenzkalkül für
  \textit{intuitionistische Aussagenlogik\index{Aussagenlogik!intuitionistisch}\index{intuitionistische Aussagenlogik}}.
  
  Die Sprache $\mathcal{L}_0$ des Kalküls besteht aus
  Aussagenvariablen $\{P_1, \ldots, P_N\}$ für
  $N>0$, der aussagenlogischen Konstante $\bot$, sowie einem einzelnen
  zweistelligen Junktor $\rightarrow$.  Es gibt keine Sequenzen; der
  Kalkül beschäftigt sich nur mit Formeln.  $A$, $B$ und $C$ seien
  jeweils Formeln.  Hier sind die Axiome:
  %
  \begin{displaymath}
    \begin{array}{c}
      A\rightarrow(B\rightarrow A)
      \\
      (A\rightarrow(B\rightarrow C))\rightarrow ((A\rightarrow
      B)\rightarrow (A\rightarrow C))
      \\
      \bot\rightarrow A
    \end{array}
  \end{displaymath}
  %
  Es gibt nur eine Regel namens \textit{modus ponens\index{modus ponens}}:
  %
  \begin{displaymath}
    \begin{prooftree}
       A\qquad A\rightarrow B \justifies B
    \end{prooftree}
  \end{displaymath}
  %
  \begin{enumerate}
  \item Erweitere, wie bei der Modelltheorie für SC$_1$, eine
    Variablenbelegung sinnvoll zu einem Homomorphismus von
    $\mathcal{L}_0$ nach $\{\mathrm{W}, \mathrm{F}\}$, so daß alle
    Axiome den Wahrheitswert W bekommen und außerdem die Modus-Ponens-Regel
    gültig ist.  Gleichzeitig sollen nicht alle Formeln den
    Wahrheitswert W bekommen.
  \item Schreibe einen Beweisbaum für $A\rightarrow
    A$.
  \item Die \textit{relativierte Herleitbarkeit\index{Herleitbarkeit!relativiert}\index{relativierte Herleitbarkeit}} in MPC$_0$ ist
    folgendermaßen definiert:

    Es sei $M =\{G_1, \ldots, G_n\}$ für $n\geq 0$ eine endliche
    Formelmenge in $\mathcal{L}_0$.  Eine Formel $F$ aus
    $\mathcal{L}_0$ ist in MPC$_0$ \textit{relativ zu $M$} herleitbar,
    wenn es einen Beweisbaum für $F$ gibt, in dem oben neben den
    Axiomen auch Formeln aus $M$ stehen können.  Schreibweise:
    %
    \(M \vdash F\),
    %
    Beispiel:
    %
    \(\{A,A\rightarrow B\} \vdash B\)

    Nun gilt in MPC$_0$ der sogenannte \textit{Deduktionssatz\index{Deduktionssatz}}.  Er
    lautet:
    %
    \begin{displaymath}
      \{G\}\cup M\vdash F \textrm{~gdw.~} M\vdash G\rightarrow F
    \end{displaymath}
    %
    Vervollständige den folgenden Beweis für den Deduktionssatz:

    "<$\Leftarrow$"> Es gilt, einen Beweis für $F$ zu konstruieren, in
    dem $G$ oben vorkommen darf.  Hier ist er:
    %
    \begin{displaymath}
      \begin{prooftree}
        G \qquad G \rightarrow F \justifies F
      \end{prooftree}
    \end{displaymath}

    "<$\Rightarrow$"> Induktion über die Höhe des Beweisbaumes:  Es
    sei also $T$ der Beweisbaum für $G\rightarrow F$, an dem oben die
    Formeln $G, G_1, \ldots, G_n$ (bei $M=\{G_1, \ldots, G_n\}$) sowie
    die Axiome auftreten können.
    
    Angenommen, $T$ hat die Höhe 1.  $F$ ist also bereits der Beweis
    selbst.  Dann muß $F$ ein Axiom oder ein Element aus $\{G\}\cup M$
    sein.  Es gilt dann also, eine Herleitung für $G\rightarrow F$ zu
    konstruieren.  Es gibt jetzt zwei Möglichkeiten: entweder ist
    $F=G$ oder nicht.  Schreibe Beweisbäume für $G\rightarrow F$
      für beide Fälle!

    Angenommen, $T$ hat eine Höhe $>1$.  Dann muß ganz unten in $T$
    eine Anwendung der Modus-Ponens-Regel stehen, die so aussieht:
    %
    \begin{displaymath}
      \begin{prooftree}
        {H \qquad H\rightarrow F} \justifies {F}
      \end{prooftree}
    \end{displaymath}
    %
    Nun kann ein Beweisbaum für $G\rightarrow F$ konstruiert werden:
    %
    \begin{displaymath}
      \begin{prooftree}
        \[
        {H \qquad H\rightarrow (G\rightarrow H)} \justifies
        G\rightarrow H
        \]
        \qquad
        \[ \cdots \justifies {(G\rightarrow
        H)\rightarrow (G\rightarrow F)} \]
        \justifies {G\rightarrow F}
      \end{prooftree}
    \end{displaymath}
    %
    Vervollständige den Beweisbaum!  Hinweis: Es sind nur noch
    zwei Anwendungen der Modus-Ponens-Regel sowie zwei Anwendungen von
    Axiomen notwendig!

    Wo und wie kommt die Induktionsvoraussetzung ins Spiel?
\item Benutze den Deduktionssatz, um
\begin{displaymath}
      \begin{array}{c}
        \vdash (A\rightarrow
    B)\rightarrow ((B\rightarrow C)\rightarrow(A\rightarrow C))\\
        \vdash (A\rightarrow B)\rightarrow((C\rightarrow
        A)\rightarrow(C\rightarrow B))
      \end{array}
    \end{displaymath}
    zu
    beweisen!
  \end{enumerate}
\end{aufgabe}

%%% Local Variables: 
%%% mode: latex
%%% TeX-master: "i1"
%%% End: 
