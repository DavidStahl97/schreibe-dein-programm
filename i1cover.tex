\documentclass{tueteubner}

\usepackage[latin1]{inputenc}
\pagestyle{empty}

\begin{document}

\sffamily

\centerline{\Large Herbert Klaeren \qquad Michael Sperber}

\medskip

\centerline{\textbf{\huge Die Macht der Abstraktion}}

\medskip

\centerline{\Large Einf�hrung in die Programmierung}

\medskip

\textit{Die Macht der Abstraktion} ist eine Einf�hrung in die
Entwicklung von Programmen und die dazugeh�rigen formalen Grundlagen.
Im Zentrum stehen \textit{Konstruktionsanleitungen}, welche die
systematische Konstruktion von Programmen f�rdern, sowie Techniken zur
\textit{Abstraktion}, welche die Umsetzung der
Konstruktionsanleitungen erm�glichen.  In der Betonung systematischer
Konstruktion unterscheidet sich dieses Buch drastisch von den meisten
anderen Einf�hrungen in die Programmierung.

Die vermittelten Grundlagen und Techniken sind unabh�ngig von einer
bestimmten Programmiersprache.  Zur Illustration und zum Training der
Programmierung dient \textit{Scheme}, eine kleine und leicht
erlernbare Programmiersprache, die es erlaubt, die Konzepte der
Programmierung zu pr�sentieren, ohne Zeit mit der Konstruktvielfalt
anderer Programmiersprachen zu verlieren.  Entsprechend vermittelt
dieses Buch fortgeschrittene Techniken.  Scheme-K�nner sind in der
Lage, andere Programmiersprachen in k�rzester Zeit zu erlernen.

\textit{Die Macht der Abstraktion} ist aus der Praxis der
Informatik-Grundausbildung an der Universit�t T�bingen entstanden:
�ber mehrere Vorlesungszyklen wurden Stoffauswahl und Pr�sentation
stetig verbessert.  Gegen�ber dem Vorg�ngerbuch \textit{Vom
  Problem zum Programm} wurde ein Gro�teil des Materials neu
entwickelt.  Das Buch enth�lt viele
Beispiele und �bungsaufgaben.  Alle n�tigen mathematischen
Grundlagen werden vermittelt.

\subsubsection*{Inhalt}~

\vspace*{-0.5ex}

\setlength\itemsep{-4pt}
\medskip

\begin{minipage}[t]{0.5\textwidth}
\begin{itemize}
\item Was ist Informatik?
\item Elemente des Programmierens
\item Fallunterscheidungen und Verzweigungen
\item Zusammengesetzte und gemischte Daten
\item Induktive Definitionen
\item Rekursion
\item Praktische Programme mit Listen
\item Higher-Order-Programmierung
\end{itemize}
\end{minipage}
\qquad
\begin{minipage}[t]{0.5\textwidth}
\begin{itemize}
\item Zeitabh�ngige Modelle
\item Abstrakte Datentypen
\item Bin\"are B\"aume
\item Zuweisungen und Zustand
\item Objektorientiertes Programmieren
\item Logische Kalk\"ule
\item Der $\lambda $-Kalk\"ul
\item Interpretation von Scheme
\end{itemize}
\end{minipage}

\subsubsection*{Zielgruppen}~

\vspace*{-2ex}

\begin{itemize}
\item Studierende der Informatik im Haupt- und Nebenfach an
  Fachhochschulen und Universit�ten
\item Studierende, die einen spannenden Einstieg in die Programmierung
  suchen
\end{itemize}

\vspace{-2.5ex}

\subsubsection*{Die Autoren}~

\vspace*{-0.5ex}

Prof.~Dr.~Herbert Klaeren, Universit�t T�bingen\\
Dr.~Michael Sperber, freiberuflicher Software-Entwickler

\vfill

\end{document}
