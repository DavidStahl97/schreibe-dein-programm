% Diese Datei ist Teil des Buchs "Schreibe Dein Programm!"
% Das Buch ist lizensiert unter der Creative-Commons-Lizenz
% "Namensnennung - Weitergabe unter gleichen Bedingungen 4.0 International (CC BY-SA 4.0)"
% https://creativecommons.org/licenses/by-sa/4.0/deed.de

\chapter{Mathematische Grundlagen}
\label{cha:math}

Dieser Anhang erläutert
die in diesem Buch verwendeten Begriffe und Notationen
aus der Mathematik.

\section{Aussagenlogik}
\label{sec:aussagenlogik}

\index{Aussagenlogik}Eine \emph{Aussage\index{Aussage}} ist ein Satz, der prinzipiell einen \emph{Wahrheitswert\index{Wahrheitswert}}
W\index{W} (für "<wahr\index{wahr}">) oder F\index{F} (für "<falsch\index{falsch}">)
besitzt.
%kann. Wir verwenden eine \emph{zweiwertige Logik} mit den Wahrheitswerten

%Beispiele für Aussagen und ihre Wahrheitswerte:
%brauchen wir nicht mehr, da i1calc das jetzt tut...
%\begin{enumerate}
%\item 6 ist eine Primzahl (F).
%\item 1024 ist eine Zweierpotenz (W).
%\item Es gab früher Leben auf der Venus (unbekannt, aber jedenfalls W oder F).
%\end{enumerate}

%Keine Aussage in diesem Sinne ist der Satz: "<Der Wahrheitswert dieses Satzes
%ist F">, denn es ist offensichtlich unmöglich, diesem Satz einen Wahr\-heits\-wert
%W oder F zuzuordnen.

Aus \emph{primitiven (elementaren) Aussagen}
\index{primitive Aussage}
\index{Aussage!primitiv}
\index{Aussage!elementar}
\index{elementare Aussage}
werden mit Hilfe sogenannter 
aussagenlogischer \emph{Junktoren\index{Junktor}} zusammengesetzte Aussagen
aufgebaut.  Es gibt zwei vordefinierte primitive Aussagen $\top$\index{*@$\top$} und
$\bot$\index{*@$\bot$} mit den Wahrheitswerten W bzw.\ F.  Die
wichtigsten Junktoren sind:
\begin{description}
\item["<und"> ($\wedge$):] $a \wedge b$ \index{*@$\wedge$}
        hat den Wahrheitswert W genau dann, wenn
     $a$ und $b$ beide den Wert W haben.
\item["<oder"> ($\vee$):] $a \vee b$ \index{*@$\vee$}
        hat den Wahrheitswert W genau dann, wenn von
     $a$ und $b$ mindestens eins den Wert W hat.
\item["<nicht"> ($\neg$):] $\neg a$ \index{*@$\neg$}
        hat den Wahrheitswert W genau dann, wenn $a$   
     den Wert F hat.
\end{description}

In der Aussagenlogik gilt %demnach HK: "demnach" hier sinnlos
das Prinzip, daß der Wahr\-heits\-wert 
einer
zusammengesetzten Aussage durch die Wahrheitswerte seiner Bestandteile
bestimmt ist.

Statt $\neg a$ wird gelegentlich auch $\overline{a}$ geschrieben.

Meistens werden logische Junktoren durch sogenannte \emph{Wahrheitstafeln}
\index{Wahrheitstafel} definiert:
%
\begin{center}
\begin{tabular}{c||c|c}
$\wedge$   &  W   &  F \\ \hline \hline
    W      &  W   &  F \\ \hline
    F      &  F   &  F
\end{tabular}
\hspace{3em}
\begin{tabular}{c||c|c}
$\vee$     &  W   &  F \\ \hline \hline
    W      &  W   &  W \\ \hline
    F      &  W   &  F 
\end{tabular}
\hspace{3em}
\begin{tabular}{c||c}
$\neg$     \\ \hline \hline
    W      &  F \\ \hline
    F      &  W
\end{tabular}
\end{center}
%
Andere Junktoren, die ebenfalls häufig verwendet werden, sind:
\begin{description}
\item["<impliziert"> ($ \Rightarrow $):]\index{*@$\Rightarrow$}
\index{Implikation}%
\hspace{2em}\begin{tabular}{c||c|c}
$ \Rightarrow $     &  W   &  F \\ \hline \hline
    W      &  W   &  F \\ \hline
    F      &  W   &  W 
\end{tabular}
\end{description}

$a \Rightarrow b$ spricht sich als "<wenn $a$, dann $b$"> oder "<aus $a$
folgt $b$">.  F $\Rightarrow$ W besitzt ebenso wie F $\Rightarrow$ F
den Wahrheitswert W! In der formalen Aussagenlogik folgt
aus einer falschen Voraussetzung jede Folgerung.  %Zum Ableiten
%wahrer Aussagen aus anderen wahren Aussagen verwendet man deshalb den \emph{Modus ponens:} Wenn $a$ und $a\Rightarrow b$ beide den Wahrheitswert W
%haben, dann kann man den Wahrheitswert W für $b$ folgern.

\begin{description}
\item["<genau-dann-wenn"> ($\Leftrightarrow$):]
\hspace{2em}\begin{tabular}{c||c|c}
$\Leftrightarrow$     &  W   &  F \\ \hline \hline
    W      &  W   &  F \\ \hline
    F      &  F   &  W 
\end{tabular}\index{*@$\Leftrightarrow$}
\end{description}

Häufig werden Wahrheitstafeln auch in einer etwas
ausführlicheren Form notiert, wie im folgenden gezeigt.  Dabei sind die
Wahrheitstafeln für alle vorgestellten Junktoren in einer Tabelle
zusammengefaßt:

\begin{center}
\begin{tabular}{c|c||c|c|c|c|c}
$a$ & $b$ & $a\wedge b$ & $a\vee b$ & $\neg a$ & $a\Rightarrow b$ & $a\Leftrightarrow b$\\ \hline\hline
W   &  W  &     W       &     W     &    F     &         W        & W\\ \hline
W   &  F  &     F       &     W     &    F     &         F        & F\\ \hline
F   &  W  &     F       &     W     &    W     &         W        & F\\ \hline
F   &  F  &     F       &     F     &    W     &         W        & W
\end{tabular}
\end{center}
%
Zur Einsparung von Klammern wird vereinbart, daß $\neg$ am stärksten 
\label{prio}
bindet, gefolgt von $\wedge$, dann $\vee$, dann $\Rightarrow$ und zum Schluß 
$\Leftrightarrow$.

Eine zusammengesetzte Aussage heißt \emph{allgemeingültig} oder eine
\emph{Tautologie}, \index{Tautologie}\index{allgemeingültig}
wenn sie stets den Wahrheitswert W besitzt, unabhängig vom
Wahr\-heits\-wert ihrer elementaren  Aussagen. Beispiele für Tautologien sind
etwa $a \vee \overline{a}$ ("<Satz vom ausgeschlossenen Dritten">)
und $\overline{a \wedge \overline{a}}$ ("<Satz vom \index{Widerspruch}Widerspruch">). Zwei 
Aussagen
$a$ und $b$ heißen \emph{äquivalent}, wenn $a \Leftrightarrow b$ eine
Tautologie ist.

Es ist möglich, jede aussagenlogische Aussage durch
Wahrheitstafeln auf ihre Allgemeingültigkeit hin zu überprüfen.  Auch
die Äquivalenz von Ausdrücken läßt sich durch Wahrheitstafeln
überprüfen.  In der Regel ist es jedoch einfacher, mit diesen
Aussagen formal zu rechnen.
Die folgenden Tautologien stellen Rechenregeln für die Aussagenlogik
dar:

\begin{lemma}\label{rechenregeln}Für Aussagen $a,b,c$ gilt:
%
\[\begin{array}{llll}
a \wedge a \Leftrightarrow a & \hspace{2.5em} & a \vee a \Leftrightarrow a &\textrm{Idempotenzgesetze}\index{Idempotenzgesetz}\\
(a \wedge b) \wedge c \Leftrightarrow a \wedge (b \wedge c) & & (a \vee b) \vee
c \Leftrightarrow a \vee (b \vee c) &\textrm{Assoziativgesetze}\index{Assoziativgesetz}\\
a \wedge b \Leftrightarrow b \wedge a & & a \vee b \Leftrightarrow b \vee a &\textrm{Kommutativgesetze}\index{Kommutativgesetz}\\
a \wedge (a \vee b) \Leftrightarrow a & & a \vee (a \wedge b) \Leftrightarrow a &\textrm{Absorptivgesetze}\index{Absorptivgesetz}\\
a \wedge (b \vee c) \Leftrightarrow (a \wedge b) \vee (a \wedge c)
     & & a \vee (b \wedge c) \Leftrightarrow (a \vee b) \wedge (a \vee c) &\textrm{Distributivgesetze}\index{Distributivgesetz}\\
\overline{a \wedge b} \Leftrightarrow \overline{a} \vee \overline{b} & 
       & \overline{a \vee b} \Leftrightarrow \overline{a} \wedge \overline{b}
       &\textrm{DeMorgan'sche Gesetze\index{DeMorgan'sches Gesetz}}\\
       \overline{\overline a} \Leftrightarrow a
\end{array}\]
\end{lemma}

\section{Mengen}
%% Güntzer: besser direkt Multimengen.  Eine interessante Literaturstelle
%% dazu:
%% Banatre, Metayer: Programming by Multiset Transformation, CACM 36(1), 
%% 98-111

Die ursprüngliche Definition des Begriffs "<Menge\index{Menge}">
lautet: 

%Eine ausführliche Behandlung der Mengenlehre 
%auf naiver, also nicht axiomatischer Basis findet sich in \cite{Halmos1969}.

\begin{quote}
"<Unter einer Menge verstehen wir eine Zusammenfassung von bestimmten
wohlunterschiedenen Objekten unserer Anschauung oder unseres Denkens zu
einem Ganzen."> \textsc{(G. Cantor)}
\end{quote}

Die Objekte einer Menge $M$ heißen \emph{Elemente\index{Element}} von $M$. Die Notation
$ x \in M$ \index{*@$\in$}\index{*@$\not\in$}
bedeutet, daß $x$ ein Element von $M$ ist, $x \not \in M$, daß $x$ kein
Element von $M$ ist.

In der Informatik wie in der Mathematik werden häufig Mengen von \emph{Zahlen} gebraucht.  Für die wichtigsten
Zahlenmengen gibt es feste Bezeichnungen.  So bezeichnet $\enn$
die Menge der \emph{natürlichen Zahlen\index{natürliche Zahl}}; \index{N@$\enn$} in diesem Buch gilt
$0\in\enn$. 
$\zet$\index{Z@$\zet$} bezeichnet die Menge der \emph{ganzen Zahlen\index{ganze Zahlen}} und $\err$\index{R@$\err$} die Menge der
\emph{reellen Zahlen\index{reelle Zahlen}}.

Endliche Mengen\index{Menge!endlich}\index{endliche Menge}, also Mengen mit endlich vielen Elementen
können als Aufreihung ihrer Elemente aufgeschrieben werden:
\[M = \{ 11, 13, 17, 19\}.\]
Häufig werden Mengen jedoch auch durch eine bestimmte Eigenschaft de\-fi\-niert,
die ihre Elementen haben:
\[M = \{x\ |\ x\ \mbox{ist Primzahl}, 10 \le x \le 20\}.\]
Die \emph{leere Menge\index{leere Menge}} %\index{$\varnothing$}
ist die Menge, die keine Elemente besitzt und wird
durch $\varnothing$\index{*@$\varnothing$} bezeichnet.

\emph{$A$ heißt Teilmenge\index{Teilmenge} von $B$}, in Zeichen $A \subseteq B$, \index{*@$\subseteq$}
wenn jedes Element von $A$ auch Element von $B$ ist:
\[A \subseteq B\ \ \diff\ \ 
a \in A \Rightarrow a \in B.\]
Zwei Mengen sind gleich, wenn sie die gleichen Elemente besitzen; 
dies läßt sich mit Hilfe der Teilmengenbeziehung auch so ausdrücken:
\[A = B\ \ \diff\ \ A \subseteq B\ \mbox{und}\ B \subseteq A.\]
Hieraus ergibt sich für die oben erwähnte Darstellung endlicher Mengen
z.B.
\[\{11,13,17,19\} =\{17,13,19,11\},\]
das heißt die \emph{Reihenfolge\index{Reihenfolge}} der Elemente ist unerheblich (bzw.\ es gibt
gar keine ausgezeichnete Reihenfolge) und
\[\{11,13,17,19\} = \{11,13,11,17,17,11,13,19\},\]
das heißt es spielt keine Rolle, wie oft ein bestimmtes Element erwähnt wird; es
ist trotzdem nur einmal in der Menge enthalten.

Die Notation $A \not \subseteq B$ bedeutet, daß $A \subseteq B$ nicht gilt, $A
\not = B$, daß $A = B$ nicht gilt. $A$ heißt \emph{echte Teilmenge\index{Teilmenge!echt}\index{echte Teilmenge}} von $B$,
wenn $A \subseteq B$, aber $A \not = B$. Die Notation dafür ist $A \subset B$.
\index{*@$\subset$}\index{*@$\supseteq$}\index{*@$\supset$}
Es bedeutet $B \supseteq A$, daß $A \subseteq B$ gilt, ebenso für
$B \supset A$. 

Die \emph{Vereinigung\index{Vereinigung}} $A \cup B$ \index{*@$\cup$}
zweier Mengen $A$ und $B$ ist definiert durch
\[A \cup B\ \ \deq\ \ \{a\ |\ a \in A \vee a \in B\}.\]

Der \emph{Durchschnitt\index{Durchschnitt}} $A \cap B$ \index{*@$\cap$}
zweier Mengen $A$ und $B$ ist definiert durch
\[A \cap B\ \ \deq\ \ \{a\ |\ a \in A \wedge a \in B\}.\]

Die \emph{Differenz\index{Differenz}} $A \setminus B$ \index{*@$\setminus$}
zweier Mengen $A$ und $B$ ist definiert durch
\[A \setminus B\ \ \deq\ \ \{a\ |\ a \in A \wedge a \not \in B\}.\]

\begin{definition}
\label{cart}
   Das \emph{kartesische Produkt} 
   \index{kartesisches Produkt} \index{*@$\times$} $A \times B$ zweier Mengen 
   $A$ und $B$ ist definiert durch
   \[ A \times B\ \ \deq \ \ \{(a,b)\ |\ a \in A,\ b \in B\}.\]
   Für $n \geq 2$ Mengen
   $A_1$, \ldots, $A_n$ ist definiert:
   \[A_1 \times \cdots \times A_n\ \ 
        \deq \ \ \{(a_1,\ldots,a_n)\ |\ a_i \in A_i\}.\]
   Für eine Menge $A$ und eine natürliche Zahl $n \geq 2$ ist
   \[A^n \ \ \deq  \ \ A \times 
           \stackrel{\stackrel{n}{\frown}}{\cdots} \times A.\]
   Um die Fälle $n=0$ und $n=1$
   nicht immer ausschließen zu müssen,
   wird außerdem definiert:
   \begin{eqnarray*}
   A^1 & \deq & A\\
   A^0 & \deq & \{()\} \ .
   \end{eqnarray*}
   $A^0$ ist also eine einelementige Menge, deren einziges Element in
   Über\-ein\-stim\-mung mit der Tupelschreibweise $(a_1,\ldots,a_n)$ mit $()$
   bezeichnet wird.\index{*@$()$}
\end{definition}

Für eine Menge $A$ wird die Anzahl ihrer Elemente, 
ihre \textit{Mäch\-tig\-keit}, \index{Mächtigkeit}\index{*@$\mid\:\mid$}
als $|A|$ geschrieben. Für unendliche Mengen wird
$|A| = \infty$ definiert.

Für eine Menge $A$ heißt 
\[ \mathcal{P}(A) \deq \{ T\ |\ T \subseteq A\}\]
die \emph{Potenzmenge} \index{Potenzmenge}\index{P(A)@$\mathcal{P}(A)$}
von $A$. Für endliche Mengen gilt $|\mathcal{P}(A)| = 2^{|A|}$.

\begin{definition}
    Für $T \in \mathcal{P}(A)$ ist das \emph{Komplement} von $T$ in $A$
    definiert durch\index{Komplement}\index{*@$\setminus$}
    \[ \overline{T} \deq A \setminus T\ .\]
\end{definition}

\begin{lemma}
   Für $A,B \in \mathcal{P}(M)$ gelten die sogenannten \textit{DeMorgan'schen Gesetze}:
   \index{DeMorgan'sches Gesetz}
   \[ \overline{A \cup B} = \overline{A} \cap \overline{B}\]
   \[ \overline{A \cap B} = \overline{A} \cup \overline{B}\]
\end{lemma}

Es ist üblich,
Teilmengen $T \subseteq \mathcal{P}(A)$ durch sog.\ \emph{charakteristische
Funktionen} dar\-zu\-stel\-len:\index{charakteristische Funktion}

\begin{definition}\label{def:charfunction}
 Sei $A$ eine Menge, $T \in \mathcal{P}(A)$. Die \emph{charakteristische
    Funk\-tion von T} ist definiert durch\index{chiT@$\chi_T$}\index{*@$\chi_T$}
    \[ \chi_T:\ A \longrightarrow \{0,1\}\]

    \[\chi_T(x) \deq \left \{ \begin{array}{ll}
                                1 & \textrm{falls\ } x \in T\\
                                0 & \textrm{falls\ } x \not \in T
                              \end{array}
                     \right . \]
    Ist umgekehrt $f:\ A \rightarrow \{0,1\}$ eine (totale) Abbildung, so läßt
    sich hieraus eine Menge $T_f \in \mathcal{P}(A)$ ableiten durch
    \[ T_f \deq \{x \in A \mid f(x) = 1\} .\]
\end{definition}

\noindent Die Zuordnung $T \mapsto \chi_T$ ist bijektiv. 


\section{Prädikatenlogik}
\label{sec:praedikate}

\index{Prädikatenlogik}Viele Aussagen haben die Form "<es gibt (mindestens) ein Objekt
(\emph{Individuum\index{Individuum}}) mit der Eigenschaft\index{Eigenschaft}\ldots"> oder "<für alle
Objekte aus einem bestimmten Be\-reich gilt\ldots">.  Solche Aussagen heißen
\emph{Prädikate} 
\index{Prädikat} und für ihre mathematische Behandlung
werden sogenannte \emph{Quantoren} eingeführt, und zwar der \emph{All\-quan\-tor}\index{Quantor}\index{Allquantor}\index{Existenzquantor}
(Universalquantor\index{Universalquantor}) $\forall$ \index{*@$\forall$} und der \emph{Existenzquantor}
$\exists$. \index{*@$\exists$} Im folgenden werden Großbuchstaben zur Bezeichnung
von Prädikaten und Kleinbuchstaben für Individuen verwendet. Die \emph{Stelligkeit\index{Stelligkeit}} eines Prädikats ist die Anzahl der Individuen, über die hier
eine Aussage gemacht wird.

Ist $Q$ ein $n$--stelliges Prädikat und sind $x_1,\ldots,x_n$ Individuen
eines Individuenbereichs, so ist die Behauptung, daß $Q$ auf $x_1,\ldots,x_n$
zutrifft (ab\-ge\-kürzt $Qx_1\ldots x_n$) eine prädikatenlogische Aussage. 
Mathematische
Ausdrücke wie $x \in M$ oder $n < m$ oder aussagenlogische Aussagen sind
ebenfalls  
prädikatenlogische Aussagen, nur in anderer Schreibweise.  
%Dies ist nur eine Vereinfachung der
%Notation, da in Wirklichkeit $x\in M$, $n<m$ und ähnliche Ausdrücke
%auch zweistellige Prädikate sind.  

Statt der Individuen selbst kommen in den
Quantorenausdrücken \emph{Individuenvariablen\index{Individuenvariable}\index{Variable!Individuen-}} vor; diese sind
Platzhalter für die Individuen selbst.  Die prädikatenlogische Aussage
$\forall x:~Qx$\index{*@$\forall$}
bedeutet: Für alle Individuen $a$ gilt die Aussage (das Prädikat)
$Q$, wobei für die Variable $x$ das Individuum $a$ eingesetzt wird. 
Dementsprechend heißt
$\exists x:~Qx$:\index{*@$\exists$}
Es gibt mindestens ein Individuum $a$, so daß die Aussage $Qa$ wahr ist.
Die Variable $x$ heißt in beiden Fällen eine \emph{gebundene 
Variable}\index{Variable!gebunden}\index{gebundene Variable}
des Prädikatsausdrucks.  Variablen, die nicht gebunden sind, heißen
\emph{freie Variablen}.\index{Variable!frei}\index{freie Variable}

Eine wichtige Eigenschaft der Quantoren beschreibt der folgende Satz:

\begin{satz}
Ist $Q$ ein einstelliges Prädikat, so ist die Aussage $\neg \forall x:~Qx$
genau dann wahr, wenn $\exists x:~\neg Qx$ wahr ist.  Umgekehrt ist
die Aussage $\neg(\exists x:~Qx)$ genau dann wahr, wenn $\forall x:~\neg
Qx$ wahr ist.
\end{satz}

Im Prinzip wäre es also möglich, mit nur einem der beiden Quantoren
auszukommen.

Häufig werden Einschränkungen an den Individuenbereich gemacht,  z.B.\ 
in der Form
\[ \forall x \in M:~\exists y \in N:\ Pxy\ . \]
Dies dient als Abkürzung für die kompliziertere Aussage
\[ \forall x:~(x \in M\ \Rightarrow \exists y \in N:\ Pxy)\]
oder die noch kompliziertere Aussage
\[ \forall x:~(x \in M\ \Rightarrow \exists y:~(y \in N \wedge Pxy))\ .\]




\section{Multimengen}
\label{sec:multisets}

Mengen sind so definiert, daß sie jedes Element "<nur einmal enthalten">. 
In einer \emph{Multimenge},
\index{Multimenge} kann jedes Element mit einer bestimmten
\emph{Multiplizität\index{Multiplizität}} (Vielfachheit) vorkommen.  Technisch werden
Multimengen als Kreuzprodukte $M \subseteq G \times (\mathbb{N}\setminus\{0\})$ beschrieben, wobei $G$
die \emph{Grundmenge\index{Grundmenge}} heißt und die natürliche Zahl die Multiplizität jedes Elements
der Grundmenge angibt.

Die Schreibweise $x\in M$\index{*@$\in$} ist eine Abkürzung für
die Tatsache $(x,n)\in M$.  Die Multiplizität $\mathcal{M}(M, x)$ eines Elements bzw.\
Nicht-Elements $x$  ist so definiert:
%
\begin{displaymath}
  \mathcal{M}(M, x)\deq{}
  \begin{cases}
    0 & x\not\in M\\
    n & (x, n)\in M
  \end{cases}
\end{displaymath}

Die mengentheoretischen Operationen sind wie folgt definiert:

\begin{eqnarray*}
  P\cup Q &\deq& \{(x,m)\mid x\in P \vee x \in Q,
  m=\max(\mathcal{M}(P, x), \mathcal{M}(Q, x))\}\\
  P\cap Q &\deq& \{(x,m)\mid(x,p)\in P, (x,q)\in Q, m=\min(p,q), m > 0\}\\
  P\setminus Q &\deq& \{(x,m)\mid (x, p)\in P, m = p\ominus \mathcal{M}(Q, x), m > 0\}
\end{eqnarray*}
Dabei ist $\ominus$ die positive Differenz mit dem Wert $0$ für
$q>p$.

\section{Relationen und Abbildungen}
\label{sec:relationen}

\begin{definition} Eine \emph{(binäre) Relation\index{binäre Relation}\index{Relation}}
ist eine Teilmenge $\rho \subseteq A \times B$.
%$\rho$ heißt 
%   \begin{itemize}
%   \item \emph{rechtseindeutig} \index{rechtseindeutig}
%   $\diff$ für alle $a \in A$ gibt es höchstens ein
%         $b \in B$ mit $(a,b) \in \rho$,
%   \item \emph{linkseindeutig} \index{linkseindeutig}
%   $\diff$ für alle $b \in B$ gibt es höchstens
%         ein $a \in A$ mit $(a,b) \in \rho$.
%   \end{itemize}
Eine alternative Notation für $(a,b)\in\rho$ ist
$a\rho b$.
Für eine Relation $\rho$ heißt
\[ \rho^{-1}\ \ \deq\ \ \{(b,a)\mid (a,b) \in \rho\}\]
die \emph{Umkehrrelation} \index{Umkehrrelation}
von $\rho$. 
\end{definition}

\begin{definition}\label{def:relations}
 Eine Relation $\rho \subseteq A\times A$ heißt
  \begin{itemize}
  \item \emph{reflexiv} \index{reflexiv}\index{Relation!reflexiv} $\diff$ für alle $a\in A$ gilt $a\rho
    a$, 
%  \item \emph{irreflexiv} \index{irreflexiv} $\diff$ für kein $a\in A$ gilt $a\rho
%    a$, 
  \item \emph{symmetrisch} \index{symmetrisch}\index{Relation!symmetrisch} $\diff$ aus $a\rho b$ folgt
    $b\rho a$,
  \item \emph{antisymmetrisch} \index{antisymmetrisch}\index{Relation!antisymmetrisch} $\diff$ aus $a\rho b$ und
    $b\rho a$ folgt $a=b$,
  \item \emph{transitiv} \index{transitiv}\index{Relation!transitiv} $\diff$ aus $a\rho b$ und
    $b\rho c$ folgt $a\rho c$,
  \item \emph{Äquivalenzrelation} \index{Relation!Äquivalenz-}\index{Äquivalenzrelation} $\diff$ $\rho$ ist 
    reflexiv, symmetrisch und transitiv.
  \end{itemize}
\end{definition}
%
Äquivalenzrelationen werden oft dazu verwendet, Elemente einer Menge
in \emph{Äquivalenzklassen\index{Äquivalenzklasse}} einzuteilen.  Die Äquivalenzklasse $[a]$ eines
Elements $a\in A$ bezüglich einer Äquivalenzrelation $\cong$ ist die
Menge aller Elemente $b\in A$ mit der Eigenschaft $b\cong a$.

Für eine Relation $\rho \subseteq A\times A$ wird häufig eine
verwandte Relation $\rho' \supseteq \rho$ betrachtet, die eine oder
mehrere der oberen Eigenschaften zusätzlich zu $\rho$ besitzt.  Eine
solche Relation heißt jeweils \textit{Abschluß\index{Abschluß}}:
%
\begin{definition}[Abschlüsse über Mengen]
  \label{def:relation-closure}
  Gegeben sei eine Relation $\rho \subseteq A\times A$.  Eine Relation
  $\rho' \subseteq A\times A$ heißt
  \begin{itemize}
  \item \textit{reflexiver Abschluß} von $\rho$ $\diff$ $\rho'$ ist
    die kleinste reflexive Relation mit $\rho\subseteq\rho'$,
  \item \textit{symmetrischer Abschluß} von $\rho$ $\diff$ $\rho'$ ist
    die kleinste symmetrische Relation mit $\rho\subseteq\rho'$,
  \item \textit{transitive Abschluß} von $\rho$ $\diff$ $\rho'$ ist
    die kleinste transitive Relation mit $\rho\subseteq\rho'$,
  \item \textit{reflexiv-transitiver Abschluß} von $\rho$ $\diff$
    $\rho'$ ist die kleinste reflexive und transitive Relation mit
    $\rho\subseteq\rho'$.
  \end{itemize}
  %
  Dabei heißt "<kleinste Relation"> jeweils, daß für jede andere
  Relation $\rho''$, welche die jeweilige Eigenschaft erfüllt, gilt
  $\rho' \subseteq \rho''$.
  
  Die transitive Abschluß von $\rho$ wird $\overset{+}{\rho}$\index{*@$\overset{+}{\rho}$}
  geschrieben, der reflexiv-transitive Abschluß
  $\overset{\ast}{\rho}$\index{*@$\overset{\ast}{\rho}$}.
\end{definition}


\begin{definition}\label{abbildung} Eine \emph{Abbildung} \index{Abbildung}
ist ein Tripel $f = (A,\rho_f,B)$, wobei
   $A$ und $B$ Mengen sind und $\rho_f\subseteq A \times B$ eine Relation, so
   daß für jedes $a\in A$ genau ein $b\in B$ existiert mit $a\rho_f b$. 
   $A$ heißt \emph{Vorbereich\index{Vorbereich}} von $f$, $B$ heißt \emph{Nachbereich\index{Nachbereich}} und
   $\rho_f$ der \emph{Graph\index{Graph}} von $f$.
   Für
   $f=(A,\rho_f,B)$ steht auch $f:A\rightarrow B$ oder 
   $A\stackrel{f}{\longrightarrow}B$. Für $(a,b) \in \rho_f$ steht
   normalerweise $f(a) = b$. %Die Teilmenge
%   \[\textrm{Def}(f) \ \ \deq \ \ \{a\in A\mid \textrm{es\ gibt }\ b \in B\ \textrm{mit}
%           \ f(a) = b\}\]
%   \index{Definitionsbereich}\index{Def$(f)$}
%   von $A$ heißt \emph{Definitionsbereich} von $f$, die Teilmenge
%   \[\textrm{Im}(f) \ \ \deq \ \ \{b\in B\mid \textrm{es\ gibt }\ a \in A\ \textrm{mit}
%           \ f(a) = b\}\]
%   \index{Bildbereich}\index{Im$(f)$}
%   von $B$ heißt \emph{Bildbereich} von $f$.
\end{definition}

Nach dieser Definition sind
zwei Abbildungen $f=(A,\rho_f,B)$ und $g=(C,\rho_g,D)$ \emph{gleich} genau
dann, wenn $A=C$, $B=D$ und $\rho_f=\rho_g$.
%    Beachte, daß die Gleichheit der Graphen auch die Gleichheit von
%    De\-fi\-ni\-ti\-ons- und Bildbereich mit sich bringt.

\begin{definition} 
 Für eine Menge $A$ ist die \emph{Identitätsabbildung} 
\index{Identitätsabbildung} 
   durch $\textrm{id}_A \deq (A,\rho_{\textrm{id}_A},A)$ mit $\rho_{\textrm{id}_A}
   \deq \{(a,a)\mid a \in A\}$ definiert.
\end{definition}

\begin{definition}
Eine Abbildung (auch: Funktion) $A \stackrel{f}{\longrightarrow} B$ heißt
    \begin{itemize}
    \item \emph{surjektiv} $\diff$ für alle $b\in B$ gibt es ein $a\in A$, so daß $f(a)=b$,\index{surjektiv}
    \item \emph{injektiv} $\diff$ aus $f(a_1) = f(a_2)$ folgt
          $a_1 = a_2$ für beliebige $a_1,a_2 \in A$,\index{injektiv}
    \item \emph{bijektiv} $\diff A \stackrel{f}{\longrightarrow} B$
      ist injektiv und surjektiv.\index{bijektiv}
    \end{itemize}
    Ist $A \stackrel{f}{\longrightarrow} B$ bijektiv, so heißen $A$ und $B$
    \emph{isomorph}, in Zeichen $A \cong B$.\index{isomorph}

\end{definition}

\begin{definition}
    Für zwei Abbildungen $A \stackrel{f}{\longrightarrow} B$ und
    $B \stackrel{g}{\longrightarrow} C$ definiere durch
    \[ g \circ f\ \ \deq\ \ (A,\rho_{g \circ f},C)\]
     die \emph{Komposition}
    von $g$ und $f$ \index{Komposition}\index{*@$\circ$} mit
    \[ \rho_{g \circ f}\ \ \deq\ \ \{(a,c)\mid \exists b \in B:\
         (a,b) \in \rho_f\ \wedge (b,c) \in \rho_g\} .\]
\end{definition}

\begin{lemma} Die Komposition von Abbildungen ist assoziativ, das heißt
   für $A \stackrel{f}{\longrightarrow} B$, $B \stackrel{g}{\longrightarrow} C$
   und $C \stackrel{h}{\longrightarrow} D$ gilt
   \[ (h \circ g) \circ f = h \circ (g \circ f)\ .\]
\end{lemma}
Es ist deshalb möglich, die Klammern ganz wegzulassen und $h \circ g \circ f$
zu  schreiben.

\begin{lemma} Für eine bijektive Abbildung $A \stackrel{f}{\longrightarrow} B$
   mit $f =(A, \rho_f,B)$ existiert eine \emph{Umkehrabbildung\index{Umkehrabbildung}}
   $B \stackrel{f^{-1}}{\longrightarrow} A$ mit $f^{-1} \deq (B,\rho_{f^{-1}},A)$,
   wobei $\rho_{f^{-1}} \deq {\rho_f}^{-1}$. Es gilt $f^{-1} \circ f = \textrm{id}_A$
   und $f \circ f^{-1} = \textrm{id}_B$.
\end{lemma}

\begin{definition}
  \label{def:power_fun}
  Sei $A$ eine Menge, 
  $f:A\rightarrow A$ eine Abbildung.  Dann ist definiert:
  \begin{eqnarray*}
    f^0 &\deq& \textrm{id}_A\\
    f^{n+1} &\deq& f\circ f^n
  \end{eqnarray*}
\end{definition}

\section{Ordnungen}
\label{sec:halbordnungen}

\begin{definition}
   Sei $M$ eine Menge. Eine Relation $\rho \subseteq M \times M$ ist eine
   \textit{partielle Ordnung}\index{partielle
     Ordnung}\index{Ordnung!partiell}
   (auch
   \textit{Halb\-ord\-nung}\index{Halbordnung}\index{Ordnung!Halb-} genannt)
 auf $M$ \ $\diff \ \rho$ ist
   reflexiv, transitiv und antisymmetrisch (vgl.\ Definition~\ref{def:relations}).
\end{definition}

Ein Beispiel für eine partielle Ordnung ist etwa die Relation "<$\subseteq$"> zwischen 
Mengen. Eine \emph{partiell geordnete Menge} $(M;\rho)$ 
ist eine Menge $M$ mit einer
partiellen Ordnung $\rho$. In einer 
partiell ge\-ord\-ne\-ten Menge kann es
\emph{unvergleichbare Elemente\index{unvergleichbare Elemente}} geben, das heißt Elemente $x,y \in M$, für die weder
$x \rho y$ noch $y \rho x$ gilt. In der halbgeordneten Menge $(M;\subseteq)$ mit
\[ M = \{\{1,2\},\{2,3\},\{1,2,3\}\} \]
sind z.B.\ die Elemente $\{1,2\}$ und $\{2,3\}$ unvergleichbar. 

\begin{definition}
  \label{def:total-order}
    Eine \emph{totale Ordnung} \index{totale Ordnung}\index{Ordnung!total}
    auf einer Menge $M$ ist eine partielle Ordnung
    $\rho$, bei der zusätzlich gilt
    \[ \forall x,y \in M:\ x \rho y \vee y \rho x\ .\]
\end{definition}
\begin{definition}
    Sei $(M;\rho)$ eine halbgeordnete Menge. Ein Element $x \in M$ heißt
    \begin{description}
    \item[\it minimales Element:] $\diff \forall y \in M:\  y \rho x 
                                         \Rightarrow y=x$
                                         \index{minimales Element}
    \item[\it kleinstes Element:] $\diff \forall y \in M:\  x \rho y$
                \index{kleinstes Element}
    \item[\it maximales Element:] $\diff \forall y \in M:\  x \rho y 
                                         \Rightarrow y=x$
                                         \index{maximales Element}
    \item[\it größtes Element:] $\diff \forall y \in M:\  y \rho x$
                \index{größtes Element}
    \end{description}
\end{definition}

Die Begriffe "<minimales"> und 
"<kleinstes Element"> werden gern
verwechselt. Das liegt vielleicht daran, daß die Relation "<$\leq$"> auf
Zahlen, die oft als "<Modell"> für eine partielle Ordnung herhalten muß, in
Wirklichkeit eine totale Ordnung ist. Bei totalen Ordnungen fallen die Begriffe
"<minimales"> und "<kleinstes Element"> jedoch zusammen.
Deshalb hier noch einmal eine verbale Definition:
\begin{description}
\item[\it minimales Element] heißt, daß es kein Element gibt, das kleiner
             ist. Es kann aber Elemente geben, die unvergleichbar mit einem
             minimalen Element sind. (In der oben angegebenen Menge $M$ sind
             $\{1,2\}$ und $\{2,3\}$ beide minimal.)
\item[\it kleinstes Element] heißt, daß alle anderen Elemente größer
             sind. Damit ist auch die Vergleichbarkeit gegeben. Es gibt in einer
             Menge höchstens ein kleinstes Element. (In der oben
             angegebenen Menge $M$ gibt es kein kleinstes Element.)
\end{description}

\begin{definition}
  Eine totale Ordnung heißt \textit{wohlfundierte
    Ordnung}\index{wohlfundierte Ordnung}\index{Ordnung!wohlfundiert}, wenn es keine unendlich langen
  absteigenden Ketten\index{Kette} gibt, d. h. keine unendliche Folge $(a_i\mid i\in\enn)$
  mit der Eigenschaft
  \[\forall i\in\enn:\ a_i\rho^{-1}a_{i+1}\ .\]
\end{definition}
In einer wohlgeordneten Menge besitzt jede
  nichtleere Teilmenge ein kleinstes Element.

\section*{Aufgaben}
\begin{aufgabe}
 $A \stackrel{f}{\rightarrow}B\stackrel{g}{\rightarrow}C$ sei injektiv
   (surjektiv). Welche Aussagen über $A\stackrel{f}{\rightarrow}B$ bzw.\ 
   $B\stackrel{g}{\rightarrow}C$ gelten dann mit Bestimmtheit?
\end{aufgabe}

\begin{aufgabe} 
   Zeige, daß die Umkehrfunktion $B \stackrel{f^{-1}}{\longrightarrow}
   A$ einer bijektiven Funktion $A \stackrel{f}{\rightarrow} B$ stets
   injektiv ist.
\end{aufgabe}

\begin{aufgabe}
        Für eine endliche Menge $A$ beweise
        \[|\mathcal{P}(A)| = 2^{|A|}\]
\end{aufgabe}

\begin{aufgabe}
        Sei $A$ eine endliche Menge. Beweise, daß für $T\in\mathcal{P}(A)$
        gilt
\[|T| + |\overline T| = |A|\]
\end{aufgabe}

\begin{aufgabe}
  Beweise anhand der Wahrheitstafeln für die Aussagenlogik:
  \begin{itemize}
  \item Die DeMorgan'schen Gesetze:
    \begin{eqnarray*}
      \neg(A\wedge B) = \neg A \vee \neg B\\
      \neg(A\vee B) = \neg A \wedge \neg B
    \end{eqnarray*}
  \item Das erste Distributivgesetz:
    \begin{displaymath}
      A\vee(B\wedge C) = (A\vee B)\wedge (A\vee C)
    \end{displaymath}
  \item Die Kontraposition:
    \begin{displaymath}
      (A\Rightarrow B) \Leftrightarrow (\neg B\Rightarrow \neg A)
    \end{displaymath}
  \item Folgenden Satz über die Implikation:
    \begin{displaymath}
    (A\Rightarrow (B\Rightarrow C))\Rightarrow((A\Rightarrow
      B)\Rightarrow (A\Rightarrow C))
    \end{displaymath}
  \end{itemize}
\end{aufgabe}

%\begin{exe}\rm
%        Stellen Sie den folgenden aussagenlogischen Ausdruck durch
%        ausschließliche Verwendung von ${\sf nor}$ dar:

%        \[(a\wedge b) \wedge (c\vee \bar d)\]
%\end{exe}

\begin{aufgabe}
  In
  einem an Schulen populären Lehrbuch der Mathematik wird folgende
  Aussage gemacht:
  %
  \begin{quote}
    Für eine partielle Ordnung $\sqsubseteq$ auf einer Menge $D$ und
    $d_1,d_2\in D$ mit $d_1 \not\sqsubseteq d_2$ gilt $d_2\sqsubseteq
    d_1$.
  \end{quote}
  %
  Gib eine partielle Ordnung an, für welche die
  Aussage nicht gilt!
\end{aufgabe}

%%% Local Variables: 
%%% mode: latex
%%% TeX-master: "i1"
%%% End: 
