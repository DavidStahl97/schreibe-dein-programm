% Diese Datei ist Teil des Buchs "Schreibe Dein Programm!"
% Das Buch ist lizensiert unter der Creative-Commons-Lizenz
% "Namensnennung - Weitergabe unter gleichen Bedingungen 4.0 International (CC BY-SA 4.0)"
% https://creativecommons.org/licenses/by-sa/4.0/deed.de

\chapter{Eigenschaften von Funktionen}
\label{cha:properties}

Daß $1+1$ gleich $2$ ist, ist ein \textit{Beispiel} für die
Arbeitsweise der Addition.  Dieses Beispiel könnte auch als Testfall
für eine programmierte Version der Addition durchgehen.  Unter
Umständen kann ein Beispiel, als Testfall formuliert, einen Fehler in
einem Programm finden.  Allerdings ist das Formulieren von Beispielen
mühsam.  Schlimmer noch, eine Menge von Testfällen reicht nur selten
aus, um die Korrektheit einer Funktion auch zu garantieren: Die
Testfälle decken meist nicht alle möglichen Anwendungen einer Funktion
ab.  Darum ist es oft sinnvoll, statt isolierter Beispiele allgemeine
Eigenschaften zu formulieren und zu überprüfen~-- am besten sogar,
diese zu beweisen.  Dieses Kapitel zeigt, wie das geht.

\section{Eigenschaften von eingebauten Operationen}

In diesem Abschnitt wird die Formulierung und Überprüfung von
Eigenschaften anhand von bekannten eingebauten Operationen wie
\texttt{+}, \texttt{and}, \texttt{=} etc.\ demonstriert.

\subsection{Binäre Operationen}
\label{sec:eigenschaften-binaere-operationen}

Eine allgemein bekannte Eigenschaft der Addition ist die
\index{Kommutativität}\textit{Kommutativität}:
%
\begin{displaymath}
a + b = b + a
\end{displaymath}
%
Auch wenn intuitiv die Bedeutung klar ist, ist  die Eigenschaft genau
genommen so noch nicht präzise schriftlich festgehalten, da nicht
notiert ist, was $a$ und $b$ sind: Die Idee ist natürlich,
daß $a$ und $b$ beliebige \emph{Zahlen} sind.  Im allgemeinen also:
%
\begin{displaymath}
\forall a \in \mathbb{C}, b \in \mathbb{C}:\ a + b = b + a 
\end{displaymath}
%
(Wer sich mit der Vorstellung komplexer Zahlen nicht wohlfühlt, kann
das $\mathbb{C}$ auch durch $\mathbb{R}$ oder $\mathbb{Q}$ ersetzen.)

\begin{feature}{\texttt{for-all}}{scheme:for-all}
  \texttt{For-all}\index{for-all@\texttt{for-all}} ermöglicht das
  Formulieren von \textit{Eigenschaften\index{Eigenschaft}}.  Ein
  \texttt{for-all}-Ausdruck hat die folgende allgemeine Form:
%
\begin{alltt}
(for-all ((\(v\sb{1}\) \(c\sb{1}\)) \(\ldots\) (\(v\sb{n}\) \(c\sb{n}\))) \(b\))
\end{alltt}
%
Dabei müssen die $v_i$ Variablen sein, die $c_i$ Signaturen und $b$ (der
Rumpf) ein Ausdruck, der entweder einen booleschen Wert oder eine
Eigenschaft liefert.  Der \texttt{for-all}-Ausdruck hat als Wert eine
Eigenschaft, die besagt, daß \(b\) gilt für \emph{alle} Werte der
$v_i$, welche die Signaturen $c_i$ erfüllen.
\end{feature}

In den Lehrspachen läßt sich diese Eigenschaft für die eingebaute Funktion
\texttt{+} aufschreiben~-- das $\forall$ ist auf
Tastaturen nicht vertreten und wird darum ausgeschrieben (siehe
Abbildung~\ref{scheme:for-all}):
%
\begin{alltt}
(for-all ((a number)
          (b number))
  (= (+ a b) (+ b a)))
\end{alltt}
%
Das Ergebnis dieses Ausdrucks wird in der REPL etwas undurchsichtig
angezeigt:
%
\begin{alltt}
\evalsto{} #<:property>
\end{alltt}
%
\begin{feature}{\texttt{check-property}}{scheme:check-property}

\texttt{Check-property}\index{check-property@\texttt{check-property}}
testet eine Eigenschaft analog zu \texttt{check-expect}.  Eine
\texttt{check"=property}"=Form sieht so aus:
%
\begin{alltt}
(check-property \(e\)) 
\end{alltt}
%
$e$ ist ein Ausdruck, der eine Eigenschaft liefern muß~-- in der Regel
also ein \texttt{for-all}-Ausdruck.

Bei der Auswertung setzt \texttt{check-property} für die Variablen der
\texttt{for-all}-Ausdrücke verschiedene Werte ein und testet, ob die
Eigenschaft jeweils erfüllt ist.

\texttt{Check-property} funktioniert nur für Eigenschaften, bei denen
aus den Signaturen sinnvoll Werte generiert werden können.  Dies ist
für die meisten eingebauten Signaturen der Fall, aber nicht für
Signaturvariablen und Signaturen, die mit \texttt{predicate},
\texttt{property} oder \texttt{define-record-procedures} definiert
wurden.
\end{feature}
%
Bessere Informationen lassen sich erzielen, wenn \texttt{for-all}-Ausdrücke in
eine  \texttt{check-property}-Form
(siehe Abbildung~\ref{scheme:check-property}) eingebettet werden:
%
\begin{alltt}
(check-property
  (for-all ((a number)
            (b number))
    (= (+ a b) (+ b a))))
\end{alltt}
%
\texttt{Check-property} fungiert, wie \texttt{check-expect} oder
\texttt{check-within}, als Testfall und wird auch als solcher
ausgewertet.  Da \texttt{+} tatsächlich kommutativ ist, läuft der
Testfall auch anstandslos durch.

Interessanter wird es erst bei Eigenschaften, die nicht stimmen.  Zum
Beispiel ist die Subtraktion \texttt{-} nicht kommutativ:
%
\begin{alltt}
(check-property
 (for-all ((a number)
           (b number))
   (= (- a b)
      (- b a))))
\end{alltt}
%
Hierfür liefert \drscheme{} folgende Meldung:
%
\begin{alltt}
        Eigenschaft falsifizierbar mit a = \framebox{#i1.25} b = \framebox{#i-4.0}
\end{alltt}
%
\textit{Falsifizierbar}\index{falsifizierbar} bedeutet, daß es ein
\textit{Gegenbeispiel\index{Gegenbeispiel}} für die Eigenschaft gibt,
also Werte für die Variablen \texttt{a} und \texttt{b}, welche die
Eigenschaft falsch werden lassen.  \drscheme{} hat in diesem Fall ein
Gegenbeispiel gefunden, bei dem \texttt{a} den Wert $0.0$ und
\texttt{b} den Wert $1.5$ hat:
%
\begin{alltt}
(- 0.0 -1.5)
\evalsto{} 1.5
(- -1.5 0.0)
\evalsto{} -1.5
\end{alltt}
%
Dieses Beispiel widerlegt also tatsächlich die Behauptung der Eigenschaft.

Hinter den Kulissen hat \drscheme{} verschiedene Werte für \texttt{a} und
\texttt{b} durchprobiert und in die Eigenschaft eingesetzt, also effektiv
nach einem Gegenbeispiel gesucht.  Für die Kommutativität von
\texttt{+} gibt es kein Gegenbeispiel~-- \drscheme{} konnte also auch
keins finden.  Daß ausgerechnet das merkwürdige Beispiel $0.0$ und
$-1.5$ herauskam, liegt an der relativ komplexen Suchstrategie von
\drscheme{}.

Auf diese Art und Weise lassen sich eine Reihe von interessanten
Eigenschaften formulieren, so zum Beispiel die Assoziativität\index{Assoziativität} von
\texttt{+}:\label{sec:plus-not-associative}
%
\begin{alltt}
(check-property
 (for-all ((a number)
           (b number)
           (c number))
    (= (+ a (+ b c))
       (+ (+ a b) c))))
\end{alltt}
%
Hierbei gibt es allerdings eine böse Überraschung~-- \drscheme{} produziert
ein Gegenbeispiel:

FIXME: inexakte Zahlen

%
\begin{alltt}
Eigenschaft falsifizierbar mit
  a = \framebox{2.6666666666666665} b = \framebox{6.857142857142857} c = \framebox{-6.857142857142857}
\end{alltt}
%
Es ist kein Zufall, daß es sich um Zahlen mit vielen Nachkommastellen
handelt.  Wenn dieses Gegenbeispiel in die Eigenschaft eingesetzt
wird, liefert die REPL folgende Ergebnisse:
%
\begin{alltt}
(+ 2.6666666666666665 (+ 6.857142857142857 -6.857142857142857))
\evalsto{} 2.6666666666666665
(+ (+ 2.6666666666666665 6.857142857142857) -6.857142857142857)
\evalsto{} 2.666666666666667
\end{alltt}
%
Hier wird sichtbar daß, wie bereits in
Abschnitt~\ref{sec:programming-elements} erwähnt, bei Berechnungen mit
sogenannten \textit{inexakten Zahlen\index{inexakte Zahlen}}, das sind
Zahlen mit einem Dezimalpunkt, die mathematischen Operationen nur mit
einer begrenzten Anzahl von Stellen durchgeführt werden und dann runden~-- da
auch noch binär und nicht dezimal gerundet wird, sieht das Ergebnis
dieser Rundung oft unintuitiv aus.  Dieses Beispiel zeigt nun, daß
Addition plus binäre Rundung nicht assoziativ ist.  Die Assoziativität
gilt nur für das Rechnen mit \textit{exakten Zahlen\index{exakte
    Zahlen}}.  Immerhin sind alle Zahlen mit der Signatur
\texttt{rational} exakt, die
Eigenschaft läßt sich also reformulieren:
%
\begin{alltt}
(check-property
 (for-all ((a rational)
           (b rational)
           (c rational))
    (= (+ a (+ b c))
       (+ (+ a b) c))))
\end{alltt}
%
Und tatsächlich, in dieser Form wird die Eigenschaft nicht
beanstandet.

Kommutativität und Assoziativität sind jeweils Eigenschaften einer
einzelnen Operation, in diesem Fall \texttt{+}.  Manche Eigenschaften
beschreiben auch das Zusammenspiel mehrerer Operationen, wie zum
Beispiel die Distributivität, die für Addition und Multiplikation
gilt:
%
\begin{displaymath}
\forall a \in \mathbb{C}, b \in \mathbb{C}, c \in \mathbb{C}:\
a\cdot(b+c) = a\cdot b + b\cdot c
\end{displaymath}
%
Auch dies läßt sich direkt in Code übersetzen, diesmal gleich mit
\texttt{rational} statt \texttt{number}:
%
\begin{alltt}
(check-property
 (for-all ((a rational)
           (b rational)
           (c rational))
   (= (* a (+ b c))
      (+ (* a b) (* a c)))))
\end{alltt}
%
Auch hier hat \drscheme{} nichts zu meckern.

Neben der Addition ist auch die Multiplikation kommutativ:
%
\begin{alltt}
(check-property
 (for-all ((a rational)
           (b rational))
   (= (* a b)
      (* b a))))
\end{alltt}
%
Wenn Sie diese Eigenschaft neben die Kommutativität für \texttt{+}
legen, sehen Sie, daß diese fast identisch sind und damit natürliche
Kandidaten für Abstraktion: Nur die Operation~-- \texttt{*} im einen
und \texttt{+} im anderen Fall~-- ist unterschiedlich.  Wenn wir über
die Operation abstrahieren, bekommen wir so etwas wie eine allgemeine
Definition der Kommutativität, und das sieht so aus:

\begin{alltt}
(define commutativity
  (lambda (op)
    (for-all ((a rational)
              (b rational))
      (= (op a b)
         (op b a)))))
\end{alltt}
%
Mit Hilfe dieser Definition können wir die Kommutativität von
\texttt{+} und \texttt{*} deutlich kompakter formulieren:
%
\begin{alltt}
(check-property (commutativity *))
(check-property (commutativity +))
\end{alltt}
%
Über dem \texttt{check-property} können wir nicht abstrahieren~-- es
muß ganz außen stehen, damit \drscheme{} Fehlermeldungen den dazu
passenden Programmstellen zuordnen kann.

Der Vollständigkeit halber braucht \texttt{commutativity} noch eine
Signatur: \texttt{+} und \texttt{*} sind jeweils Funktionen, die zwei
Zahlen akzeptieren und wieder eine Zahl zurückliefern.  Der
Rückgabewert von \texttt{commutativity} ist eine Eigenschaft, für die
in \drscheme{} die Signatur \texttt{property} fest eingebaut ist.  Die
fertige Signatur ist also diese hier:
%
\begin{alltt}
(: commutativity ((rational rational -> rational) -> property))
\end{alltt}

Diese drei Eigenschaften~-- Kommutativität, Assoziativität und
Distributivität~-- tauchen immer wieder auf, da sie nicht nur für
arithmetische Operationen gelten (auch die Multiplikation ist
kommutativ und assoziativ) sondern auch anderswo.  

Zum Beispiel gelten
Kommutativität und Assoziativität auch für das logische \texttt{and}:
%
\begin{alltt}
(check-property
 (for-all ((a boolean)
           (b boolean))
    (boolean=? (and a b)
               (and b a))))

(check-property
 (for-all ((a boolean)
           (b boolean)
           (c boolean))
    (boolean=? (and a (and b c))
               (and (and a b) c))))
\end{alltt}
%
Hier muß die eingebaute Funktion \texttt{boolean=?} verwendet werden,
die boolesche Werte vergleicht, analog zu \texttt{=}, die nur Zahlen
vergleichen kann.

Schön wäre natürlich, wenn wir auch für die Kommutativität von
\texttt{and} die obige Funktion \texttt{commutativity} verwenden
könnten: Das Problem ist aber, daß sich die Kommutativität von
\texttt{and} an zwei weiteren Stellen von der Kommutativität für
\texttt{*} und \texttt{+} unterscheidet, nämlich bei der Signatur
(\texttt{boolean} statt \texttt{rational}) und auch bei der
Vergleichsoperation (\texttt{boolean=?} statt \texttt{=}).  Um auch
\texttt{and} in den Einzugsbereich von \texttt{commutativity} zu
holen, müssen wir also auch noch über diese beiden Werte abstrahieren:
%
\begin{alltt}
(define commutativity
  (lambda (op sig =?)
    (for-all ((a sig)
              (b sig))
      (=? (op a b)
          (op b a)))))
\end{alltt}
%
Für \texttt{*} und \texttt{+} müssen wir \texttt{commutativity} nun
wie folgt aufrufen:
%
\begin{alltt}
(check-property (commutativity * (signature rational) =))
(check-property (commutativity + (signature rational) =))
\end{alltt}
%
Denken Sie an das \texttt{signature}, das immer notwendig ist, wenn
eine Signatur außerhalb einer Signatur-Deklaration mit \texttt{:} sowie
einem \texttt{for-all} vorkommt.

Um \texttt{commutativity} auch auf \texttt{and} und \texttt{or}
loszulassen, gibt es allerdings noch ein weiteres Hindernis: Das
Argument zu \texttt{op} muß eine Funktion sein~-- \texttt{and} und
\texttt{or} sind aber Spezialformen.  Wir können Sie aber zu
Funktionen machen, indem wir \texttt{lambda}s darumwickeln:
%
\begin{alltt}
(check-property (commutativity (lambda (a b) (and a b))
                               (signature boolean) boolean=?))
(check-property (commutativity (lambda (a b) (or a b))
                               (signature boolean)
                               boolean=?))
\end{alltt}
%
Bei der neuen Version von \texttt{commutativity} fehlt noch die
Signatur.  Wir müssen dazu die ursprüngliche Signatur
%
\begin{alltt}
(: commutativity ((rational rational -> rational) -> property))
\end{alltt}
%
ziemlich radikal renovieren: Das erste Argument ist zwar immer noch
eine zweistellige Funktion, aber nicht mehr notwendigerweise auf
rationalen Zahlen.  Wir skizzieren erstmal, was wir wissen:
%
\begin{alltt}
(: commutativity ((? ? -> ?) signature (? ? -> boolean) -> property))
\end{alltt}
%
Die eingebaute Signatur \texttt{signature} ist für Signaturen
zuständig~-- das zweite Argument ist ja eine Signatur.  Von der
Vergleichsfunktion an dritter Stelle ist klar, daß sie ein
\texttt{boolean} liefert.  Für die restlichen Fragezeichen ist die
genaue Signatur abhängig vom konkreten Operator und dieser (ebenfalls
variablen) Signatur, wir müssen also Signaturvariablen verwenden.

Was ist noch bekannt?  Die beiden Argumente der Funktion \texttt{op}
müssen auf dieselbe Signatur passen, da sie ja vertauschbar sind:
%
\begin{verbatim}
(: commutativity ((%a %a -> ?) signature (? ? -> boolean) -> property))
\end{verbatim}
%
Außerdem wird der Rückgabewert von \texttt{op} in die
Vergleichsfunktion gefüttert, für die restliche drei Fragezeichen
müssen wir also dieselbe Signatur einsetzen.  Ist erforderlich, daß
der Rückgabewert von \texttt{op} auf die gleiche Signatur paßt wie die
Argumente?  Der Rückgabewert wird nicht wieder in \texttt{op}
hineingefüttert, die Antwort ist also nein.  Wir können also eine
von \verb|%a| verschiedene Signaturvariable benutzen:
%
\begin{verbatim}
(: commutativity ((%a %a -> %b) signature (%b %b -> boolean) -> property))
\end{verbatim}
%
Genauso wie bei der Kommutativität können wir auch bei der
Assoziativität abstrahieren.  Hier die Abstraktion, die dabei
herauskommt:
%
\begin{verbatim}
(define associativity
  (lambda (op sig =?)
    (for-all ((a sig)
              (b sig)
              (c sig))
      (=? (op a (op b c))
          (op (op a b) c)))))
\end{verbatim}
%
Benutzen können wir Sie ähnlich wie bei der Kommutativität:
%
\begin{alltt}
(check-property (associativity + (signature rational) =))
(check-property (associativity * (signature rational) =))
(check-property (associativity (lambda (a b) (and a b))
                               (signature boolean)
                               boolean=?))
(check-property (associativity (lambda (a b) (or a b))
                               (signature boolean)
                               boolean=?))
\end{alltt}
%
Auch hier die Formulierung der Signatur nicht so einfach.  Die erste
Skizze könnte so aussehen:
%
\begin{alltt}
(: associativity ((? ? -> ?) signature (? ? -> boolean) -> property))
\end{alltt}
%
Wie bei \texttt{commutativity} wird der Rückgabewert von \texttt{op}
als Argument für die Vergleichsfunktion verwendet: Die letzten drei
Fragezeichen müssen also wieder gleich sein.  Anders als bei der
Kommutativität wird der Rückgabewert von \texttt{op} auch wieder als
Argument in \texttt{op} hereingefüttert.  Damit müssen auch die ersten
beiden Fragezeichen den anderen entsprechen.  Die beste Signatur ist
also wie folgt:
%
\begin{alltt}
(: associativity ((%a %a -> %a) signature (%a %a -> boolean) -> property))
\end{alltt}
%
\texttt{And} und
\texttt{or} erfüllen auch zwei Distributivgesetze.  Damit beschäftigt
sich Aufgabe~\ref{aufgabe:boolean-distrib}.

Auch das \textrm{DeMorgan'sche Gesetz\index{DeMorgan'sches Gesetz}}
(siehe Abschnitt~\ref{sec:aussagenlogik}) können wir als Code
formulieren:
%
\begin{alltt}
(check-property
 (for-all ((a boolean)
           (b boolean))
   (boolean=? (not (and a b))
              (or (not a) (not b)))))
\end{alltt}
%
Bei vielen Operationen ist außerdem interessant, ob sie ein
\textit{neutrales Element\index{neutrales Element}} besitzen, also ein
Argument, das dafür sorgt, daß die Operation ein anderes Argument
unverändert zurückgibt.  Die Addition hat z.B.\ die $0$ als neutrales
Element:
%
\begin{alltt}
(check-property
  (for-all ((a rational))
    (= (+ a 0) a)))
\end{alltt}
%
Streng genommen ist damit nur gesichert, daß $0$ \textit{rechtsneutrales
  Element} ist, also von rechts addiert das andere Argument
unverändert herauskommt.  Aus der Kommutativität folgt aber, daß jedes
rechtsneutrale Element auch ein linksneutrales Element ist.

Bei manchen Operationen gibt es neben dem neutralen Element zu jedem
Element auch ein \textit{inverses Element\index{inverses Element}}:
Wenn eine binäre Operation auf ein Element und sein Inverses
angewendet wird, so muß das neutrale Element herauskommen.  Bei der
Addition entsteht das Inverse zu einer Zahl durch Umdrehen des
Vorzeichens:
%
\begin{alltt}
(check-property
 (for-all ((a rational))
   (= (+ a (- a)) 0)))

(check-property
 (for-all ((a rational))
   (= (+ (- a) a) 0)))
\end{alltt}
%
Hier noch einmal eine Zusammenfassung der in diesem Abschnitt
behandelten Eigenschaften, mit Kurzfassungen der mathematischen
Formulierungen:

\begin{mantra}[Eigenschaften von binären Operationen]
%
Folgende Eigenschaften sind prinzipiell auf alle \textit{binären}
Operationen denkbar, die zwei Elemente einer Menge $M$ akzeptieren
und wiederum ein Element von $M$ zurückgeben.
\begin{itemize}
\item Kommutativität $a \star b = b \star a$
\item Assoziativität $(a \star b) \star c = a \star (b \star c)$
\item Distributivität $a \otimes (b \star c) = (a \otimes b) \star (a
  \otimes c)$; $(b \star c) \otimes a = (b \otimes a) \star (c
  \otimes a)$
\item neutrales Element ($a\star \nu = a$; $\nu\star a = a$)
\item inverses Element $a\star a^{-1} = \nu$; $a^{-1}\star a = \nu$
\end{itemize}
\end{mantra}

\subsection{Eigenschaften von binären Prädikaten}

Die Funktion \texttt{=} paßt nicht in das Schema der Eigenschaften
des folgenden Abschnitts.  Sie hat folgende Signatur:
%
\begin{alltt}
(: = (number number -> boolean))
\end{alltt}
%
Damit akzeptiert sie zwar zwei Argumente aus derselben Menge, liefert
aber einen booleschen Wert zurück.  Stattdessen handelt es sich um ein
\textit{binäres Prädikat\index{binäres
    Prädikat}\index{Prädikat!binär}} bzw.\ eine \textit{binäre
  Relation\index{binäre Relation}\index{Relation!binär}}.  Für binäre
Relationen kommt ein anderer Satz von Eigenschaften in Frage.  (Die
mathematische Seite ist in Anhang~\ref{sec:relationen} beschrieben.)
Insbesondere ist \texttt{=} eine
\textit{Äquivalenzrelation\index{Äquivalenzrelation}} und damit
\textit{reflexiv\index{reflexiv}},
\textit{symmetrisch\index{symmetrisch}} und
\textit{transitiv\index{transitiv}}.

Die Reflexivität besagt, daß jedes Element der Grundmenge (in diesem
Fall die Menge der Zahlen) zu sich selbst in Beziehung steht:
%
\begin{alltt}
(check-property
 (for-all ((a number))
   (= a a)))
\end{alltt}
%
Die Symmetrie bedeutet für \texttt{=}, daß aus \texttt{(= a b)
  \evalsto{} \#t} das "<Spiegelbild"> \texttt{(= b a) \evalsto{} \#t}
folgt.  Mathematisch geschrieben sähe das so aus:
%
\begin{displaymath}
  \forall a \in \mathbb{C}, b\in\mathbb{C}:\ a = b \Rightarrow b = a
\end{displaymath}
%
\begin{feature}{\texttt{==>}}{scheme:implies}

Eine \textit{Implikation}\index{Implikation} in einer Eigenschaft wird
folgendermaßen geschrieben:
\begin{alltt}
(==> \(e\) \(e\sb{p}\))
\end{alltt}
%
Dabei muß $e$ ein Ausdruck mit booleschem Wert sein (die
\textit{Voraussetzung}) und \(e\sb{p}\) eine Eigenschaft oder ein
boolescher Ausdruck.  Die Implikation liefert ihrerseits wieder eine
Eigenschaft, die gilt, wenn \(e\sb{p}\) immer dann gilt, wenn die
Voraussetzung erfüllt ist, also \verb|#t| liefert.
\end{feature}

Der Implikationspfeil $\Rightarrow$ wird in den Lehrsprachen
\texttt{==>}\index{==>@\texttt{==>}} geschrieben.  (Siehe
Abbildung~\ref{scheme:implies}.)  Der Test der
Symmetrie sieht also folgendermaßen aus:
%
\begin{alltt}
(check-property
 (for-all ((a number)
           (b number))
   (==> (= a b)
        (= b a))))
\end{alltt}
%
Ähnlich läuft es mit der Transitivität: Wenn zwei Zahlen $a$ und $b$
gleich sind sowie $b$ und eine dritte Zahl $c$, dann müssen auch $a$
und $c$ gleich sein:
%
\begin{alltt}
(check-property
 (for-all ((a number)
           (b number)
           (c number))
   (==> (and (= a b) (= b c))
        (= a c))))
\end{alltt}
%
Neben den drei Eigenschaften von Äquivalenzrelationen tritt auch
gelegentlich die Eigenschaft
\textit{Antisymmetrie}\index{antisymmetrisch} auf (die mathematische
Definition steht in Anhang~\ref{sec:relationen}).

\begin{mantra}[Eigenschaften von binären Prädikaten]
%
Folgende Eigenschaften sind für binäre Prädikate denkbar:
\begin{itemize}
\item Reflexivität $a \leftrightsquigarrow a$
\item Symmetrie $a \leftrightsquigarrow b \Rightarrow b \leftrightsquigarrow a$
\item Transitivität $a \leftrightsquigarrow b \wedge b
  \leftrightsquigarrow c
  \Rightarrow a \leftrightsquigarrow c$
\item Antisymmetrie $a \leftrightsquigarrow b \wedge  b
  \leftrightsquigarrow a \Rightarrow a = b$
\end{itemize}
\end{mantra}

\section{Eigenschaften von Funktionen auf Listen}

Es wird Zeit, Eigenschaften von selbstgeschriebenen Funktionen zu
überprüfen.  In diesem Abschnitt geht es um einige der Funktionen, die
auf Listen operieren: \texttt{concatenate}, \texttt{invert},
und \texttt{list-sum}.

\subsection{Eigenschaften von \texttt{concatenate}}

Die Funktion
\texttt{concatenate}\index{concatenate@texttt{concatenate}} aus
Abschnitt~\ref{sec:concatenate} hängt zwei Listen aneinander.  Auch
\texttt{concatenate} ist assoziativ: Wenn drei Listen mit Hilfe von
\texttt{concatenate} aneinandergehängt werden, spielt es keine Rolle,
ob zuerst die ersten beiden oder zuerst die letzten beiden Listen
aneinandergehängt werden.  Nach dem Muster der Assoziativität von
\texttt{+} und \texttt{and} sieht der Test dieser Eigenschaft
folgendermaßen aus:
%
\begin{alltt}
(check-property
 (associativity concatenate (signature (list-of number)) \ldots))
\end{alltt}
%
Beim Test ist die Signatur von \texttt{lis-1}, \texttt{lis-2} und
\texttt{lis-3} jeweils \texttt{(list-of number)}.  Die Signatur von \texttt{concatenate}
%
\begin{alltt}
(: concatenate ((list-of %a) (list-of %a) -> (list-of %a)))
\end{alltt}
%
suggeriert allerdings, daß die Signatur von \texttt{lis-1},
\texttt{lis-2} und \texttt{lis-3} jeweils \texttt{(list-of \%a)} lauten
sollte, also allgemeiner als \texttt{(list-of number)}.  Signaturen mit
Signaturvariablen funktionieren allerdings nicht im Zusammenhang mit
Eigenschaften, wie folgendes Beispiel zeigt:
%
\begin{alltt}
(check-property
  (for-all ((x %a))
    ...))
\end{alltt}
%
Dieser Code liefert die Fehlermeldung "<Signatur hat keinen
Generator">: Das liegt daran, daß die Signaturvariable \texttt{\%a}
zuwenig Information über die zugrundeliegenden Werte liefert, als daß
\drscheme{} sinnvoll Werte für die Tests generieren könnte.  Aus diesem
Grund müssen in \texttt{for-all} immer "<konkrete"> Signaturen ohne
Signaturvariablen angegeben werden.  (Aus ähnlichen Gründen
funktionieren auch einige andere Arten von Signaturen nicht bei
\texttt{for-all}, inbesondere Record-Signaturen.  Funktionsignaturen sind
allerdings zulässig und werden in Abschnitt~\ref{sec:ho-props} behandelt.)

Für \texttt{concatenate} wäre es zwar gründlicher, die Tests auch noch
für andere Sorten von Listenelementen als \texttt{number}
durchzuführen~-- da aber \texttt{concatenate} mit den Listenelementen
nichts anfängt, außer sie in weitere Liste zu stecken, reicht die
Formulierung der Eigenschaft mit \texttt{(list-of number)} aus.

Es bleibt noch ein weiteres Problem bei der Formulierung der
Assoziativität für \texttt{concatenate}: Es steht noch keine Funktion
für den Vergleich der beiden Listen zur Verfügung, die muß erst noch
geschrieben werden.  Kurzbeschreibung und Signatur:\index{number-list=?@texttt{number-list=?}}
%
\begin{alltt}
; Zwei Listen aus Zahlen vergleichen
(: number-list=? ((list-of number) (list-of number) -> boolean))
\end{alltt}
%
Die Testfälle sollten insbesondere Listen unterschiedlicher Länge
berücksichtigen:
%
\begin{alltt}
(check-expect (number-list=? empty empty) #t)
(check-expect (number-list=? (list 1.0 2.0 3.0) (list 1.0 2.0 3.0)) #t)
(check-expect (number-list=? (list 1.0 2.0 3.0) (list 1.0 2.0)) #f)
(check-expect (number-list=? (list 1.0 2.0) (list 1.0 2.0 3.0)) #f)
(check-expect (number-list=? (list 1.0 2.0 3.0) (list 1.0 2.1 3.0)) #f)
\end{alltt}
%
Die Schablone für Listen als Eingabe, ausgewählt nach dem ersten Listenparameter
\texttt{lis-1}, sieht so aus:
% 
\begin{alltt}
(define number-list=?
  (lambda (lis-1 lis-2)
    (cond
      ((empty? lis-1)
       ...)
      ((cons? lis-1)
       ... (first lis-1) ...
       ... (number-list=? (rest lis-1) ...) ...))))
\end{alltt}
%
Die Schablone für den zweiten Listenparameter \texttt{lis-2} wird in
beide Zweige des \texttt{cond} eingesetzt:
%
\begin{alltt}
(define number-list=?
  (lambda (lis-1 lis-2)
    (cond
      ((empty? lis-1)
       (cond
         ((empty? lis-2) ...)
         ((cons? lis-2)
          ... (first lis-2) ...
          ... (number-list=? ... (rest lis-2)))))
      ((cons? lis-1)
       ... (first lis-1) ...
       ... (number-list=? (rest lis-1) ...) ...
       (cond
         ((empty? lis-2) ...)
         ((cons? lis-2)
          ... (first lis-2) ...
          ... (number-list=? ... (rest lis-2))))))))
\end{alltt}
%
Es gibt also insgesamt vier Fälle bei den Verzweigungen:
\begin{itemize}
\item Im ersten
Fall sind beide Listen leer, das Ergebnis ist also \texttt{\#t}.
\item Im zweiten Fall ist die erste Liste leer und die zweite
  nichtleer.  Das Ergebnis ist also \texttt{\#f} und die
  Schablonenelemente sind überflüssig.
\item Im dritten Fall ist die erste Liste nichtleer und die zweite
  leer.  Das Ergebnis ist also wiederum \texttt{\#f}.
\item Im vierten Fall sind beide Listen nichtleer und in der Schablone
  stehen die jeweils ersten Elemente von \texttt{lis-1} und
  \texttt{lis-2}.  Die beiden Listen sind nur gleich, wenn die beiden
  ersten Elemente gleich sind.  Außerdem müssen natürlich die beiden
  Reste der Listen ebenfalls gleich sind~-- die beiden rekursiven
  Aufrufe aus den Schablonen können also kombiniert werden:
\end{itemize}
%
\begin{alltt}
(define number-list=?
  (lambda (lis-1 lis-2)
    (cond
      ((empty? lis-1)
       (cond
         ((empty? lis-2) #t)
         ((cons? lis-2) #f)))
      ((cons? lis-1)
       (cond
         ((empty? lis-2) #f)
         ((cons? lis-2)
          (and (= (first lis-1) (first lis-2))
               (number-list=? (rest lis-1) (rest lis-2)))))))))
\end{alltt}
%
Damit kann jetzt die Assoziativität von \texttt{concatenate} getestet werden:
%
\begin{alltt}
(check-property
 (associativity concatenate (signature (list-of number)) number-list=?))
\end{alltt}
%
\texttt{Concatenate} hat außerdem ein neutrales Element, und zwar
sowohl im linken als auch im rechten Argument:
%
\begin{alltt}
(check-property
 (for-all ((lis (list-of number)))
   (number-list=? lis (concatenate empty lis))))

(check-property
 (for-all ((lis (list-of number)))
   (number-list=? lis (concatenate lis empty))))
\end{alltt}
%
\texttt{Concatenate} ist allerdings demonstrierbar nicht kommutativ.
Der entsprechende Test sieht so aus:
%
\begin{alltt}
(check-property
 (commutativity concatenate (signature (list-of number)) number-list=?))
\end{alltt}
%
\drscheme{} liefert hierfür ein Gegenbeispiel:
%
\begin{alltt}
Eigenschaft falsifizierbar mit
         lis-1 = \framebox{#<list -3.75>} lis-2 = \framebox{#<list 1.5 1.5>}
\end{alltt}

\subsection{Eigenschaften von \texttt{number-list=?}}

Wie der Zufall so will, hat auch die Hilfsfunktion
\texttt{number-list=?} interessante Eigenschaften: Wie \texttt{=} muß
auch \texttt{number-list=?} eine Äquivalenzrelation sein~--
schließlich testet sie wie \texttt{=} auf Gleichheit.  Die
dazugehörigen Eigenschaften~-- Reflexivität, Symmetrie und
Transitivität~-- können ebenso wie bei \texttt{=} formuliert werden:

Reflexivität:
%
\begin{alltt}
(check-property
 (for-all ((lis (list-of number)))
   (number-list=? lis lis)))
\end{alltt}
Symmetrie:
\begin{alltt}
(check-property
  (for-all ((lis-1 (list-of number))
            (lis-2 (list-of number)))
    (==> (number-list=? lis-1 lis-2)
         (number-list=? lis-2 lis-1))))
\end{alltt}
 Transitivität
\begin{alltt}
(check-property
 (for-all ((lis-1 (list-of number))
           (lis-2 (list-of number))
           (lis-3 (list-of number)))
   (==> (and (number-list=? lis-1 lis-2)
             (number-list=? lis-2 lis-3))
        (number-list=? lis-1 lis-3))))
\end{alltt}
%

\subsection{Eigenschaften von \texttt{invert}}

Die Funktion \texttt{invert} aus Abschnitt~\ref{sec:invert} dreht die
Reihenfolge der Elemente einer Liste um.  Eine naheliegende
Eigenschaft von \texttt{invert} ist, daß zweimaliges Umdrehen wieder
die Ursprungsliste liefern sollte:
%
\begin{alltt}
(check-property
 (for-all ((lis (list-of number)))
   (number-list=? lis (invert (invert lis)))))
\end{alltt}
%
Auch bei \texttt{invert} enthält die Signatur eine Signaturvariable:
%
\begin{alltt}
(: invert ((list-of %a) -> (list-of %a)))
\end{alltt}
%
Genau wie bei \texttt{concatenate} macht \texttt{invert} mit den
Listenelementen nichts spezielles, es können also auch zum Beispiel Zeichenketten
benutzt werden.  Diese Änderung allein funktioniert allerdings nicht:
%
\begin{alltt}
(check-property
 (for-all ((lis (list-of string)))
   (\textbf{number-list=?} lis (invert (invert lis)))))
\end{alltt}
%
Die Funktion \texttt{number-list=?} funktioniert nur auf Listen von
Zahlen.  Es wäre möglich, \texttt{number-list=?} über der
Vergleichsfunktion auf den Elementen zu abstrahieren, aber es wäre
trotzdem umständliche Arbeit nur für den Zweck des Testens.  Deshalb 
gibt es eine Vereinfachung analog zu \texttt{check-expect}.  Die eingebaute Form \texttt{expect}
akzeptiert zwei beliebige Werte und ist dann erfüllt, wenn diese Werte
gleich sind.  (Siehe Abbildung~\ref{scheme:expect}.)  Die Eigenschaft
von \texttt{invert} sieht damit so aus:

\begin{feature}{\texttt{expect}}{scheme:expect}
  \texttt{Expect}\index{expect@\texttt{expect}} liefert eine
  Eigenschaft analog zur Funktionsweise von
  \texttt{check-expect}.  Ein \texttt{expect}-Ausdruck hat folgende
  Form:
  %
\begin{alltt}
(expect \(e\sb{1}\)  \(e\sb{2}\))
\end{alltt}
%
$e_1$ und $e_2$ sind Ausdrücke.  Die resultierende Eigenschaft ist
erfüllt, wenn $e_1$ und $e_2$ den gleichen Wert liefern~-- der
Vergleich wird dabei wie bei \texttt{check-expect} angestellt.

\end{feature}

\begin{alltt}
(check-property
 (for-all ((lis (list-of string)))
   (expect lis (invert (invert lis)))))
\end{alltt}
%
Viele Funktionen auf Listen haben Eigenschaften, welche die Funktion
jeweils im Zusammenspiel mit einer oder mehreren anderen Funktionen
zeigen.  Bei Funktionen mit Listen ist es häufig interessant, das
Zusammenspiel mit \texttt{concatenate} zu betrachten.  Damit
\texttt{concatenate} etwas sinnvolles tun kann, sind zwei Listen
notwendig:
%
\begin{alltt}
(check-property
  (for-all ((lis-1 (list-of number))
            (lis-2 (list-of number)))
    ...))
\end{alltt}
%
Auf diese zwei Listen kann \texttt{concatenate} aber auch jeweils
\texttt{invert} angewendet werden:
%
%
\begin{alltt}
(check-property
  (for-all ((lis-1 (list-of number))
            (lis-2 (list-of number)))
    ... (invert lis-1) ...
    ... (invert lis-2) ...
    ... (invert (concatenate lis-1 lis-2)) ...))
\end{alltt}
%
Wie läßt sich die Liste \texttt{(invert (concatenate lis-1 lis-2))}
noch beschreiben?
Angenommen, \texttt{lis-1} ist die Liste \texttt{\#<list 1 2 3>} und
\texttt{lis-2} die Liste \texttt{\#<list 4 5 6>}.  Dann gilt:
%
\begin{alltt}
(invert (concatenate lis-1 lis-2))
\(=\)
(invert (concatenate #<list 1 2 3> #<list 4 5 6>))
\(\Longrightarrow\) \ldots{} \(\Longrightarrow\) (invert #<list \(\underbrace{\texttt{1 2 3}}\sb{\texttt{lis-1}}\) \(\underbrace{\texttt{4 5 6}}\sb{\texttt{lis-2}}\)>))

\(\Longrightarrow\) \ldots{} \(\Longrightarrow\) #<list \(\underbrace{\texttt{6 5 4}}\sb{\texttt{(invert lis-2)}}\) \(\underbrace{\texttt{3 2 1}}\sb{\texttt{(invert lis-1)}}\)>
\end{alltt}
%
Dies läßt vermuten, daß die gesuchte Eigenschaft folgendermaßen aussieht:
%
\begin{alltt}
(check-property
 (for-all ((lis-1 (list-of number))
           (lis-2 (list-of number)))
   (expect (invert (concatenate lis-1 lis-2))
           (concatenate (invert lis-2) (invert lis-1)))))
\end{alltt}
%
\begin{mantra}[Eigenschaften von Funktionen auf Listen]
  Funktionen, die Listen akzeptieren, haben häufig interessante
  Eigenschaften im Zusammenspiel mit \texttt{concatenate}.
\end{mantra}

\subsection{Eigenschaften von \texttt{list-sum}}

\texttt{List-sum}\index{list-sum@\texttt{list-sum}} aus
Abschnitt~\ref{sec:list-sum} ist, wie \texttt{invert}, eine Funktion,
die eine Liste akzeptiert.  Genau wie bei \texttt{invert} ist es eine
gute Idee, die Interaktion zwischen \texttt{list-sum} und
\texttt{concatenate} zu untersuchen.  Es müssen also wieder zwei
Listen her~-- die zu \texttt{invert} analoge Vorgehensweise liefert
folgende Schablone:
%
\begin{alltt}
(check-property
 (for-all ((lis-1 (list-of number))
           (lis-2 (list-of number)))
    ... (list-sum lis-1) ...
    ... (list-sum lis-2) ...
    ... (list-sum (concatenate lis-1 lis-2)) ...
\end{alltt}
%
Da \texttt{list-sum} die Elemente der Liste addiert und die Addition
assoziativ ist, müßte folgendes gelten:
%
\begin{alltt}
(check-property
 (for-all ((lis-1 (list-of number))
           (lis-2 (list-of number)))
    (expect (+ (list-sum lis-1) (list-sum lis-2))
            (list-sum (concatenate lis-1 lis-2)))))
\end{alltt}
%
Hier allerdings schlägt das Rundungsproblem aus
Abschnitt~\ref{sec:plus-not-associative} wieder zu: Die Addition auf
\texttt{number} ist eben \emph{nicht} assoziativ, aber immerhin auf
\texttt{rational}.  Der fertige Test muß also so aussehen:
%
\begin{alltt}
(check-property
 (for-all ((lis-1 (list-of rational))
           (lis-2 (list-of rational)))
    (expect (+ (list-sum lis-1) (list-sum lis-2))
            (list-sum (concatenate lis-1 lis-2)))))
\end{alltt}
%
Eine Alternative ist die Verwendung der Form
\texttt{expect-within}\index{expect-within@\texttt{expect-within}},
die eine Eigenschaft analog zu \texttt{check-within} erzeugt.  (Siehe
Abbildung~\ref{scheme:expect-within}.)

\begin{feature}{\texttt{expect-within}}{scheme:expect-within}
  \texttt{Expect-within}\index{expect-within@\texttt{expect-within}} liefert eine
  Eigenschaft analog zur Funktionsweise von
  \texttt{check-within}.  Ein \texttt{expect-within}-Ausdruck hat folgende
  Form:
  %
\begin{alltt}
(expect-within \(e\sb{1}\)  \(e\sb{2}\) \(e\sb{\delta}\))
\end{alltt}
%
$e_1$, $e_2$ und $e\sb{\delta}$ sind Ausdrücke, wobei $e\sb{\delta}$
eine reelle Zahl liefern muß.  Die resultierende Eigenschaft ist
erfüllt, wenn $e_1$ und $e_2$ den gleichen Wert liefern~-- der
Vergleich wird dabei wie bei \texttt{check-within} angestellt, das heißt
alle inexakten Zahlen in den Ergebnissen von $e_1$ und $e_2$ müssen
nicht gleich sein, dürfen sich aber höchstens um $e\sb{\delta}$
voneinander unterscheiden.
\end{feature}

Mit \texttt{expect-within} sieht der Testfall folgendermaßen aus:
%
\begin{alltt}
(check-property
 (for-all ((lis-1 (list-of number))
           (lis-2 (list-of number)))
    (expect-within (+ (list-sum lis-1) (list-sum lis-2))
                   (list-sum (concatenate lis-1 lis-2))
                   0.1)))
\end{alltt}
%
Auch dieser Testfall läuft durch.

Auch der Test für die Assoziativität von \texttt{+} aus
Abschnitt~\ref{sec:plus-not-associative} kann mit
\texttt{expect-within} formuliert werden:
%
\begin{alltt}
(check-property
 (for-all ((a number)
           (b number)
           (c number))
    (expect-within (+ a (+ b c))
                   (+ (+ a b) c)
                   0.1)))
\end{alltt}
%
So wie sich die Assoziativität von \texttt{+} in einer Eigenschaft von
\texttt{list-sum} niederschlägt, tut dies auch die Kommutativität: Sie
besagt, daß die Reihenfolge der Elemente der Liste keine Rolle spielt.
Eine einfache Möglichkeit, dies zu testen, ist wiederum mit zwei
Listen zu arbeiten und diese einmal in der einen und dann in der
anderen Richtung aneinanderzuhängen:\label{sec:list-sum-commutative}
%
\begin{alltt}
(check-property
 (for-all ((lis-1 (list-of rational))
           (lis-2 (list-of rational)))
   (expect (list-sum (concatenate lis-1 lis-2))
           (list-sum (concatenate lis-2 lis-1)))))
\end{alltt}

\section{Eigenschaften von Funktionen höherer Ordnung}
\label{sec:ho-props}

In Abschnitt~\ref{sec:currying} wurde bereits eine Eigenschaft von
\texttt{curry}\index{curry@\texttt{curry}} und
\texttt{uncurry}\index{uncurry@\texttt{uncurry}} aufgeführt:
%
\begin{center}
  \texttt{(uncurry (curry \(p\))) \(\equiv\) \(p\)}
\end{center}
%
Mit anderen Worten: \texttt{curry} und \texttt{uncurry} sind jeweils
Inverse voneinander.  Auch diese Eigenschaft läßt sich direkt mit
\texttt{check-property} und \texttt{for-all} formulieren.  Zu beachten
ist wieder, obwohl \texttt{curry} und \texttt{uncurry} polymorphe
Signaturen mit Signaturvariablen haben, daß für den Test mit
\texttt{check-property} eine "<konkrete"> Signatur ohne
Signaturvariablen für das Funktion-Argument benutzt werden muß, also
zum Beispiel \texttt{string}:
%
\begin{alltt}
(check-property
 (for-all ((proc (string string -> string)))
   (expect (curry (uncurry proc))
           proc)))
\end{alltt}
%
Leider schlägt dieser Test fehl, und zwar mit einer mysteriösen
Meldung:
%
\begin{alltt}
Eigenschaft falsifizierbar mit \framebox{#<procedure:?>}
\end{alltt}
%
Offenbar ist \drscheme{} also der Ansicht, es hat eine Funktion gefunden,
welche die Eigenschaft nicht erfüllt, kann aber nicht genau sagen,
welche Funktion: Das liegt daran, daß es prinzipiell unmöglich ist,
zwei Funktionen auf Gleichheit zu überprüfen~-- Gleichheit zweier
Funktionen heißt ja, daß die eine Funktion angewendet auf einen Satz
Argumente \emph{immer} das gleiche Ergebnis wie die andere Funktion
liefert.  Im obigen Beispiel akzeptiert \texttt{proc} zwei
Zeichenketten, von denen es unendlich viele gibt; die Gleichheit zu
überprüfen, würde also unendlich lange dauern.  \texttt{Expect}
versucht es darum gar nicht erst, sondern sieht es als notwendige
(nicht als hinreichende) Bedingung für die Gleichheit zweier
Funktionen an, daß sie Wert desselben \texttt{lambda}-Ausdrucks sind.
\texttt{Expect} testet also bei Funktionen auf sogenannte
\textit{intensionale Gleichheit}\index{intensionale
  Gleichheit}\index{Gleichheit!intensional}, was soviel heißt, daß
\texttt{expect} vergleicht, \emph{auf welche Weise} die beiden
Funktionen entstanden sind, nicht aber, ob sich die beiden Funktionen
\emph{gleich verhalten}.  Die letztere Eigenschaft heißt \textit{extensionale
  Gleichheit}\index{Gleichheit!extensional}\index{extensionale
  Gleichheit}~-- und ist, wie gesagt, nicht effektiv testbar.

Der \texttt{lambda}-Ausdruck der Funktion, die von \texttt{(curry
  (uncurry proc))} geliefert wird, ist aber der Rumpf von
\texttt{curry}, während der \texttt{lambda}-Ausdruck von \texttt{proc}
i.d.R.\ woanders steht; damit sind die beiden Funktionen intensional
ungleich, und der obige Test muß fehlschlagen, auch wenn die beiden
Operanden von \texttt{expect} äquivalent sind.

Damit ein \texttt{check-property}-Test die Äquivalenz testen kann, muß
er selbst \texttt{(curry (uncurry proc))} anwenden:
%
\begin{alltt}
(check-property
 (for-all ((a string)
           (b string)
           (proc (string string -> string)))
    (expect ((uncurry (curry proc)) a b)
            (proc a b))))
\end{alltt}
%
Dieser Test funktioniert.

\section{Programme beweisen}

\texttt{Check-property} ist nützlich, um zu überprüfen, ob eine
Eigenschaft gilt oder nicht.  Da \texttt{check-property} allerdings
nur eine endliche Menge zufälliger Tests durchführt, reicht es nicht
aus, um sicherzugehen, daß eine bestimmte Eigenschaft für alle Werte
der \texttt{for-all}-Variablen gilt: Dazu ist ein mathematischer
Beweis notwendig.

An verschiedenen Stellen im Buch wurden Beweise für mathematische
Funktionen durchgeführt~-- zuletzt in Kapitel~\ref{cha:indu} für
eine rekursive Funktion.  Beweise über mathematische Funktionen
erlauben, bei jedem Schrittan beliebigen Stellen Gleichungen
einzusetzen.  Beweise über Funktionen in Programmen sind schwieriger,
da sie das Substitutionsmodell berücksichtigen müssen: Bei jedem
Reduktionsschritt kommt immer nur eine bestimmte Substitution in
Frage.

\subsection{Arithmetische Gesetze beweisen}
\label{sec:scheme-arithmetik-beweis}

Als erstes Beispiel für den Beweis an einem Programm dient der Beweis
der Kommutativität von \texttt{+}.  Der Beweis ist nicht besonders
tiefgreifend, demonstriert aber die wichtigsten Techniken, die beim
Beweisen von Programmen zum Einsatz kommen.  Zu beweisen ist:
%
\begin{alltt}
(= (+ a b) (+ b a))
\(\Longrightarrow\ldots\Longrightarrow\) \#t
\end{alltt}
%
\ldots{} und zwar für beliebige Bindungen von \texttt{a} und
\texttt{b} an Zahlen.  Seien die Zahlen, die an \texttt{a} bzw.\
\texttt{b} gebunden sind, $x$ und $y$ mit $x,y\in\mathbb{C}$.  (Die
"<mathematischen"> Namen könnten auch $a$ und $b$ sein, aber das birgt
ein Verwirrungsrisiko mit \texttt{a} und \texttt{b}.)  Wenn nun also
der obige Term für bestimmte Werte von $x$ und $y$ im
Substitutionsmodell reduziert wird, wird zuerst $x$ für \texttt{a} und
$y$ für \texttt{b} eingesetzt.  Für $x=5$ und $y=17$ also:
%
\begin{alltt}
(= (+ 5 17) (+ 17 5))
\end{alltt}
%
FIXME: \valof{\_} schwierig

Der Beweis soll aber für beliebige Werte für $x$ und $y$
funktionieren: $x$ und $y$ müssen also im Beweis auftauchen.
Um den Unterschied zwischen Variablen des Programms \texttt{a} und
\texttt{b} und den Zahlen zu machen, die für $x$ und $y$ eingesetzt
werden, werden diese noch mit \valof{\_} umgeben: \valof{x} in
einem Reduktionsschritt des Substitutionsmodell ist also ein
Platzhalter für "<die Zahl, für die $x$ steht">~-- entsprechend für $y$.
Es gilt also:

\begin{alltt}
(= (+ a b) (+ b a))
\(=\) (= \redex{(+ \valof{x} \valof{y})} (+ \valof{y} \valof{x}))
\end{alltt}
%
Dort ist der erste Teilausdruck unterstrichen, der beim ersten
Substitutionsschritt ersetzt wird.  Wenn die Programm-Funktion
\texttt{+} tatsächlich die mathematische Operation $+$
realisiert,\footnote{Die Komplikationen durch inexakte Zahlen und
  Rundungen bleiben hier unberücksichtigt.}
wird der Teilausdruck \texttt{(+ \valof{x} \valof{y})} durch $x+y$
ersetzt~-- beziehungsweise durch die Zahl, für die der mathematische
Ausdruck $x+y$ steht.  Es kommt also wieder \valof{\_} zum Einsatz:
%
\begin{alltt}
(= \redex{(+ \valof{x} \valof{y})} (+ \valof{y} \valof{x}))
\(\Longrightarrow\) (= \valof{x+y} \redex{(+ \valof{y} \valof{x})})
\end{alltt}
%
Entsprechend geht es weiter mit der zweiten Summe und schließlich der
Vergleichsoperation \texttt{=}, die dem mathematischen $=$ entspricht:
%
\begin{alltt}
\(\Longrightarrow\) \redex{(= \valof{x+y} \valof{y+x})}
\(\Longrightarrow\) \valof{x+y = y+x}
\(=\) \#t
\end{alltt}
%
Die Kommutativität der Programm-Funktion \texttt{+} folgt also aus der
Kommutativität des mathematischen $+$, durch das sie definiert ist.

\section{Rekursive Programme beweisen}
\label{sec:rek-scheme-beweisen}

Beweise über rekursive Programme sind anspruchsvoller als der
Beweis der Kommutativität von \texttt{+}, benutzen aber die gleichen
Techniken sowie~-- genau wie bei Beweisen mathematischer rekursiver
Funktionen~-- Induktion als Beweisprinzip. 

\subsection{Rekursive Programme auf Listen}

Als erstes Beispiel dient
die Reflexivität.  Es gilt für die Bindung von \texttt{lis} an eine
beliebige Liste von Zahlen folgendes zu beweisen:
%
\begin{alltt}
(number-list=? lis lis)
\(\Longrightarrow\ldots\Longrightarrow\) \#t
\end{alltt}
%
Wieder wird für den Wert der Bindung eine mathematische Variable
eingeführt~-- $l$.  Dann läuft der Beweis auf folgendes hinaus:
%
\begin{alltt}
(number-list=? lis lis)
\(=\) (number-list=? \valof{l} \valof{l})
\(\Longrightarrow\ldots\Longrightarrow\) \#t
\end{alltt}
%
\ldots~und dies ist zu beweisen für \emph{alle} Listen von Zahlen $l$.
Für diesen Beweis kommt uns die induktive Struktur der Listen zur
Hilfe, die der Struktur der endlichen Folgen~-- bekannt aus
Abschnitt~\ref{sec:finite-sequences} auf
Seite~\pageref{sec:finite-sequences}~-- entspricht.  Dazu müssen wir
zunächst einmal zwischen allen Fällen der gemischten
Datendefinition unterscheiden, also zwischen den leeren und den nichtleeren Listen.

\paragraph{Leere Liste}
Angenommen, $l$ ist die leere Liste.  Dann beginnt die Reduktion
folgendermaßen:
%
\begin{alltt}
(number-list=? \valof{l} \valof{l})
\(\Longrightarrow\) ((lambda (lis-1 lis-2) \ldots{}) \valof{l} \valof{l})
\(\Longrightarrow\) (cond ((empty? \valof{l}) \ldots) ((cons? \valof{l}) \ldots))
\(\Longrightarrow\) (cond ((empty? \valof{l}) \ldots) ((cons? \valof{l}) \ldots))
\(\Longrightarrow\) (cond (#t (cond \ldots)) ((cons? \valof{l}) #f))
\(\Longrightarrow\) (cond ((empty? \valof{l}) #t) ((cons? \valof{l}) #f))
\(\Longrightarrow\) (cond (#t #t) ((cons? \valof{l}) #f))
\(\Longrightarrow\) #t
\end{alltt}
%

\paragraph{Nichtleere Liste}
Für diesen Fall stimmt die Behauptung also.  Angenommen, $l$ ist
\emph{nicht} die leere Liste, hat also erstes Element $f$ und Rest
$r$.  In diesem Fall können wir strukturelle Induktion benutzen.  Die
\emph{Induktionsvoraussetzung} bezieht sich dann auf den Rest $r$:
%
\begin{alltt}
(number-list=? \valof{r} \valof{r})
\(\Longrightarrow\ldots\Longrightarrow\) #t
\end{alltt}
%
Diese können wir benutzen.  Zunächst einmal müssen wir aber soweit
reduzieren wie möglich:
%
\begin{alltt}
(number-list=? \valof{l} \valof{l})
\(\Longrightarrow\) ((lambda (lis-1 lis-2) \ldots{}) \valof{l} \valof{l})
\(\Longrightarrow\) (cond ((empty? \valof{l}) \ldots) ((cons? \valof{l}) \ldots))
\(\Longrightarrow\) (cond (#f \ldots) ((cons? \valof{l}) \ldots))
\(\Longrightarrow\) (cond (#t (cond \ldots)))
\(\Longrightarrow\) (cond ((empty? \valof{l}) \ldots) ((cons? \valof{l}) \ldots))
\(\Longrightarrow\) (cond (#f \ldots) ((cons? \valof{l}) \ldots))
\(\Longrightarrow\) (cond (#t (and \ldots)))
\(\Longrightarrow\) (and (= (first \valof{l}) (first \valof{l})) (number-list=? \ldots))
\end{alltt}
%
Da \texttt{(first \valof{l})} das erste Element $f$ liefert, geht es
so weiter:
\begin{alltt}
\(\Longrightarrow\) (and (= \valof{f} (first \valof{l})) (number-list=? \ldots))
\(\Longrightarrow\) (and (= \valof{f} \valof{f}) (number-list=? \ldots))
\(\Longrightarrow\) (and #t (number-list=? \ldots))
\(\Longrightarrow\) (number-list=? (rest \valof{l}) (rest \valof{l}))
\end{alltt}
%
Da \texttt{(rest \valof{l})} den Rest $r$ liefert, geht es dann so
weiter:
%
\begin{alltt}
\(\Longrightarrow\) (number-list=? \valof{r} (rest \valof{l}))
\(\Longrightarrow\) (number-list=? \valof{r} \valof{r})
\end{alltt}
%
Nach der Induktionsvoraussetzung wissen wir aber, daß der letzte
Ausdruck zu \verb|#t| reduziert wird.  Die Behauptung ist damit
bewisen.


\subsection{Rekursive Programme auf Zahlen}

Die Definition von \texttt{factorial}\index{factorial@\texttt{factorial}} am Anfang von
Abschnitt~\ref{sec:factorial} folgt der induktiven Definition der
zugrundeliegenden Daten, der natürlichen Zahlen.  Dementsprechend ist
der Induktionsbeweis für dessen Korrektheit einfach.  Es ist aber
entscheidend, die zu beweisende Eigenschaft, welche die Korrektheit
von \texttt{factorial} begründet, sorgfältig aufzuschreiben:

Die Funktion \texttt{factorial} soll die Fakultät berechnen, es soll
also für alle natürlichen Zahlen $k$ gelten:
%
\begin{alltt}
(factorial \valof{k})
\(\Longrightarrow\ldots\Longrightarrow\) \valof{k!}
\end{alltt}
%
(Diese Eigenschaft läßt sich nicht sinnvoll mit \texttt{for-all}
hinschreiben, da die mathematische Fakultät nicht fest eingebaut ist.)

Da es um natürliche Zahlen geht, ist vollständige Induktion
anwendbar.  Wir verwenden das Schema aus
Abschnitt~\ref{sec:nat-induction-ka} auf
Seite~\pageref{sec:nat-induction-ka}:
%
\begin{enumerate}
\item Die Behauptung ist bereits in der geforderten Form.
\item $k = 0$:
\begin{alltt}
(factorial 0)
\(\Longrightarrow\) \valof{0!}
\end{alltt}
%
\item Beweis für $k=0$:
\begin{alltt}
(factorial 0)
\(\Longrightarrow\) ((lambda (n) \ldots) 0)
\(\Longrightarrow\) (if (= 0 0) \ldots)
\(\Longrightarrow\) (if #t 1 \ldots)
\(\Longrightarrow\) 1
\(=\) \valof{0!}
\end{alltt}
%
\item Induktionsvoraussetzung:

\begin{alltt}
(factorial \valof{k})
\(\Longrightarrow\ldots\Longrightarrow\) \valof{k!}
\end{alltt}

\item Induktionsschluß (zu zeigen):

\begin{alltt}
(factorial \valof{k+1})
\(\Longrightarrow\ldots\Longrightarrow\) \valof{(k+1)!}
\end{alltt}
\item 

Der Beweis sieht so aus:
%
\begin{alltt}
(factorial \valof{k+1})
\(\Longrightarrow\) ((lambda (n) \ldots) \valof{k+1})
\(\Longrightarrow\) (if (= \valof{k+1} 0) \ldots)
\(\Longrightarrow\) (if #f 1 (* \ldots))
\(\Longrightarrow\) (* \valof{k+1} (factorial (- \valof{k+1} 1)))
\(\Longrightarrow\) (* \valof{k+1} (factorial \valof{k}))
\(=\) (* \valof{k+1} (factorial \valof{k}))
\end{alltt}
%
Mit der Induktionsannahme kann \texttt{(factorial \valof{k})} ersetzt werden:
%
\begin{alltt}
(* \valof{k} (factorial \valof{l}))
\(\Longrightarrow\ldots\Longrightarrow\) (* \valof{k+1} \valof{k!})
\(\Longrightarrow\) \valof{(k+1) \cdot k!}
\(=\) \valof{(k+1)!}
\end{alltt}
%
\end{enumerate}

Damit ist der Beweis fertig.

Die Technik funktioniert auch mit Beispielen, bei denen die zu
beweisende Eigenschaft nicht so einfach zu sehen ist wie bei
\texttt{factorial}.

Die folgende Funktion verrät nicht auf den ersten Blick, was
sie berechnet:
%
\begin{alltt}
(: f (natural -> rational))

(define f
  (lambda (n)
    (if (= n 0)
        0
        (+ (f (- n 1))
           (/ 1 (* n (+ n 1)))))))
\end{alltt}
%
Tatsächlich berechnet der Funktionaufruf \texttt{(f \valof{k})} für
eine natürliche Zahl $k$ die Zahl
\(\frac{k}{k+1}\).

Die Eigenschaft ist plausibel, wie sich mit \texttt{check-property}
feststellen läßt:
%
\begin{alltt}
(check-property
 (for-all ((k natural))
   (= (f k) (/ k (+ k 1)))))
\end{alltt}
%
Ein Beweis schafft Sicherheit.  Wieder gehen wir nach dem bekannten
Schema vor:
%
\begin{enumerate}
\item Behauptung:

  Für alle $k\in\mathbb{N}$ gilt:

\begin{alltt}
(f \valof{k})
\(\Longrightarrow\ldots\Longrightarrow\) \valof{\frac{k}{k+1}}
\end{alltt}

\item $k=0$:
%
\begin{alltt}
(f \valof{0})
\(\Longrightarrow\)\ldots\(\Longrightarrow\) \valof{\frac{0}{0+1}}
\end{alltt}

\item Beweis:
\begin{alltt}
(f \valof{0})
\(\Longrightarrow\) ((lambda (n) \ldots) 0)
\(\Longrightarrow\) (if (= 0 0) 0 \ldots)
\(\Longrightarrow\) (if #t 0 \ldots)
\(\Longrightarrow\) 0
\(=\) \valof{\frac{0}{0+1}}
\end{alltt}
%
\item Induktionsvoraussetzung:

\begin{alltt}
(f \valof{k})
\(\Longrightarrow\ldots\Longrightarrow\) \valof{\frac{k}{k+1}}
\end{alltt}

\item
Induktionsschluß (zu zeigen):
%
\begin{alltt}
(f \valof{k+1})
\(\Longrightarrow\ldots\Longrightarrow\) \valof{\frac{k+1}{(k+1)+1}}
\end{alltt}
%
\item Beweis:
\begin{alltt}
(f \valof{k+1})
\(\Longrightarrow\) ((lambda (n) (if \ldots)) \valof{k+1})
\(\Longrightarrow\) (if (= \valof{k+1} 0) \ldots)
\(\Longrightarrow\) (if #f \ldots (+ \ldots))
\(\Longrightarrow\) (+ (f (- \valof{k+1} 1)) (/ 1 (* \valof{k+1} (+ \valof{k+1} 1))))
\(\Longrightarrow\) (+ (f \valof{k}) (/ 1 (* \valof{k+1} (+ \valof{k+1} 1))))
\(\Longrightarrow\ldots\Longrightarrow\) (+ \valof{\frac{k}{k+1}} (/ 1 (* \valof{k+1} (+ \valof{k+1} 1))))
\(\Longrightarrow\) (+ \valof{\frac{k}{k+1}} (/ 1 (* \valof{k+1} \valof{(k+1)+1})))
\(\Longrightarrow\) (+ \valof{\frac{k}{k+1}} (/ 1 \valof{(k+1)\cdot((k+1)+1)}))
\(\Longrightarrow\) (+ \valof{\frac{k}{k+1}} \valof{\frac{1}{(k+1)\cdot((k+1)+1)}})
\(\Longrightarrow\) \valof{\frac{k}{k+1} + \frac{1}{(k+1)\cdot((k+1)+1)}}
\(=\) \valof{\frac{k}{k+1} + \frac{1}{(k+1)\cdot(k+2)}}
\(=\) \valof{\frac{k\cdot(k+2) + 1}{(k+1)\cdot(k+2)}}
\(=\) \valof{\frac{k\sp{2} + 2k + 1}{(k+1)\cdot(k+2)}}
\(=\) \valof{\frac{(k+1)\sp{2}}{(k+1)\cdot(k+2)}}
\(=\) \valof{\frac{k+1}{k+2}}
\(=\) \valof{\frac{k+1}{(k+1)+1}}
\end{alltt}
%
\end{enumerate}
Damit die die Behauptung bewiesen.

% Das gleiche Spiel läßt sich auch am Beispiel der Funktion \texttt{f}
% aus Abschnitt~\ref{sec:rek-scheme-beweisen} demonstrieren.  Eine
% iterative Variante von \texttt{f} ist:
% %
% \begin{alltt}
% (define f
%   (lambda (n)
%     (f-helper n 0)))

% (define f-helper
%   (lambda (n result)
%     (if (= n 0)
%         result
%         (f-helper (- n 1)
%                   (+ (/ 1 (* n (+ n 1)))
%                      result)))))
% \end{alltt}
% %
% Zur Erinnerung: \texttt{(f $n$)} liefert \(\frac{n}{n+1}\).

\section{Invarianten}
\label{sec:invarianten}

Die bisher angewendete Technik für den Beweis rekursiver Funktionen
mit Induktion funktioniert bei Funktionen mit Akkumulator nicht mehr
direkt: Angenommen, die Korrektheit der endrekursiven Fakuktät
\texttt{!} aus Abschnitt~\ref{page:factorial-tail} soll ähnlich die
die Korrektheit von \texttt{factorial} bewiesen werden.  Wieder sei
\texttt{n} an eine natürliche Zahl $k$ gebunden:
%
\begin{alltt}
(! n)
\(=\) (! \valof{k})
\(\Longrightarrow\) ((lambda (n) (!-helper n 1)) \valof{k})
\(\Longrightarrow\) (!-helper \valof{k} 1)
\(\Longrightarrow\) ((lambda (n acc) \ldots) \valof{k} 1)
\(\Longrightarrow\) (if (= \valof{k} 0) 1 (!-helper (- \valof{k} 1) (* 1 \valof{k})))
\end{alltt}
%
Wie bei \texttt{factorial} muß zwischen $k=0$ und $k>0$ unterschieden
werden.  Für $k=0$ geht es folgendermaßen weiter:
%
\begin{alltt}
\(\Longrightarrow\) (if (= \valof{k} 0) 1 (!-helper (- \valof{k} 1) (* 1 \valof{k})))
\(\Longrightarrow\) (if #t 1 (!-helper (- \valof{k} 1) (* 1 \valof{k})))
\(\Longrightarrow\) 1
\end{alltt}
%
Für $k=0$ funktioniert also der Beweis.  Für $k>0$ allerdings verläuft
die Reduktion folgendermaßen:
%
\begin{alltt}
\(\Longrightarrow\) (if (= \valof{k} 0) 1 (!-helper (- \valof{k} 1) (* 1 \valof{k})))
\(\Longrightarrow\) (if #f 1 (!-helper (- \valof{k} 1) (* 1 \valof{k})))
\(\Longrightarrow\) (!-helper (- \valof{k} 1) (* 1 \valof{k}))
\end{alltt}
%
Hier gibt es zwar einen rekursiven Aufruf mit Argument \texttt{(-
  \valof{k} 1)}, aber \emph{der Akkumulator hat sich auch verändert.}
Damit ist die naheliegende Induktionsannahme für \texttt{(!-helper (-
  \valof{k} 1) \valof{a})} (falls der Wert des Akkumulators
\texttt{acc} $a$ ist) wertlos.  Funktionen mit Akkumulator sind also
nicht nur schwieriger zu schreiben als "<regulär"> rekursive
Funktionen~-- sie sind auch schwerer zu beweisen.

Stattdessen ist es bei Funktionen mit Akkumulator nützlich, eine
\textit{Invariante\index{Invariante}} aufzustellen, also eine
Eigenschaft, welche Zwischenergebnis und noch zu leistende Arbeit in
Beziehung setzt.  Wie in Abschnitt~\ref{page:factorial-tail}
beschrieben, geht die Fakultätsfunktion mit Akkumulator folgendermaßen
vor, um \texttt{(! 4)} auszurechnen:
%
\begin{displaymath}
  (((1 \cdot 4)\cdot 3)\cdot 2)\cdot 1
\end{displaymath}
%
Bei jedem rekursiven Aufruf läßt sich dieser Aufruf in "<geleistete
Arbeit"> (die durch den Akkumulator repräsentiert ist) und "<noch zu
leistende Arbeit"> unterteilen.  Zum Beispiel entsteht ein rekursiver
Aufruf \texttt{(!-helper 2 12)}, bei der Akkumulator der Wert des
unterstrichenen Teilaufrufs ist:
%
\begin{displaymath}
  (\underline{((1 \cdot 4)\cdot 3)}\cdot 2)\cdot 1
\end{displaymath}
%
Es ist zu sehen, daß die noch zu leistende Arbeit gerade darin
besteht, den Akkumulator noch mit der Fakultät von 2 zu
multiplizieren.  Wenn bei einem rekursiven Aufruf von \texttt{!-helper} der
Wert von \texttt{n} $k$ ist und der Wert des Akkumulators $a$, und am
Ende die Fakultät von $N$ berechnet werden soll, dann gilt bei jedem
rekursiven Aufruf \texttt{(!-helper \valof{n} \valof{a})} immer:
%
\begin{displaymath}
  a \cdot k! = N!
\end{displaymath}
%
Dies ist die Invariante von \texttt{!-helper} und heißt so, weil sie beim
rekursiven Aufruf von \texttt{!-helper} unverändert bleibt.  Dies
ist zunächst nur eine Behauptung, aber wenn sie gelten sollte, dann
folgt daraus automatisch die Korrektheit der Funktion, da bei $k=0$
gilt:
%
\begin{displaymath}
  a \cdot 0! = a \cdot 1 = a = N!
\end{displaymath}
%
Daß $a \cdot k! = N!$ tatsächlich die Invariante ist, läßt sich
folgendermaßen folgern:
%
\begin{itemize}
\item Sie gilt für den ersten Aufruf von \texttt{!-helper} von
  \texttt{!}, da dort $k=N$ und $a=1$ gilt, also:
  \begin{displaymath}
    a \cdot k! = 1 \cdot N! = N!
  \end{displaymath}
\item Jeder rekursive Aufruf erhält die Invariante.  Angenommen, sie
  gilt für $k$ und $a$, dann sind die neuen Werte für $k$ und $a$ im
  rekursiven Aufruf \texttt{(!-helper (- \valof{n} 1) (* \valof{a} \valof{k}))} gerade
  $k\mapsto k-1$ und $a \mapsto a\cdot n$, die Invariante wäre also:
  \begin{eqnarray*}
    (a \cdot k) \cdot (k-1)! &=& a \cdot (k \cdot (k-1)!\\
    &=& a\cdot k!\\
    &=& N!
  \end{eqnarray*}
\end{itemize}
%
Diese Technik funktioniert auch bei weniger offensichtlichen
Funktionen.  Hier eine Funktion, die äquivalent ist zu der Funktion
\texttt{f} aus Abschnitt~\ref{sec:rek-scheme-beweisen}:

\begin{alltt}
; n/(n+1) berechnen
(: f (natural -> natural))

(define f
  (lambda (n)
    (f-helper n 0)))

(define f-helper
  (lambda (n acc)
    (if (= n 0)
        acc
        (f-helper (- n 1)
                  (+ (/ 1 (* n (+ n 1)))
                     acc)))))
\end{alltt}
%
Die Funktion geht folgendermaßen vor, um das Ergebnis für eine Eingabe
$n$ zu berechnen:
%
\begin{displaymath}
  (\ldots ((\frac{1}{n \cdot (n + 1)}
  + \frac{1}{(n - 1) \cdot ((n - 1) + 1)})
  + \frac{1}{(n - 2) \cdot ((n - 2) + 1)})
  + \ldots
  + \frac{1}{1 \cdot (1 + 1)})
\end{displaymath}
%
Diese Summe ist bei jedem rekursiven Aufruf aufgeteilt als Summe von
zwei Teilen, z.B.\ wie folgt:
%
\begin{displaymath}
  (\ldots \underline{((\frac{1}{n \cdot (n + 1)}
  + \frac{1}{(n - 1) \cdot ((n - 1) + 1)})
  + \frac{1}{(n - 2) \cdot ((n - 2) + 1)})}
  + \ldots
  + \frac{1}{1 \cdot (1 + 1)})
\end{displaymath}
%
Der unterstrichene Teil ist gerade der Wert des Akkumulators, die
Summe rechts davon der noch zu berechnende Summand.  
Wenn die Funktion tatsächlich $n/(n+1)$ berechnen sollte, ist dieser
rechte Teil im Beispiel $(n-3)/((n-3)+1)$.  Damit ergibt
sich die Invariante als $a + n/(n+1)$, wobei $a$ der Wert von
\texttt{acc} ist.  Um die Annahme zu beweisen, daß dies die Invariante
ist, muß im wesentlichen folgende Gleichung bewiesen werden:
\[a+ \frac{n}{n+1} = a + \frac{n-1}{n} + \frac{1}{n\cdot (n+1)}\]
Dies ist eine lohnende Fingerübung.


\section*{Aufgaben}

\begin{aufgabe}
Welche interessanten Eigenschaften hat die Division?  Schreiben Sie
diese als Eigenschaften von \texttt{/} auf.
\end{aufgabe}

\begin{aufgabe}
  Schreiben Sie eine möglichst vollständige Liste interessanter
  Eigenschaften sowohl der Ihnen bekannten arithmetischen Operationen
  als auch der logischen Operationen auf.  Beziehen Sie dazu auch die
  Vergleichsoperationen $<$, $\leq$ etc.\ ein.  Finden Sie außerdem
  für jede Operation eine interessante Eigenschaft, die \emph{nicht}
  gilt und überprüfen, ob \drscheme{} jeweils ein Gegenbeispiel
  findet.
\end{aufgabe}

\begin{aufgabe}
  \label{aufgabe:boolean-distrib}
  Für \texttt{and} und \texttt{or} gelten zwei Distributivgesetze
  analog dem Distributivgesetz für \texttt{*} und \texttt{+}:
  Formulieren Sie diese als Eigenschaften und lassen Sie \drscheme{}
  sie überprüfen.

  Abstrahieren Sie dann über die nun insgesamt drei Distributivgesetze
  analog zu \texttt{commutativity} und \texttt{associativity} und
  formulieren Sie die drei Distributivgesetze mit Hilfe der
  Abstraktion neu.  Schreiben Sie eine möglichst aussagekräftige
  Signatur für Ihre Abstraktion!
\end{aufgabe}

\begin{aufgabe}  Schreiben Sie Abstraktionen analog zu \texttt{commutativity} und
  \texttt{associativity} für folgende Eigenschaften:
  \begin{enumerate}
  \item DeMorgan
  \item Reflexivität
  \item Symmetrie
  \item Antisymmetrie
  \item Transitivität
  \item linksneutrales Element
  \item rechtsneutrales Element
  \item inverses Element
  \end{enumerate}
\end{aufgabe}


\begin{aufgabe}
  Versuchen Sie, die Eigenschafts-Tests für \texttt{number-list=?}
  auszutricksen, also eine fehlerhafte Version von
  \texttt{number-list=?} zu schreiben, die alle drei
  \texttt{check-property}-Tests besteht.
  Die \texttt{check-expect}-Tests sind für diese Aufgabe nicht relevant.
\end{aufgabe}

\begin{aufgabe}
  Formulieren Sie Eigenschaften von \texttt{filter} und \texttt{map}
  im Zusammenhang mit \texttt{concatenate} und testen Sie diese.
\end{aufgabe}

\begin{aufgabe}
  Finden Sie eine präzisere Formulierung der Kommutativität von
  \texttt{list-sum} als die in
  Abschnitt~\ref{sec:list-sum-commutative}, also eine, an der sich die
  Eigenschaft, daß die "<Reihenfolge der Elemente der Liste keine
  Rolle spielt"> klarer zu sehen ist.

  Schreiben Sie dazu eine Funktion, welche die Reihenfolge der
  Elemente einer Liste abhängig von einer natürlichen Zahl $n$
  verändert, z.B.\ indem die $n$te Permutation der Elemente ausgewählt
  wird.
\end{aufgabe}

\begin{aufgabe}
  Schreiben Sie einen \texttt{check-property}-Test für folgende Eigenschaft:
  \begin{center}
    \texttt{(uncurry (curry $p_2$))} $\equiv$ $p_2$
  \end{center}
\end{aufgabe}

\begin{aufgabe}
  Formulieren Sie sinnvolle Eigenschaften von \texttt{compose} und
  \texttt{repeat} von Seite~\ref{page:repeat} und überprüfen Sie
  diese mit \texttt{check-property}!
\end{aufgabe}

\begin{aufgabe}
  Beweise, daß für Funktionen $p_1$ mit einem Parameter, die
  einparametrige Funktionen zurückgeben, und Funktionen $p_2$ mit zwei
  Parametern gilt:
  %
  \begin{center}
    \texttt{(curry (uncurry $p_1$))} $\equiv$ $p_1$\\
    \texttt{(uncurry (curry $p_2$))} $\equiv$ $p_2$
  \end{center}
\end{aufgabe}

\begin{aufgabe}
  Beweisen Sie die entsprechend dem Beweis der Kommutativität von
  \texttt{+} in Abschnitt~\ref{sec:scheme-arithmetik-beweis} die
  Assoziativität von \texttt{+} sowie die Distributivität von
  \texttt{+} und \texttt{*} aus
  Abschnitt~\ref{sec:eigenschaften-binaere-operationen}.
\end{aufgabe}

\begin{aufgabe}
Beweisen Sie, daß die
folgende Funktion natürliche Zahlen quadriert:
%
\begin{alltt}
; Quadrat einer Zahl berechnen
(: square (natural -> natural))
(define square
  (lambda (n)
    (if (= n 0)
        0
        (+ (square (- n 1))
           (- (+ n n) 1)))))
\end{alltt}
%
Formulieren Sie dazu auch eine Eigenschaft und überprüfen Sie diese
mit \texttt{check-property}.
%
\end{aufgabe}

\begin{aufgabe}
Beweisen Sie, daß auch die folgende Funktion \texttt{square}
natürliche Zahlen quadriert.  Geben Sie die Invariante von
\texttt{square-helper} an!
%
\begin{alltt}
; Quadrat einer Zahl berechnen
(: square (natural -> natural))
(define square
  (lambda (n)
    (square-helper n 0)))

(define square-helper
  (lambda (n acc)
    (if (= n 0)
        acc
        (square-helper (- n 1)
                       (+ acc
                          (- (+ n n) 1))))))
\end{alltt}
Formulieren Sie dazu auch eine Eigenschaft und überprüfen Sie diese
mit \texttt{check-property}.
\end{aufgabe}

\begin{aufgabe}
 Beweise mit Hilfe des Substitutionsmodells, daß
  die \texttt{concatenate}-Funktion aus Abschnitt~\ref{sec:lists}
  assoziativ ist, daß also für Listen $l_1$, $l_2$ und $l_3$ gilt:
\begin{alltt}
(concatenate \(l\sb{1}\) (concatenate \(l\sb{2}\) \(l\sb{3}\))) \(=\) (concatenate (concatenate \(l\sb{1}\) \(l\sb{2}\)) \(l\sb{3}\))
\end{alltt}
\end{aufgabe}


%%% Local Variables: 
%%% mode: latex
%%% TeX-master: "i1"
%%% End: 

