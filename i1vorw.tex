% Diese Datei ist Teil des Buchs "Schreibe Dein Programm!"
% Das Buch ist lizensiert unter der Creative-Commons-Lizenz
% "Namensnennung 4.0 International (CC BY 4.0)"
% http://creativecommons.org/licenses/by/4.0/deed.de

\chapter*{Vorwort}
\thispagestyle{empty}

\markboth{Vorwort}{Vorwort}

Das hier präsentierte Material entstammt einer Reihe von einführenden
Vorlesungen zur Informatik für Haupt- und Nebenfachstudenten und für
Geisteswissenschaftler sowie Erfahrungen in zahlreichen Schulungen,
Tutorials und anderen Veranstaltungen.  Inhalte und Präsentation
wurden dabei unter Beobachtung der Studenten und ihres Lern\-er\-folgs
immer wieder verbessert.  Der Stoff dieses Buchs entspricht einer
einsemestrigen Vorlesung \textit{Informatik~I} mit vier
Vorlesungsstunden und zwei Übungsstunden.

\textit{Schreibe Dein Programm!} ist aus dem Vorgängerbuch \textit{Die
  Macht der Abstraktion} entstanden, das seinerseits aus dem
Vorgängerbuch \textit{Vom Problem zum Programm} entstanden ist.  Wir
bemerkten nach der Veröffentlichung von \textit{Die Macht der
  Abstraktion}, daß wir das Buch einerseits einem breiten Publikum
einfach zugänglich machen wollten, andererseits kontinuierlich
Verbesserungen einarbeiten wollten.  Beides war mit unserem damaligen
Verleger leider nicht zu machen.  Entsprechend haben wir uns
entschieden, unsere Arbeit unter neuem Titel fortzuführen und frei
zugänglich zu machen.  Es wird hoffentlich die letzte Titeländerung
bleiben.

Wir hatten zwar bereits viel Material aus \textit{Die Macht der
  Abstraktion} bis zur Unkenntlichkeit revidiert.  Die
"<TBD">-Abschnitte in \textit{Schreibe Dein Programm!} kennzeichnen
Stellen, deren Notwendigkeit bereits in \textit{Die Macht der
  Abstraktion} etabliert werde, die aber noch geschrieben werden
müssen.

\section*{Danksagungen}

Wir, die Autoren, haben bei der Erstellung dieses Buchs immens von der
Hilfe anderer profitiert.  Robert Giegerich, Ulrich Güntzer, Peter
Thiemann, Martin Plümicke, Christoph Schmitz und Volker Klaeren
machten viele Verbesserungsvorschläge zum Vorgängerbuch \textit{Vom
  Problem zum Programm}.

Martin Gasbichler hielt einen Teil der Vorlesungen der letzten
\textit{Informatik I}, half bei der Entwicklung der DMdA-Erweiterungen
und ist für eine große Anzahl von Verbesserungen verantwortlich, die
sich in diesem Buch finden.  Eric Knauel, Marcus Crestani, Sabine
Sperber, Jan-Georg Smaus und Mayte Fleischer brachten viele Verbesserungsvorschläge
ein.  Andreas Schilling, Torsten Grust und Michael Hanus hielten
Vorlesungen auf Basis dieses Buches und brachten ebenfalls viele
Verbesserungen ein.
Besonderer Dank gebührt den Tutoren und Studenten unserer Vorlesung
\textit{Informatik I}, die eine
Fülle wertvoller Kritik und exzellenter Verbesserungsvorschläge
lieferten.

Wir sind außerdem dankbar für die Arbeit unserer Kollegen, die
Pionierarbeit in der Entwicklung von Konzepten für die
Programmierausbildung geliefert haben.  Eine besondere
Stellung nehmen Matthias Felleisen, Robert Bruce Findler, Matthew
Flatt und Shriram Krishnamurthi und ihr Buch~\textit{How to Design
  Programs} \cite{FelleisenFindlerFlattKrishnamurthi2001} ein, das
entscheidende didaktische Impulse für dieses Buch gegegen hat.
Felleisens Arbeit im Rahmen des PLT-Projekts hat uns stark beeinflußt;
das PLT-\drscheme{}-System ist eine entscheidende Grundlage für die
Arbeit mit diesem Buch.

\begin{flushright}
  Herbert Klaeren

  Michael Sperber

  Tübingen, April 2014
\end{flushright}


\newpage

\thispagestyle{empty}

%%% Local Variables: 
%%% mode: latex
%%% TeX-master: "i1"
%%% End: 

