% Diese Datei ist Teil des Buchs "Schreibe Dein Programm!"
% Das Buch ist lizensiert unter der Creative-Commons-Lizenz
% "Namensnennung 4.0 International (CC BY 4.0)"
% http://creativecommons.org/licenses/by/4.0/deed.de

\chapter*{Vorwort}
\thispagestyle{empty}

\markboth{Vorwort}{Vorwort}

TBD

\textit{Schreibe Dein Programm!} ist aus dem Vorg�ngerbuch \textit{Die
  Macht der Abstraktion} entstanden, das seinerseits aus dem
Vorg�ngerbuch \textit{Vom Problem zum Programm} entstanden ist.  Wir
bemerkten nach der Ver�ffentlichung von \textit{Die Macht der
  Abstraktion}, da� wir das Buch einerseits einem breiten Publikum
einfach zug�nglich machen wollten, andererseits kontinuierlich
Verbesserungen einarbeiten wollten.  Beides war mit unserem damaligen
Verleger leider nicht zu machen.  Entsprechend haben wir uns
entschieden, unsere Arbeit unter neuem Titel fortzuf�hren und frei
zug�nglich zu machen.  Es wird hoffentlich die letzte Titel�nderung
bleiben.

Wir hatten zwar bereits viel Material aus \textit{Die Macht der
  Abstraktion} bis zur Unkenntlichkeit revidiert.  Die
"`TBD"'-Abschnitte in \textit{Schreibe Dein Programm!} kennzeichnen
Stellen, deren Notwendigkeit bereits in \textit{Die Macht der
  Abstraktion} etabliert werde, die aber noch geschrieben werden
m�ssen.

\section*{Lehre mit diesem Buch}

Das hier pr�sentierte Material entstammt einer Reihe von einf�hrenden
Vorlesungen zur Informatik f�r Haupt- und Nebenfachstudenten und f�r
Geisteswissenschaftler sowie Erfahrungen in zahlreichen Fortbildungen.
Inhalte und Pr�sentation wurden dabei unter
Beobachtung der Studenten und ihres Lern\-er\-folgs immer wieder
verbessert.  Der Stoff dieses Buchs entspricht einer einsemestrigen
Vorlesung \textit{Informatik~I} mit vier Vorlesungsstunden und zwei
�bungsstunden.

Anhang~\ref{cha:math} erl�utert die im Buch verwendeten mathematischen
Notationen und Termini.

\section*{Programmiersprache}

Leider sind die heute in der Industrie popul�ren Programmiersprachen
f�r die Lehre nicht geeignet: ihre
Abstraktionsmittel sind begrenzt, und das Erlernen ihrer komplizierten Syntax 
kostet wertvolle Zeit und Kraft.

Aus diesem Grund verwendet der vorliegende Text eine Serie von
speziell f�r die Lehre entwickelten Programmiersprachen,
die auf \textit{Racket\index{Scheme}} und
\textit{Scheme\index{Scheme}} basieren.  Diese sind �ber die
\drscheme"=Entwicklungsumgebung Anf�ngern besonders gut zug�nglich.

\section*{Software und Material zum Buch}

Die Programmierbeispiele dieses Buchs bauen auf der
Programmierumgebung \drscheme{}\index{\drscheme{}} auf, die speziell f�r die
Programmierausbildung entwickelt wurde.  Insbesondere unterst�tzt
\drscheme{} die Verwendung sogenannter \textit{Sprachebenen}, Varianten
der Sprache, die speziell f�r die Ausbildung zugeschnitten wurden.
Dieses Buch benutzt spezielle Sprachebenen, die Teil der sogenannten
\textit{DMdA-Erweiterungen}\index{DMdA-Erweiterungen} von \drscheme{} sind.

\drscheme{} ist kostenlos im Internet auf der Seite
\url{http://www.racket-lang.org/} erh�ltlich und l�uft auf Windows-, Mac-
und Unix-/Linux-Rechnern.  Die DMdA-Erweiterungen sind von der
Homepage zu \textit{Schreibe Dein Programm!	} erh�ltlich:
%
\begin{verbatim}
http://www.deinprogramm.de/
\end{verbatim}
%
Dort steht auch eine Installationsanleitung.

Auf der Homepage befindet sich weiteres Material zum Buch,
insbesondere Quelltext f�r alle Programmbeispiele zum Herunterladen.
 
\section*{Danksagungen}

Wir, die Autoren, haben bei der Erstellung dieses Buchs immens von der
Hilfe anderer profitiert.  Robert Giegerich, Ulrich G�ntzer, Peter
Thiemann, Martin Pl�micke, Christoph Schmitz und Volker Klaeren
machten viele Verbesserungsvorschl�ge zum Vorg�ngerbuch \textit{Vom
  Problem zum Programm}.

Martin Gasbichler hielt einen Teil der Vorlesungen der letzten
\textit{Informatik I}, half bei der Entwicklung der DMdA-Erweiterungen
und ist f�r eine gro�e Anzahl von Verbesserungen verantwortlich, die
sich in diesem Buch finden.  Eric Knauel, Marcus Crestani, Sabine
Sperber, Jan-Georg Smaus und Mayte Fleischer brachten viele Verbesserungsvorschl�ge
ein.  Andreas Schilling, Torsten Grust und Michael Hanus hielten
Vorlesungen auf Basis dieses Buches und brachten ebenfalls viele
Verbesserungen ein.
Besonderer Dank geb�hrt den Tutoren und Studenten unserer Vorlesung
\textit{Informatik I}, die eine
F�lle wertvoller Kritik und exzellenter Verbesserungsvorschl�ge
lieferten.

Wir sind au�erdem dankbar f�r die Arbeit unserer Kollegen, die
Pionierarbeit in der Entwicklung von Konzepten f�r die
Programmierausbildung geliefert haben.  Eine besondere
Stellung nehmen Matthias Felleisen, Robert Bruce Findler, Matthew
Flatt und Shriram Krishnamurthi und ihr Buch~\textit{How to Design
  Programs} \cite{FelleisenFindlerFlattKrishnamurthi2001} ein, das
entscheidende didaktische Impulse f�r dieses Buch gegegen hat.
Felleisens Arbeit im Rahmen des PLT-Projekts hat uns stark beeinflu�t;
das PLT-\drscheme{}-System ist eine entscheidende Grundlage f�r die
Arbeit mit diesem Buch.

\begin{flushright}
  Herbert Klaeren

  Michael Sperber

  T�bingen, April 2014
\end{flushright}


\newpage

\thispagestyle{empty}

%%% Local Variables: 
%%% mode: latex
%%% TeX-master: "i1"
%%% End: 

