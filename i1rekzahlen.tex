% Diese Datei ist Teil des Buchs "Schreibe Dein Programm!"
% Das Buch ist lizensiert unter der Creative-Commons-Lizenz
% "Namensnennung - Weitergabe unter gleichen Bedingungen 4.0 International (CC BY-SA 4.0)"
% https://creativecommons.org/licenses/by-sa/4.0/deed.de

\chapter{Funktionen über natürlichen Zahlen}
\label{cha:recursion-numbers}

%\section{Funktionen über natürlichen Zahlen} HK: Hier die einzige numerierte
%section, also zunächst überflüssig
\label{sec:factorial}

Aus der induktiven Definition der natürlichen Zahlen ergibt sich direkt eine Schablone für die
Konstruktion von Funktionen auf den natürlichen Zahlen.  Angenommen,
es sei eine Funktion gefragt, die für eine Zahl $n$ das Produkt aller
Zahlen von $1$ bis $n$ berechnet. 
Dieses Produkt heißt 
\textit{Fakultät}\index{Fakultät}, auf Englisch
\textit{factorial\index{factorial@\texttt{factorial}}}. 
  Beschreibung und Signatur sind wie folgt:
%
\begin{alltt}
; das Produkt aller Zahlen von 1 bis n berechnen
(: factorial (natural -> natural))
\end{alltt}
\label{page:factorial}
%
Das Gerüst ergibt sich aus der Signatur:
%
\begin{alltt}
(define factorial
  (lambda (n)
    ...))
\end{alltt}
%
Wie oben festgestellt, handelt es sich bei den natürlichen Zahlen um
eine Fallunterscheidung mit zwei Fällen.  In der Schablone muss also
eine Verzweigung mit zwei Fällen stehen:
%
\begin{alltt}
(define factorial
  (lambda (n)
    (cond
      (... ...)
      (... ...))))
\end{alltt}
%
Der erste Fall ist die $0$, der andere Fall deckt alle anderen Zahlen
ab:
%
\begin{alltt}
(define factorial
  (lambda (n)
    (cond
      ((= n 0) ...)
      (else ...))))
\end{alltt}
%
Diese Fallunterscheidung lässt sich leichter mit \texttt{if}
schreiben:\footnote{Es wäre genauso richtig, beide Zweige
  mit einem Test zu versehen, wie bei den Funktionen über Listen:
%
\begin{alltt}
(define factorial\\
\hspace*{0em}~~(lambda (n)\\
\hspace*{0em}~~~~(cond\\
\hspace*{0em}~~~~~~((= n 0) ...)\\
\hspace*{0em}~~~~~~((> n 0) ...))))
\end{alltt}
%
Umgekehrt ließe sich auch die Fallunterscheidung bei 
Listen mit \texttt{if} schreiben.  Welche
Variante am besten ist, ist Geschmackssache.}
%
\begin{alltt}
(define factorial
  (lambda (n)
    (if (= n 0)
        ...
        ...)))
\end{alltt}
%
Der Fall $0$ ist hier gar nicht so einfach zu beantworten.  Was sind
schließlich "<die Zahlen von $1$ bis $0$">?  Dafür ist der andere Zweig
einfacher zu ergänzen: hier ist nämlich die Konstruktionsanleitung für
zusammengesetzte Daten anwendbar.  Als Selektor wird die
Vorgängeroperation angewendet, es steht also \texttt{(- n 1)} in der
Alternative.
Genau wie bei den endlichen Folgen liegt nahe, dass auf \texttt{(- n 1)} ein
rekursiver Aufruf erfolgt:
%
\begin{alltt}
(define factorial
  (lambda (n)
    (if (= n 0)
        ...
        ... (factorial (- n 1)) ...)))
\end{alltt}
%
Der rekursive Aufruf \texttt{(factorial (- n 1))} soll nach
Signatur und Beschreibung der Funktion das Produkt der Zahlen von 1 bis
$\texttt{n}-1$ berechnen.  Gefragt ist aber das Produkt der Zahlen von
1 bis \texttt{n}.  Es fehlt noch \texttt{n} selbst, das mit dem
Ergebnis multipliziert werden muss:
%
\begin{alltt}
(define factorial
  (lambda (n)
    (if (= n 0)
        ...
        (* n (factorial (- n 1))))))
\end{alltt}
%
Es fehlt schließlich noch der erste Zweig der \texttt{if}-Form.  In
Fällen wie diesem, wo die Antwort nicht offensichtlich ist,
lohnt es sich, ein oder zwei einfache Beispiele durchzurechnen, die meist die
Antwort zwingend festlegen.  Das einfachste Beispiel ist wohl 1~--
das Produkt der Zahlen von 1 bis 1 sollte 1 sein.  Auswertung mit dem
Substitutionsmodell ergibt:

\begin{alltt}
(factorial 1)
\(\Longrightarrow\) (if (= 1 0) ... (* 1 (factorial (- 1 1))))
\(\Longrightarrow\) (if #f ... (* 1 (factorial (- 1 1))))
\(\Longrightarrow\) (* 1 (factorial (- 1 1)))
\(\Longrightarrow\) (* 1 (factorial 0))
\(\Longrightarrow\) (* 1 (if (= 0 0) ... (* 0 (factorial (- 0 1)))))
\(\Longrightarrow\) (* 1 (if #t ... (* 0 (factorial (- 0 1)))))
\(\Longrightarrow\) (* 1 ...)
\end{alltt}
%
Damit ist klar, dass die unbekannte Zahl, die an Stelle der 
\texttt{...} stehen muss, multipliziert mit 1 wiederum 1 ergeben muss.
Die einzige Zahl, die diese Bedingung erfüllt, ist 1 selbst.  Die
vollständige Definition von \texttt{factorial} ist also:
%
\begin{alltt}
(define factorial
  (lambda (n)
    (if (= n 0)
        1
        (* n (factorial (- n 1))))))
\end{alltt}
%
An einem größeren Beispiel lässt sich anhand des Substitutionsmodells
oder im Stepper besonders gut sehen, wie die Rekursion verläuft:
%
\begin{alltt}
(factorial 4)
\(\Longrightarrow\) (if (= 4 0) 1 (* 4 (factorial (- 4 1))))
\(\Longrightarrow\) (if #f 1 (* 4 (factorial (- 4 1))))
\(\Longrightarrow\) (* 4 (factorial (- 4 1)))
\(\Longrightarrow\) (* 4 (factorial 3))
\(\Longrightarrow\) (* 4 (if (= 3 0) 1 (* 3 (factorial (- 3 1)))))
\(\Longrightarrow\) (* 4 (if #f 1 (* 3 (factorial (- 3 1)))))
\(\Longrightarrow\) (* 4 (* 3 (factorial (- 3 1))))
\(\Longrightarrow\) (* 4 (* 3 (factorial 2)))
\(\ldots\)
\(\Longrightarrow\) (* 4 (* 3 (* 2 (factorial 1))))
\(\Longrightarrow\) (* 4 (* 3 (* 2 (if (= 1 0) 1 (* 1 (factorial (- 1 1)))))))
\(\Longrightarrow\) (* 4 (* 3 (* 2 (if #f ... (* 1 (factorial (- 1 1)))))))
\(\Longrightarrow\) (* 4 (* 3 (* 2 (* 1 (factorial (- 1 1))))))
\(\Longrightarrow\) (* 4 (* 3 (* 2 (* 1 (factorial 0)))))
\(\Longrightarrow\) (* 4 (* 3 (* 2 (* 1 (if (= 0 0) 1 (* 0 (factorial (- 0 1))))))))
\(\Longrightarrow\) (* 4 (* 3 (* 2 (* 1 (if #t 1 (* 0 (factorial (- 0 1)))))))))
\(\Longrightarrow\) (* 4 (* 3 (* 2 (* 1 1))))
\(\Longrightarrow\) (* 4 (* 3 (* 2 1)))
\(\Longrightarrow\) (* 4 (* 3 2))
\(\Longrightarrow\) (* 4 6)
\(\Longrightarrow\) 24
\end{alltt}
%
Die typischen Beobachtungen an diesem Beispiel sind:
%
\begin{itemize}
\item \texttt{Factorial} ruft sich nie mit derselben Zahl $n$ auf, mit
  der es selbst aufgerufen wurde, sondern immer mit $n-1$.
\item Die natürlichen Zahlen sind so strukturiert, dass die Kette $n,
  n-1, n-2 \ldots$ irgendwann bei $0$ abbrechen muss.
\item \texttt{Factorial} ruft sich bei $n=0$ \emph{nicht}
  selbst auf.
\end{itemize}
%
Aus diesen Gründen kommt der von
\texttt{(factorial $n$)} erzeugte Berechnungsprozess immer zum Schluss.  

Aus der Definition von \texttt{factorial} ergibt sich eine
Konstruktionsanleitung für Funktionen, die natürliche Zahlen
verarbeiten.  Die Schablone für solche Funktionen sieht folgendermaßen aus:
%
\begin{alltt}
(: proc (natural -> ...))
(define \(p\)
  (lambda (\(n\))
    (if (= \(n\) 0)
        ...
        ... (\(p\) (- \(n\) 1)) ...)))
\end{alltt}
\label{sec:ka-recursion-numbers}
%
Konstruktionsanleitung~\ref{ka:N} in
Anhang~\ref{app:konstruktionsanleitungen} fasst dies noch einmal
zusammen.

TDB: Funktionen, die Listen erzeugen etc.

\section*{Aufgaben}

\begin{aufgabe}\label{aufg:power}
  Schreiben Sie eine Funktion \texttt{power}, die für eine Basis $b$ und
  einen Exponenten $e\in\mathbb{N}$ die Potenz $b^e$ ausrechnet.  Also:
\begin{alltt}
(power 5 3)
\evalsto{} 125
\end{alltt}
  \end{aufgabe}

\begin{aufgabe}\label{ex:evensodds}
  \begin{itemize}
  \item
    Die eingebaute Funktion \texttt{even?}\index{even?@\texttt{even?}}
    akzeptiert eine ganze Zahl und liefert \verb|#t|, falls diese
    gerade ist und \verb|#f| sonst.
    Schreiben Sie mit Hilfe von \texttt{even?}
    eine Funktion namens \texttt{evens}, welche für zwei
    Zahlen $a$ und $b$ eine Liste der geraden Zahlen zwischen $a$ und
    $b$ zurückgibt:
\begin{alltt}
(evens 1 10)
\evalsto{} #<record:pair 2
     #<record:pair 4
       #<record:pair 6
         #<record:pair 8
           #<record:pair 10 #<empty-list>>>>>>
\end{alltt}
  \item
    Die eingebaute Funktion \texttt{odd?}\index{odd?@\texttt{odd?}}
    akzeptiert eine ganze Zahl und liefert \verb|#t|, falls diese
    ungerade ist und \verb|#f| sonst.
    Schreiben Sie mit Hilfe von \texttt{odd?} eine Funktion \texttt{odds}, welche für zwei
    Zahlen $a$ und $b$ eine Liste der ungeraden Zahlen zwischen $a$ und $b$
    zurückgibt:
\begin{alltt}
(odds 1 10)
\evalsto{} #<record:pair 1
     #<record:pair 3
       #<record:pair 5
         #<record:pair 7
           #<record:pair 9 #<empty-list>>>>>>
\end{alltt}
  \end{itemize}
\end{aufgabe}

\begin{aufgabe}
  Programmieren Sie folgende Funktionen auf Listen:
  \begin{enumerate}
  \item Programmieren Sie eine Funktion \texttt{take}, die als
    Argumente eine Liste \texttt{l} und eine Zahl \texttt{n}
    akzeptiert, und die ersten \texttt{n} Elemente der Liste
    \texttt{l} zurückgibt. Beispiel:
    \begin{alltt}
      (take (list 1 2 3 4 5 6 7 8) 5)
      \evalsto{} #<list 1 2 3 4 5>
    \end{alltt}
    %
    Gehen Sie davon aus, dass \texttt{l} mindestens \texttt{n}
    Elemente lang ist.

  \item Programmieren Sie eine Funktion \texttt{drop}, die als
    Argumente eine Liste \texttt{l} und eine Zahl \texttt{n}
    akzeptiert, und die Liste \texttt{l} ohne die ersten \texttt{n}
    Elemente zurückgibt.  Beispiel:
    \begin{alltt}
      (drop (list 1 2 3 4 5 6 7 8) 3)
      \evalsto{} #<list 4 5 6 7 8>
    \end{alltt}
    %
    Gehen Sie davon aus, dass \texttt{l} mindestens \texttt{n}
    Elemente lang ist.
    \end{enumerate}
\end{aufgabe}

\begin{aufgabe}
  Das folgende Zahlenmuster heißt das \textit{Pascalsche
    Dreieck}:\index{Pascalsches Dreieck}
  \begin{center}
    1\\
    1~~1\\
    1~~2~~1\\
    1~~3~~3~~1\\
    1~~4~~6~~4~~1\\
    \ldots
  \end{center}
  %
  Die Zahlen an den Kanten des Dreiecks sind allesamt 1.  Eine Zahl im
  Innern des Dreiecks ist die Summe der beiden Zahlen darüber.
  Schreibe eine Funktion \texttt{pascal}, die als Argumente die Nummer
  einer Zeile und die Nummer einer "<Spalte"> innerhalb des Dreiecks
  akzeptiert, und die Zahl im Pascalschen Dreieck berechnet.  (Sowohl
  Zeilen- als auch Spaltennummer fangen bei~1 an.)
%
\begin{alltt}
(pascal 5 3)
\evalsto{} 6
\end{alltt}
\end{aufgabe}

\begin{aufgabe}\label{ex:coke-finances}
  Schreiben Sie eine Funktion \texttt{drink-machine}, die das
  Finanzmanagement eines Getränkeautomaten durchführt.
  
  \texttt{Drink-machine} soll drei Argumente akzeptieren: den Preis eines
  Getränks (in Euro-Cents), eine Liste der Centbeträge der
  Wechselgeldmünzen, die noch im Automaten sind, und den Geldbetrag,
  den der Kunde eingeworfen hat.  (Also gibt es ein Listenelement pro
  Münze. Falls z.B.\ mehrere Groschen im Automaten sind, finden sich
  in der Liste mehrmals die Zahl 10.)  Herauskommen soll eine Liste
  der Centbeträge der Münzen, welche die Maschine herausgibt oder
  \verb|#f|, falls die Maschine nicht herausgeben kann.
  %
\begin{alltt}
(drink-machine 140 (list 50 100 500 10 10) 200)
\evalsto{} #<list 50 10>
(drink-machine 140 (list 50 20 20 20) 200)
\evalsto{} #<list 20 20 20>
\end{alltt}
\end{aufgabe}

%%% Local Variables: 
%%% mode: latex
%%% TeX-master: "i1"
%%% End: 
