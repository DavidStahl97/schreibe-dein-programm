% Diese Datei ist Teil des Buchs "Schreibe Dein Programm!"
% Das Buch ist lizensiert unter der Creative-Commons-Lizenz
% "Namensnennung - Weitergabe unter gleichen Bedingungen 4.0 International (CC BY-SA 4.0)"
% https://creativecommons.org/licenses/by-sa/4.0/deed.de

\chapter{Natürliche Zahlen}
\label{cha:recursion-numbers}

In diesem Kapitel geht es um Funktionen, die etwas zählen und sich
dementsprechend mit Zahlen beschäftigen.  Wir haben schon eine Reihe
von Programmen gesehen, die mit Zahlen rechnen, aber Zählen ist etwas
besonderes und verdient deshalb ein eigenes Kapitel mit eigener
Konstruktionsanleitung.

\section{Zahlen, die zählen}

Du kennst bestimmt aus dem Mathematikunterricht die
\textit{Potenz}\index{Potenz} einer Zahl (der
\textit{Basis}\index{Basis}) $b$ zu einem
\textit{Exponenten}\index{Exponent} $e$:
%
\begin{displaymath}
  b^e
\end{displaymath}
%
Uns interessieren hier Exponenten, die ganze Zahlen ab 0 sind.  Für
solche Exponenten ist die Potenz folgendermaßen definiert:
%
\begin{displaymath}
  b^e \deq \underbrace{b \times \ldots \times b}_{e\textrm{-mal}}
\end{displaymath}
%
Der Exponent $e$ muss also eine natürliche Zahl\index{natürliche Zahl}
sein.\footnote{Spitzfindige Leser können anmerken, dass in manchen
  Lehrbüchern die natürlichen Zahlen mit 1 beginnen, aber wir halten
  uns an DIN~5473, und da ist die 0 mit dabei.}
  
  Umgangssprachlich sind natürliche Zahlen gerade die Zahlen, die
Gegenstände zählen.  In unserer Intuition sind die natürlichen Zahlen
einfach auf einem Zahlenstrahl angeordnet und bilden eine Reihe.  Für
das Programmieren ist das aber wenig hilfreich.  Viel mehr können wir
mit der folgenden Datendefinition anfangen:
%
\begin{lstlisting}
; Eine natürliche Zahl ist eine der folgenden:
; - 0
; - der Nachfolger einer natürlichen Zahl
\end{lstlisting}
%
In dieser Definition findet man alte Bekannte wieder: Es handelt sich
um eine Fallunterscheidung, und der zweite Fall enthält einen
Selbstbezug.

Dir mag der Begriff "<Nachfolger"> etwas befremdlich erscheinen: Der
Nachfolger einer Zahl ist einfach nur die Zahl "<plus 1">.  Weil der
Nachfolger im Zusammenhang mit natürlichen Zahlen eine besondere
Bedeutung hat, spendieren wir ihm zunächst eine eigene Hilfsfunktion:
%
\begin{lstlisting}
; Nachfolger einer Zahl
(: successor (natural -> natural))

(define successor
  (lambda (n)
    (+ n 1)))
\end{lstlisting}
%
Wo ein Nachfolger ist, muss es auch einen Vorgänger geben, und der
zieht Eins ab:
%
\begin{lstlisting}
; Vorgänger einer Zahl
(: predecessor (natural -> natural))

(define predecessor
  (lambda (n)
    (- n 1)))
\end{lstlisting}
%
Allerdings haben wir bei dieser Definition etwas geschummelt, weil der
Vorgänger nur für die natürlichen Zahlen funktioniert, welche die
Nachfolger anderer natürlicher Zahlen sind~-- also für sogenannte
\textit{positive} natürliche Zahlen\index{positive natürliche Zahlen}.
Die Signatur kann das nicht zum Ausdruck bringen, da es keine
eingebaute Signatur dafür gibt.  Aber wir können eine Verzweigung
entsprechend der Datendefinition einbauen, um die Funktion vor
Missbrauch zu schützen:
%
\begin{lstlisting}
(define predecessor
  (lambda (n)
    (cond
      ((zero? n)
       (violation "0 does not have a predecessor"))
      ((positive? n)
       (- n 1)))))
\end{lstlisting}
%
Du siehst in der Funktionsdefinition die beiden eingebauten Prädikate
\lstinline{zero?}\index{zero?@\texttt{zero?}} und
\lstinline{positive?}\index{positive?@\texttt{positive?}}, welche die
beiden Fälle der Datendefinition der natürlichen Zahlen unterscheiden.

Zurück zur eigentlichen Aufgabe, der Potenz.  So könnten
Kurzbeschreibung, Signatur und zwei Tests aussehen:
%
\begin{lstlisting}
; Potenz einer Zahl berechnen
(: power (number natural -> number))

(check-expect (power 5 0) 1)
(check-expect (power 5 3) 125)
\end{lstlisting}
%
Als nächtes ist das Gerüst dran:
%
\begin{lstlisting}
(define power
  (lambda (base exponent)
    ...))
\end{lstlisting}
%
Für den Rumpf können wir eine Schablone aus der Datendefinition
ableiten für \lstinline{exponent} ableiten, das ist ja die natürliche
Zahl.  Die Datendefinition für natürliche Zahlen ist eine
Fallunterscheidung mit zwei Fällen, dementsprechend brauchen wir eine
Verzweigung mit zwei Zweigen:
%
\begin{lstlisting}
(define power
  (lambda (base exponent)
    (cond
      ((zero? exponent) ...)
      ((positive? exponent) ...))))
\end{lstlisting}
%
Im zweiten Fall der Datendefinition steckt außerdem eine
Selbstreferenz, wir schreiben also noch einen rekursiven Aufruf in die
Schablone:
%
\begin{lstlisting}
(define power
  (lambda (base exponent)
    (cond
      ((zero? exponent) ...)
      ((positive? exponent)
       ... (power base (predecessor exponent)) ...))))
\end{lstlisting}
%
Nun müssen wir noch die Lücken füllen:

Beim ersten Fall~--Exponent 0~-- muss als Potenz 1 herauskommen, das
neutrale Element bezüglich der Multiplikation: Das ist ja nichts
anderes, als die Zahlen der leeren Liste zu multiplizieren.  (Siehe
dazu auch Seite~\pageref{page:neutrales-element} in
Abschnitt~\ref{page:neutrales-element}.)

Für den anderen Fall ist das Ergebnis des rekursiven Aufruf die Potenz
von \lstinline{base} mit dem Vorgänger von \lstinline{exponent} als
Exponent.  Damit wir als Ergebnis \lstinline{base} mit sich selbst
\lstinline{exponent}-mal multipliziert bekommen, fehlt noch eine
Multiplikation mit \lstinline{base}:
%
\begin{lstlisting}
(define power
  (lambda (base exponent)
    (cond
      ((zero? exponent) 1)
      ((positive? base)
       (* base
          (power base (predecessor exponent)))))))
\end{lstlisting}
%
Fertig!

Der Begriff des Vorgängers ist hier eher im Weg: Er passt zwar gut zur
Verwendung des Worts "<Nachfolger"> in der
Datendefinition. \lstinline{(predecessor exponent)} ist aber hier das
gleiche ist wie \lstinline{(- exponent 1)}, und entsprechend werden
wir auch in der Schablone direkt \lstinline{(- ... 1)} statt
\lstinline{predecessor} verwenden.
\begin{aufgabe}
  Die Funktion \lstinline{predecessor} hat zwei Zweige: Warum ist der
  Aufruf in \lstinline{power} immer gleichbedeutend zu
  \lstinline{(- exponent 1)}~-- was ist mit dem anderen Zweig?
\end{aufgabe}


\begin{konstruktionsanleitung}{Natürliche Zahlen als Eingabe: Schablone}
  \label{ka:nats-eingabe-schablone}
  Eine Schablone für eine Funktion, die eine natürliche Zahl akzeptiert, sieht
folgendermaßen aus:
%
\begin{lstlisting}
(define |\(f\)|
  (lambda (|\ldots| |\(\mathit{n}\)| |\ldots|)
    (cond
      ((zero? |\(\mathit{n}\)|) |\ldots|)
      ((positive? |\(\mathit{n}\)|)
       |\ldots|
       (|\(f\)| (- |\(\mathit{list}\)| 1))
       |\ldots|
       ))))
\end{lstlisting}
  
\end{konstruktionsanleitung}

\section{Natürliche Zahlen und Listen}

Zwischen natürlichen Zahlen und Listen besteht eine enge Beziehung:
Die Datendefinition für natürliche Zahlen entspricht in ihrer Struktur
der von Listen.  Die Funktion~\lstinline{list-length} in
Abschnitt~\ref{page:list-length} auf Seite~\pageref{page:list-length}
macht aus einer Liste eine natürliche Zahl.  Das geht auch umgekehrt:
%
\begin{lstlisting}
; Liste aus Kopien eines Werts erzeugen
(: copies (natural %element -> (list-of %element)))

(check-expect (copies 4 23)
              (list 23 23 23 23))
(check-expect (copies 5 "Mike")
              (list "Mike" "Mike" "Mike" "Mike" "Mike"))
\end{lstlisting}
%
Die Signatur der Funktion enthält die Signaturvariable \lstinline{%element}, ist also
polymorph und macht klar, dass die Elemente der Liste nur zur Signatur
\lstinline{%element} gehören können.
%
\begin{lstlisting}
(define copies
  (lambda (count element)
    ...))
\end{lstlisting}
%
Die Konstruktionsanleitung schlägt folgende Schablone vor:
%
\begin{lstlisting}
(define copies
  (lambda (count element)
    (cond
      ((zero? count) ...)
      ((positive? count)
       ... (copies (- count 1) element) ...)))))
\end{lstlisting}
%
Für den ersten Fall brauchen wir eine Liste mit 0 Elemente.  Im
zweiten Fall liefert der rekursive Aufruf eine Liste mit einem Element
weniger, als wir brauchen. Das ergänzen wir zur fertigen Definition
so:
%
\begin{lstlisting}
(define copies
  (lambda (count element)
    (cond
      ((zero? count)
       empty)
      ((positive? count)
       (cons element
             (copies (- count 1) element))))))
\end{lstlisting}
%
Eine weitere nützliche Funktion akzeptiert die Nummer eines
Listenelementes~-- einen sogenannten \textit{Index}\index{Index}~--
und liefert das Element:
%
\begin{lstlisting}
; Nummeriertes Element aus einer Liste holen
(: nth ((list-of %element) natural -> %element))

(check-expect (nth (list 1 2 3 4 5) 0) 1)
(check-expect (nth (list 1 2 3 4 5) 2) 3)
\end{lstlisting}
%
Wir sollten uns noch überlegen, wenn es den Index in der Liste gar
nicht gibt.  Die Funktion kann kein sinnvolles Ergebnis liefern, das
zur Signatur passt~-- es muss ja ein Element der Liste herauskommen.
Die einzig sinnvolle Möglichkeit ist, einen Fehler anzuzeigen.
Daraus wird folgender Testfall:
%
\begin{lstlisting}
(check-error (nth (list 1 2 3 4 5) 5))
\end{lstlisting}
%
Das Gerüst der Funktion sieht folgendermaßen aus:
%
\begin{lstlisting}
(define nth
  (lambda (list index)
    ...))
\end{lstlisting}
%
Bei der Schablone haben wir die Wahl: Wir können entweder die
Schablone für Listen als Eingabe benutzen oder die für natürliche
Zahlen.  Oder beide.  Hier ist aber der Index federführend~-- wenn die
Funktion das zweite Element einer tausendelementigen Liste extrahiert,
ist die Länge irrelevant.  Wir halten uns also an die
Konstruktionsanleitung für natürliche Zahlen:
%
\begin{lstlisting}
(define nth
  (lambda (list index)
    (cond
      ((zero? index) ...)
      ((positive? index)
       ... (nth (rest list) (- index 1)) ...))))
\end{lstlisting}
%
Im ersten Fall ist der Index 0, also das erste Element der Liste
gefragt.  Im zweiten Fall liefert uns der rekursive Aufruf im Rest der
Liste das Element das Element an Stelle \lstinline{(- index 1)}.  Dies
ist schon das \lstinline{index}-te Element in \lstinline{list}.  Die
Funktion sieht so aus:
%
\begin{lstlisting}
(define nth
  (lambda (list index)
    (cond
      ((zero? index) (first list))
      ((positive? index)
       (nth (rest list) (- index 1))))))
\end{lstlisting}
%
Tatsächlich laufen alle Testfälle bereits durch~-- und das, obwohl wir
für den Fehlerfall gar nichts programmiert haben.
\begin{aufgabe}
  Werte mit dieser Definition folgenden Ausdruck aus:
\begin{lstlisting}
(nth (list 1 2 3 4 5) 5)
\end{lstlisting}
\end{aufgabe}
%
Die Fehlermeldung ist zwar technisch richtig, gibt aber keinen
Aufschluss auf die Ursache des Fehlers.  Das können wir korrigieren,
indem wir doch noch die Schablone für Listen als Eingabe ergänzen und
\lstinline{violation} benutzen.  (Siehe Abschnitt~\ref{sec:violation} auf
Seite~\pageref{sec:violation}.)
%
\begin{lstlisting}
(define nth
  (lambda (list index)
    (cond
      ((empty? list) (violation "nth: Die Liste ist nicht lang genug"))
      ((cons? list)
       (cond
         ((zero? index) (first list))
         ((positive? index)
          (nth (rest list) (- index 1))))))))
\end{lstlisting}
%
Fertig!

\section{Exkurs: Potenz optimieren}

FIXME

\section*{Aufgaben}

\begin{aufgabe}
  Die \textit{Fakultät}\index{Fakultät} einer Zahl $n$ ist definiert als:
  %
  \begin{displaymath}
    n! \deq 1 \times 2 \times \cdots \times n
  \end{displaymath}
  %
  Schreiben Sie eine Funktion, welche die Fakultät berechnet!
\end{aufgabe}

\begin{aufgabe}
  Programmieren Sie folgende Funktionen auf Listen:
  \begin{enumerate}
  \item Programmieren Sie eine Funktion \texttt{take}, die als
    Argumente eine Liste \texttt{l} und eine Zahl \texttt{n}
    akzeptiert, und die ersten \texttt{n} Elemente der Liste
    \texttt{l} zurückgibt. Beispiel:
    \begin{alltt}
      (take (list 1 2 3 4 5 6 7 8) 5)
      \evalsto{} #<list 1 2 3 4 5>
    \end{alltt}
    %
    Gehen Sie davon aus, dass \texttt{l} mindestens \texttt{n}
    Elemente lang ist.

  \item Programmieren Sie eine Funktion \texttt{drop}, die als
    Argumente eine Liste \texttt{l} und eine Zahl \texttt{n}
    akzeptiert, und die Liste \texttt{l} ohne die ersten \texttt{n}
    Elemente zurückgibt.  Beispiel:
    \begin{alltt}
      (drop (list 1 2 3 4 5 6 7 8) 3)
      \evalsto{} #<list 4 5 6 7 8>
    \end{alltt}
    %
    Gehen Sie davon aus, dass \texttt{l} mindestens \texttt{n}
    Elemente lang ist.
    \end{enumerate}
\end{aufgabe}


\begin{aufgabe}\label{ex:evensodds}
  \begin{itemize}
  \item
    Die eingebaute Funktion \texttt{even?}\index{even?@\texttt{even?}}
    akzeptiert eine ganze Zahl und liefert \verb|#t|, falls diese
    gerade ist und \verb|#f| sonst.
    Schreiben Sie mit Hilfe von \texttt{even?}
    eine Funktion namens \texttt{evens}, welche für zwei
    Zahlen $a$ und $b$ eine Liste der geraden Zahlen zwischen $a$ und
    $b$ zurückgibt:
\begin{alltt}
(evens 1 10)
\evalsto{} #<record:pair 2
     #<record:pair 4
       #<record:pair 6
         #<record:pair 8
           #<record:pair 10 #<empty-list>>>>>>
\end{alltt}
  \item
    Die eingebaute Funktion \texttt{odd?}\index{odd?@\texttt{odd?}}
    akzeptiert eine ganze Zahl und liefert \verb|#t|, falls diese
    ungerade ist und \verb|#f| sonst.
    Schreiben Sie mit Hilfe von \texttt{odd?} eine Funktion \texttt{odds}, welche für zwei
    Zahlen $a$ und $b$ eine Liste der ungeraden Zahlen zwischen $a$ und $b$
    zurückgibt:
\begin{alltt}
(odds 1 10)
\evalsto{} #<record:pair 1
     #<record:pair 3
       #<record:pair 5
         #<record:pair 7
           #<record:pair 9 #<empty-list>>>>>>
\end{alltt}
  \end{itemize}
\end{aufgabe}

%%% Local Variables: 
%%% mode: latex
%%% TeX-master: "i1"
%%% End: 
