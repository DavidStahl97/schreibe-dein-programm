
% Diese Datei ist Teil des Buchs "Schreibe Dein Programm!"
% Das Buch ist lizensiert unter der Creative-Commons-Lizenz
% "Namensnennung 4.0 International (CC BY 4.0)"
% http://creativecommons.org/licenses/by/4.0/deed.de

\chapter{Fallunterscheidungen und Verzweigungen}
\label{cha:conditionals}

Computerprogramme müssen bei manchen Daten, die sie
verarbeiten, zwischen verschiedenen Möglichkeiten differenzieren: Ist
die Wassertemperatur warm genug zum Baden?  Welche von fünf
Tupperschüsseln ist für eine bestimmte Menge Kartoffelsalat groß
genug?  Welches ist die richtige Abzweigung nach Dortmund?  Solche
Entscheidungen sind daran festgemacht, daß ein Wert zu einer von mehreren
verschiedenen 
Kategorien gehören kann~-- es handelt sich dann um eine sogenannte
\textit{Fallunterscheidung\index{Fallunterscheidung}}; 
mathematische Funktionen und Scheme-Funktionen operieren auf Daten mit
Fallunterscheidung durch \textit{Verzweigungen\index{Verzweigung}}.
Um diese geht es in diesem Kapitel.

\section{Fallunterscheidungen}
\label{sec:fallunterscheidungen}

Zu den "`Flensburg"'-Punkten, die es bei Verstößen gegen die
Straßenverkehrsordnung gibt, hat eine Seite im Internet folgendes zu
sagen:
%%%% Das ist leider nicht mehr die gültige Version der Punkte! HK
%
\begin{description}
\item[0 bis 3 Punkte] Keine Sanktionen
\item[4 bis 8 Punkte] Bei freiwilliger Teilnahme an Aufbauseminaren: 4 Punkte Abzug
\item[8 bis 13 Punkte] Verwarnung und Hinweis auf freiwilliges Aufbauseminar
\item[9 bis 13 Punkte] Bei freiwilliger Teilnahme an Aufbauseminaren: 2 Punkte Abzug
\item[14 bis 17 Punkte] Teilnahme an Aufbauseminar wird angeordnet
\item[14 bis 17 Punkte]
Bei freiwilliger Teilnahme an verkehrspsychologischer Beratung: 2 Punkte Abzug
\item[Ab 18 Punkte]
Führerschein wird entzogen
\end{description}
%
Wir können zu dieser Aufstellung eine Reihe von Fragestellungen
bearbeiten, zum Beispiel welche Sanktionen durch eine bestimmte
Punktezahl verpflichtend werden oder wieviele Punkte ein Autofahrer
nach einer bestimmten Maßnahme noch auf dem Konto hat.  In jedem Fall
teilt die Aufstellung mögliche Punktezahlen in bestimmte Kategorien
ein, je nach Maßnahme.  Folgende Maßnahmen gibt es:
%
\begin{itemize}
\item nichts
\item Aufbauseminar
\item verkehrspsychologische Beratung
\item Führerscheinentzug
\end{itemize}
%
Dies ist eine \textit{Aufzählung}\index{Aufzählung}.

Wir fangen mit der Fragestellung an, welche Zwangsmaßnahme für eine
bestimmte Punktezahl angeordnet wird.  Zwangsmaßnahmen gibt es nur
drei: keine, Aufbauseminar und Führerscheinentzug, da die
verkehrspsychologische Beratung rein freiwillig ist.  Entsprechend
zerfällt die Punkteskala in drei Teile: $0-13$, $14-17$ und ab $18$.
Die Punktezahl gehört also zu einer von drei
\textit{Kategorien}\index{Kategorie}.

Wenn die Menge, aus der ein Wert kommt, in eine feste
Anzahl von Kategorien aufgeteilt wird und bei einem Wert nur die
Kategorie zählt, ist diese Menge durch eine \textit{Fallunterscheidung} definiert.
Aufzählungen sind damit auch Fallunterscheidungen.

Eine Funktion, die  aus einem Punktestand die Zwangsmaßnahme ermittelt,
sieht folgendermaßen aus:
%
\begin{displaymath}
  m(p) \deq
  \begin{cases}
    \textit{nichts} & \textrm{falls $p \leq 13$}
    \\
    \textit{Aufbauseminar} & \textrm{falls $p \geq 14, p \leq 17$}
    \\
    \textit{Führerscheinentzug } & \textrm{falls $p \geq 18$}
  \end{cases}
\end{displaymath}
%
Die Notation mit der großen geschweiften Klammer heißt
\textit{Verzweigung\index{Verzweigung}} (engl.\
\textit{conditional\index{conditional}}); ein Ausdruck wie $p\leq 13$,
der wahr oder falsch sein kann,
heißt \textit{Bedingung\index{Bedingung}} oder
\textit{Test\index{Test}}.

\section{Boolesche Ausdrücke in Scheme}

Tests gibt es auch in Scheme; sie sind dort Ausdrücke.
Hier ein Beispiel:
%
\begin{alltt}
(<= -5 0)
\evalsto{} #t
\end{alltt}
%
\texttt{(<= -5 0)} ist die Scheme-Schreibweise für $-5 \leq 0$.  Als Frage
gestellt oder Aussage aufgefaßt, ist $-5 \leq 0$ "`wahr"'.
\verb|#t|\index{#t@\verb/#t/} steht für "`true\index{true}"' oder
"`wahr\index{wahr}"'.  \texttt{<=}\index{<=@\texttt{<=}} ist eine
eingebaute Funktion, welche auf "`kleiner gleich"' testet.  (Ebenso
gibt es auch \texttt{=}\index{=@\texttt{=}} für $=$, 
\texttt{<}\index{<@\texttt{<}} für $<$,
\texttt{>}\index{>@\texttt{>}} für $>$ und
\texttt{>=}\index{>=@\texttt{>=}} für $\geq$.)

Ein Test kann auch negativ ausfallen:
%
\begin{alltt}
(<= 5 0)
\evalsto{} #f
\end{alltt}
%
\verb|#f|\index{#f@\verb/#f/} steht für "`false"' oder "`falsch"'.
"`Wahr"' und "`falsch"' heißen zusammen \textit{boolesche
  Werte\index{boolescher Wert}} oder auch
\textit{Wahrheitswerte\index{Wahrheitswert}}.\footnote{Die booleschen
  Werte sind benannt nach \textit{George Boole}
  (1815--1864), der zuerst einen algebraischen Ansatz für die
  Behandlung von Logik mit den Werten "`wahr"' und "`falsch"'
  formulierte.}

Analog zu \texttt{=} für Zahlen können Zeichenketten mit
\texttt{string=?}\index{string=?@\texttt{string=?}} verglichen werden:
\begin{alltt}
(string=? "Mike" "Mike")
\evalsto{} #t
(string=? "Herbert" "Mike")
\evalsto{} #f
\end{alltt}

\verb|#t| und \verb|#f| sind wie Zahlen Literale, können also
auch in Programmen stehen:
%
\begin{alltt}
#t
\evalsto{} #t
#f
\evalsto{} #f
\end{alltt}
%
Auch mit booleschen Werten kann DrRacket rechnen.  Ein Ausdruck der
Form\index{and@\texttt{and}}
%
\begin{alltt}
(and \(e\sb{1}\) \(e\sb{2}\) \(\ldots\) \(e\sb{n}\))
\end{alltt}
%
ergibt immer dann \verb|#t|, wenn alle $e_i$ \verb|#t| ergeben, sonst
\verb|#f|.  Bei zwei Operanden $e_1$ und $e_2$ ergibt \texttt{(and
  $e_1$ $e_2$)} immer dann \verb|#t|, wenn $e_1$ \emph{und} $e_2$
\verb|#t| ergeben:
%
\begin{alltt}
(and #t #t)
\evalsto{} #t
(and #f #t)
\evalsto{} #f
(and #t #f)
\evalsto{} #f
(and #f #f)
\evalsto{} #f
\end{alltt}
%
Entsprechend gibt es Ausdrücke der Form\index{or@\texttt{or}}
%
\begin{alltt}
(or \(e\sb{1}\) \(e\sb{2}\) \(\ldots\) \(e\sb{n}\))
\end{alltt}
%
die immer dann \verb|#t| ergeben, wenn \emph{einer} der $e_i$ \verb|#t| ergibt, sonst
\verb|#f|.  Bei zwei Operanden $e_1$ und $e_2$ ergibt \texttt{(or
  $e_1$ $e_2$)} immer dann \verb|#t|, wenn $e_1$ \emph{oder} $e_2$
\verb|#t| ergeben:
%
\begin{alltt}
(or #t #t)
\evalsto{} #t
(or #f #t)
\evalsto{} #t
(or #t #f)
\evalsto{} #t
(or #f #f)
\evalsto{} #f
\end{alltt}
%
Des weiteren gibt es noch eine eingebaute Funktion
\texttt{not}\index{not@\texttt{not}}, die einen booleschen Wert
umdreht, sich also folgendermaßen verhält:
%
\begin{alltt}
(not #f)
\evalsto{} #t
(not #t)
\evalsto{} #f
\end{alltt}
%

\section{Programmieren mit Fallunterscheidungen}

Zurück zu den Punkten in Flensburg: Zunächst schreiben wir eine
Funktion, die zu einem gegebenen Punktestand die entsprechende
Zwangsmaßnahme ausrechnet.  Fast alles, was zur Datenanalyse gehört,
haben wir schon am Anfang des Kapitels gemacht: der Punktestand ist
eine natürliche Zahl, eine Zwangsmaßnahme ist nichts, ein
Aufbauseminar oder der Führerscheinentzug.  Diese Informationen müssen
wir noch in Daten umwandeln~-- dazu benutzen wir einfach die
entsprechenden Zeichenketten \verb|"nichts"|, \verb|"Aufbauseminar"|
und \verb|"Führerscheinentzug"| und halten das Ergebnis in einem
Kommentar fest:
%
\begin{alltt}
; Eine Zwangsmaßnahme ist einer der folgenden Werte:
; - "nichts"
; - "Aufbauseminar"
; - "Führerscheinentzug"
\end{alltt}
%
FIXME: Signatur benamsen, nach Motivation
%
Die Kurzbeschreibung der Funktion könnte so aussehen:
%
\begin{alltt}
; Zwangsmaßnahme bei Flensburg-Punktestand errechnen
\end{alltt}
%
Eine passende Signatur ist diese
hier:\index{points-must-do@\texttt{points-must-do}}\label{page:points-must-do}
%
\begin{alltt}
(: points-must-do (natural -> (one-of "nichts"
                                      "Aufbauseminar"
                                      "Führerscheinentzug")))
\end{alltt}
%
Die Konstruktion \texttt{one-of}\index{one-of@\texttt{one-of}}
bei Signaturen ist neu: In der obigen
Signatur bedeutet es, daß der Aggregatzustand einer der in der
\texttt{one-of}-Signatur angegegebenen Werte ist, also eine der
Zeichenketten \verb|"nichts"|, \verb|"Aufbauseminar"| und \verb|"Führerscheinentzug"|.

Hier sind zwei mögliche Testfälle:
%
\begin{alltt}
(check-expect (points-must-do 14) "Aufbauseminar")
(check-expect (points-must-do 18) "Führerscheinentzug")
\end{alltt}   
%
Es folgt das Gerüst der Funktion:
%
\begin{alltt}
(define points-must-do
  (lambda (p)
    ...))
\end{alltt}
%
Auf jeden Fall muß das \texttt{p} irgendwo im Rumpf vorkommen:
%
\begin{alltt}
(define points-must-do
  (lambda (p)
    ... p ...))
\end{alltt}
%
\begin{feature}{Verzweigung}{scheme:cond}
In Scheme werden Verzweigungen\index{Verzweigung}\index{Verzweigung}
mit der Spezialform \texttt{cond}\index{cond@\texttt{cond}} dargestellt.
Ein \texttt{cond}"=Ausdruck hat die folgende Form:
%
\begin{alltt}
(cond
  (\(t\sb{1}\) \(a\sb{1}\))
  (\(t\sb{2}\) \(a\sb{2}\))
  \(\ldots\)
  (\(t\sb{n-1}\) \(a\sb{n-1}\))
  (else \(a\sb{n}\))))
\end{alltt}
%
Dabei sind die $t_i$ und die $a_i$ ihrerseits Ausdrücke.  Der
\texttt{cond}-Ausdruck wertet nacheinander alle Tests $t_i$ aus;
sobald ein Test $t_k$ \texttt{\#t} ergibt, wird der
\texttt{cond}-Ausdruck durch das entsprechende $a_k$ ersetzt.  Wenn
alle Tests fehlschlagen, wird durch $a_n$ ersetzt.  Die Paarungen
\texttt{($t_i$ $a_i$)} heißen \textit{Zweige\index{Zweig}} des
\texttt{cond}-Ausdruckes, und der Zweig mit \texttt{else} (auf deutsch
"`sonst"') heißt
\textit{\texttt{else}-Zweig\index{else-Zweig@\texttt{else}-Zweig}}.
Der \texttt{else}-Zweig kann auch fehlen~-- dann sollte aber immer
einer der Tests \texttt{\#t} ergeben.  Wenn doch einmal bei allen
$t_i$ \verb|#f| herauskommen sollte, bricht \drscheme{} das Programm ab
und gibt eine Fehlermeldung aus.
\end{feature}
%
Jetzt brauchen wir, wie bei der mathematischen Funktion $m$ aus
Abschnitt~\ref{sec:fallunterscheidungen}, eine Verzweigung, nur eben
in Scheme.  Abbildung~\ref{scheme:cond} beschreibt das dafür
zuständige \texttt{cond}.   Es ist also von vorneherein klar, daß eine
\texttt{cond}-Form im Rumpf von \texttt{points-must-do} auftauchen muß:
%
\begin{alltt}
(define points-must-do
  (lambda (p)
    (cond ... p ...)))
\end{alltt}
%
Bei der Konstruktion der \texttt{cond}-Formen ist entscheidend,
wieviele Zweige sie hat.  Dabei gibt es eine einfache Faustregel~-- da
die Eingabe von \texttt{points-must-do}~-- die Punktzahl~-- in
\emph{drei} Kategorien zerfällt, braucht die \texttt{cond}-Form auch
\emph{drei} Zweige:
%
\begin{alltt}
(define points-must-do
  (lambda (p)
    (cond
      (... ...)
      (... ...)
      (... ...))))
\end{alltt}
%
Wir benötigen jetzt für jeden \texttt{cond}-Zweig einen Test, der die
entsprechende Kategorie bei den Punkten identifiziert.  Dazu müssen
wir nur die entsprechenden Bedingungen aus der mathematischen Fassung
nach Scheme übersetzen.  Heraus kommt folgendes:
%
\begin{alltt}
(define points-must-do
  (lambda (p)
    (cond
      ((<= p 13) ...)
      ((and (>= p 14) (<= p 17)) ...)
      ((>= p 18) ...))))
\end{alltt}
%
Der letzte Schritt ist einfach~-- wir fügen für jeden Zweig die zum
Test passende Maßnahme ein:
%
\begin{alltt}
(define points-must-do
  (lambda (p)
    (cond
      ((<= p 13) "nichts")
      ((and (>= p 14) (<= p 17)) "Aufbauseminar")
      ((>= p 18) "Führerscheinentzug"))))
\end{alltt}
%
Fertig~-- könnte man meinen.  Wenn Sie das Programm in der REPL laufen lassen,
meldet DrRacket zwei bestandene Tests.  Allerdings fällt Ihnen
vielleicht auf, daß das Programm im Definitionsfenster nach dem Lauf
so aussieht:
%
\begin{alltt}
(define points-must-do
  (lambda (p)
    (cond
      ((<= p 13) \colorbox{featuregray}{"nichts"})
      ((and (>= p 14) (<= p 17)) "Aufbauseminar")
      ((>= p 18) "Führerscheinentzug"))))
\end{alltt}
%
Das \verb|"nichts"| ist farbig unterlegt, weil DrRacket diesen
Ausdruck noch nie ausgewertet hat.  Das ist nicht gut, weil es heißt,
daß der entsprechende Zweig durch die Tests noch nicht strapaziert
wurde~-- er ist also möglicherweise fehlerhaft.  Anders gesagt: Die
\textit{Abdeckung}\index{Abdeckung} des Programms durch die Tests ist
unvollständig.  Wenn wir einen Testfall für den ersten Zweig ergänzen,
verschwindet die farbige Unterlegung:
%
\begin{alltt}
(check-expect (points-must-do 0) "nichts")
\end{alltt}
%
Trotzdem sind die bestehenden Tests noch suboptimal~-- wer sagt
schließlich, daß das Programm zum Beispiel bei 13 Punkten, also genau
an der Grenze zwischen dem ersten und zweiten Zweig, das richtige tut.
Wir sollten für diese Eckfälle\index{Eckfall} auch Testfälle
bereitstellen sowie einen Testfall, der sicherstellt, daß auch bei
Punktzahlen oberhalb von 18 immer noch der Führerschein entzogen wird:
%
\begin{alltt}
(check-expect (points-must-do 13) "nichts")
(check-expect (points-must-do 17) "Aufbauseminar")
(check-expect (points-must-do 100) "Führerscheinentzug")
\end{alltt}

\begin{mantra}[Abdeckung und Eckfälle]\label{mantra:coverage}
    Sorgen Sie für vollständige Abdeckung Ihres Programms durch die
  Testfälle!  Testen Sie möglichst alle Eckfälle!

\end{mantra}

\section{Konstruktionsanleitung für Fallunterscheidungen}

Bei der Konstruktion der Funktion \texttt{points-must-do} haben wir
ein bestimmtes Schema angewendet.  Dieses Schema geht zunächst von folgender
Frage aus:
%
\begin{quote}
  \emph{Wieviele} Kategorien gibt es bei der Fallunterscheidung?
\end{quote}
%
Ist die Frage beantwortet~-- durch eine Zahl $n$~-- können wir bereits
etwas Code in den Rumpf schreiben, nämlich eine Verzweigung mit $n$
Zweigen:
%
\begin{alltt}
(define \(p\)
  (lambda (\ldots)
    (cond
      (... ...)
      \ldots{}\hspace{1in}\textrm{(\(n\) Zweige)}
      (... ...))))
\end{alltt}
%
Solch ein "`Rumpf mit Lücken"' (die Ellipsen\index{Ellipse}
\texttt{\ldots} stehen für noch zu ergänzende Programmteile) ist eine
\textit{Schablone\index{Schablone}}.  Wir, die Autoren, empfehlen
Ihnen, die Schablone bereits hinzuschreiben, wenn Sie die Anzahl der
Kategorien bereits kennen, noch bevor Sie weiter über die
Problemstellung nachdenken.  Das hilft oft, etwaige Denkblockaden zu
lösen.

Die Schablone folgt in diesem Fall aus der Struktur der Daten, also
dem Ergebnis der Datenanalyse.  Es gibt noch andere Arten von Daten,
jede mit ihrer eigenen Schablone.  Diese werden wir im Rest des Buchs
entwickeln.
Für alle folgenden Konstruktionsanleitungen gilt deshalb folgendes Mantra:

\begin{mantra}[Schablone]\label{mantra:data-analysis}
    Benutzen Sie ausgehend von einer Datenanalyse\index{Datenanalyse}
  die passende Schablone!

\end{mantra}

Was die Fallunterscheidung betrifft, können wir die Schablone aber
noch weiterentwickeln, indem wir Tests für die einzelnen Fälle der
Fallunterscheidung ergänzen.

Die Schablone für Fallunterscheidungen ist noch einmal
als Konstruktionsanleitung~\ref{ka:fallunterscheidung} in
Anhang~\ref{app:konstruktionsanleitungen} zusammengefaßt.
(Konstruktionsanleitung~\ref{ka:allgemein} beschreibt die Konstruktion
von Funktionen im allgemeinen.)

\section{Verkürzte Tests}

Natürlich könnten wir die Funktion auch leicht abkürzen:
%
\begin{alltt}
(define points-must-do
  (lambda (p)
    (cond
      ((<= p 13) "nichts")
      ((<= p 17) "Aufbauseminar")
      (else "Führerscheinentzug"))))
\end{alltt}
%
Wenn die Auswertung den Test im zweiten Zweig erreicht, steht schon
fest, daß die Punktezahl $\geq 14$ ist, da der Test \verb|(<= p 13)|
fehlgeschlagen ist.  Diese Bedingung könnten wir also weglassen.
Ebenso der letzte Test, der dadurch, daß \verb|(<= p 17)| \verb|#f|
ergeben hat, immer \verb|#t| ergibt.  Allerdings sind die Zweige damit
von ihrer Reihenfolge abhängig: Wenn wir zum Beispiel die ersten
beiden Zweige vertauschten, funktioniert die Funktion nicht mehr
richtig.  In Fällen wie diesen, wo "`vollständige"' Tests einfach zu
formulieren sind, empfiehlt es sich, dies auch zu tun.

\begin{mantra}[Vollständige Tests]\label{mantra:comprehensive-tests}
    Schreiben Sie wenn möglich bei Verzweigungen vollständige Tests, so
  daß die Verzweigung unabhängig von der Reihenfolge der Zweige ist.

\end{mantra}

\section{Binäre Verzweigungen und syntaktischer Zucker}
\label{sec:binaere-verzweigungen}

Bei manchen Fallunterscheidungen definiert sich die letzte Kategorie
dadurch, daß ein Wert in keine der anderen Kategorien gehört.  Dann
ist die Benutzung eines \texttt{else}-Zweigs im \texttt{cond}
sinnvoll.
FIXME: Beispiel wäre schön, dann vielleicht früher.  
Manchmal gibt es dabei nur zwei Kategorien, wie
zum Beispiel beim Absolutbetrag.  Hier die Definition dazu in mathematischer
Schreibweise:
%
\begin{displaymath}
  |x| \deq{} \left\{\begin{array}{rl}
      x & \textrm{falls } x \geq 0\\
      -x & \textrm{andernfalls}
    \end{array}
    \right.
\end{displaymath}
%
Die dazu passende Scheme-Funktion unter Verwendung von \texttt{cond}
sieht so aus:\index{abs@\texttt{abs}}
%
\begin{alltt}
; Absolutbetrag einer Zahl berechnen
(: absolute (number -> number))
(define absolute
  (lambda (x)
    (cond
     ((>= x 0) x)
     (else (- x)))))
\end{alltt}
FIXME: besseres Beispiel

%
Dieser Spezialfall mit nur zwei Kategorien, genannt \textit{binäre
  Verzweigung\index{binäre Verzweigung}\index{Verzweigung!binär}} kommt in der Praxis
häufig vor.  In Scheme gibt es dafür eine eigene Spezialform,
genannt \texttt{if\index{if@\texttt{if}}}, die hier kürzer ausfällt
als \texttt{cond}:
%
\begin{alltt}
(define absolute
  (lambda (x)
    (if (>= x 0)
        x
        (- x))))
\end{alltt}
%
Eine \texttt{if}-Form hat folgende Form:
%
\begin{alltt}
(if \(t\) \(k\) \(a\))
\end{alltt}
Dabei ist $t$ der Test und $k$ und $a$ sind die
beiden Zweige: die \textit{Konsequente\index{Konsequente}} $k$ und die
\textit{Alternative\index{Alternative}} $a$.  Abhängig vom Ausgang des
Tests ist der Wert der Verzweigung entweder der Wert der Konsequente
oder der Wert der Alternative.

Tatsächlich ist \texttt{if} die "`primitivere"' Form als
\texttt{cond}: jede \texttt{cond}-Form kann in eine äquivalente
\texttt{if}-Form übersetzt werden, und zwar nach
folgendem Schema:
%
\begin{alltt}
(cond (\(t\sb{1}\) \(a\sb{1}\)) (\(t\sb{2}\) \(a\sb{2}\)) \(\ldots\) (\(t\sb{n-1}\) \(a\sb{n-1}\)) (else \(a\sb{n}\)))
  \(\mapsto\) (if \(t\sb{1}\) \(a\sb{1}\) (if \(t\sb{2}\) \(a\sb{2}\) \ldots (if \(t\sb{n-1}\) \(a\sb{n-1}\) \(a\sb{n}\))\ldots))
\end{alltt}
%
Die geschachtelte \texttt{if}-Form auf der rechten Seite der
Übersetzung wertet, genau wie die \texttt{cond}-Form, nacheinander
alle Tests aus, bis einer \verb|#t| liefert.  Die rechte Seite des
\texttt{cond}-Zweigs ist dann gerade die Konsequente des \texttt{if}s.
Erst wenn alle Tests fehlschlagen ist die Alternative des letzten
\texttt{if}-Ausdrucks dran, nämlich $a_n$ aus dem \texttt{else}-Zweig.

Da sich mit Hilfe dieser Übersetzung jede \texttt{cond}-Form durch
geschachtelte \texttt{if}-Formen ersetzen läßt, ist \texttt{cond}
streng genommen gar nicht notwendig.  \texttt{Cond} ist deswegen eine
sogenannte \textit{abgeleitete Form\index{abgeleitete
    Form}}\index{Form!abgeleitet}.  Da \texttt{cond} und andere
abgeleitete Formen trotzdem praktisch und angenehm zu verwenden sind
und damit dem Programmierer die Arbeit versüßen,
heißen abgeleitete Formen auch \textit{syntaktischer
  Zucker\index{syntaktischer Zucker}\index{Zucker, syntaktischer}}.

Um die Funktionsweise von Verzweigungen genau zu beschreiben, dient
folgende zusätzliche Regel für das Substitutionsmodell aus
Abschnitt~\ref{sec:substitution-model}:
%
\begin{description}
\item[binäre Verzweigungen] Bei der Auswertung einer Verzweigung wird
  zunächst der Wert des Tests festgestellt.  Ist dieser Wert \verb|#t|,
  so ist der Wert der Verzweigung der Wert der Konsequente.  Ist er
  \verb|#f|, so ist der Wert der Verzweigung der Wert der
  Alternative.  Ist der Wert des Tests kein boolescher Wert ist, ist das Programm fehlerhaft.
\end{description}
%
Auch \texttt{and} und \texttt{or} sind eigentlich syntaktischer Zucker:
Es ist immer möglich, einen \texttt{and}-Ausdruck in \texttt{if}s
zu übersetzen.  Es gelten folgende Übersetzungsregeln:
%
\begin{alltt}
(and) \(\mapsto\) #t
(and \(e\sb{1}\) \(e\sb{2}\) \(\ldots\)) \(\mapsto\) (if \(e\sb{1}\) (and \(e\sb{2}\) \(\ldots\)) #f)
\end{alltt}
%
Ein \texttt{and}-Ausdruck mit mehreren Operanden wird so schrittweise
in eine Kaskade von \texttt{if}-Ausdrücken übersetzt:
%
\begin{alltt}
(and a b c)
\(\mapsto{}\) (if a (and b c) #f)
\(\mapsto{}\) (if a (if b (and c) #f) #f)
\(\mapsto{}\) (if a (if b (if c (and) #f) #f) #f)
\(\mapsto{}\) (if a (if b (if c #t #f) #f) #f)
\end{alltt}
%
Ebenso lassen sich \texttt{or}-Ausdrücke immer in
\texttt{if}-Ausdrücke übersetzen, und zwar mit folgender Übersetzung:
%
\begin{alltt}
(or) \(\mapsto\) #f
(or \(e\sb{1}\) \(e\sb{2}\) \(\ldots\)) \(\mapsto\) (if \(e\sb{1}\) #t (or \(e\sb{2}\) \(\ldots\)))
\end{alltt}
%
Beispiel:
%
\begin{alltt}
(or a b c)
\(\mapsto{}\) (if a #t (or b c))
\(\mapsto{}\) (if a #t (if b #t (or c)))
\(\mapsto{}\) (if a #t (if b #t (if c #t (or))))
\(\mapsto{}\) (if a #t (if b #t (if c #t #f)))
\end{alltt}

\section{Signaturdefinitionen}

Nehmen wir uns zu Übungszwecken noch eine weitere Aufgabe vor: Nehmen
wir an, jemand nimmt bei einem bestimmten Punktestand in Flensburg an
einer freiwilligen Maßnahme teil~-- was ist der Punktestand nach der
Maßnahme?  Die bekannten Größen sind:
%
\begin{itemize}
\item Punktestand vor der Maßnahme (natürliche Zahl)
\item freiwillige Maßnahme (siehe Abschnitt~\ref{sec:fallunterscheidungen})
\end{itemize}
%
Die unbekannte Größe ist der Punktestand nach der Maßnahme.

Die Kurzbeschreibung könnte so lauten:
%
\begin{verbatim}
; Punktestand in Flensburg senken
\end{verbatim}
%
Die Signatur folgt aus der Datenanalyse:\index{improve-points@\texttt{improve-points}}
%
\begin{verbatim}
(: improve-points (natural (one-of "nichts"
                                   "Aufbauseminar"
                                   "verkehrspsychologische Beratung"
                                   "Führerscheinentzug")
                    -> natural))
\end{verbatim}
%
Der \texttt{one-of}-Teil der Signatur macht sich da ganz schön breit,
zumal er sich weitgehend deckt mit dem entsprechenden Teil der
Signatur von \texttt{points-must-do} auf
Seite~\pageref{page:points-must-do}.  Entsprechend sollten wir genauso
wie bei anderen Werten der Signatur für "`Flensburg"=Maßnahmen"' einen
Namen geben.  Das geht mit einer fast ganz normalen Definition:\index{Signaturdefinition}
%
\begin{verbatim}
(define action
  (signature
   (one-of "nichts"
           "Aufbauseminar"
           "verkehrspsychologische Beratung"
           "Führerscheinentzug")))
\end{verbatim}
%
Das Wörtchen \texttt{signature}\index{signature@texttt{signature}}\label{page:signature} ist aus technischen Gründen
nötig.\footnote{Es sorgt unter anderem dafür, daß Signaturdefinitionen
  in beliebiger Reihenfolge geschrieben werden und die Links in den
  Fehlermeldungen von DrRacket auf die richtige Stelle zeigen.}
Faustregel: Signaturen außerhalb von Formen \texttt{(: \ldots)} müssen immer in ein \texttt{(signature
  \ldots)} eingeschachtelt werden..

Mit dieser Definition gewappnet können wir die Signatur abkürzen:
%
\begin{verbatim}
(: improve-points (natural action -> natural)) 
\end{verbatim}
%
Entsprechend Mantra~\ref{mantra:coverage} versuchen wir, durch mehr
Tests als noch bei \texttt{points-must-do} bessere Abdeckung zu
erzielen:\label{page:improve-points-tests}
%
\begin{verbatim}
(check-expect (improve-points 3 "Aufbauseminar") 3)
(check-expect (improve-points 4 "nichts") 4)
(check-expect (improve-points 4 "Aufbauseminar") 0)
(check-expect (improve-points 8 "Aufbauseminar") 4)
(check-expect (improve-points 9 "Aufbauseminar") 7)
(check-expect (improve-points 13 "Aufbauseminar") 11)
(check-expect (improve-points 14 "verkehrspsychologische Beratung") 12)
(check-expect (improve-points 17 "verkehrspsychologische Beratung") 15)
(check-expect (improve-points 18 "Aufbauseminar") 18)
(check-expect (improve-points 18 "verkehrspsychologische Beratung") 18)
\end{verbatim}
%
Hier das Gerüst:
%
\begin{verbatim}
(define improve-points
  (lambda (p a)
    ...))
\end{verbatim}
%
Bei der Konstruktion der Schablone müssen wir uns entscheiden, an
welchem Parameter wir uns orientieren, \texttt{p} oder \texttt{a}.
Die Entscheidung ist willkürlich~-- wir entscheiden uns erst einmal
für \texttt{p}.  (Ausgehend von \texttt{a} kommt eine andere aber
genauso gute Lösung heraus~-- das sei Ihnen in
Aufgabe~\ref{aufgabe:improve-points-a} als Fingerübung empfohlen.)
Bei \texttt{p} gibt es in Bezug auf diese Aufgabe fünf Kategorien:
%
\begin{description}
\item[0--3 Punkte] Da bringt keine Maßnahme etwas.
\item[4--8 Punkte] Da bringt ein Aufbauseminar 4 Punkte Abzug.
\item[9--13 Punkte] Da bringt ein Aufbauseminar 2 Punkte Abzug.
\item[14--17 Punkte] Da bringt eine verkehrspsychologische Beratung 2
  Punkte Abzug.
\item[über 18 Punkte] Auch hier hilft keine Maßnahme.
\end{description}
%
Wir brauchen also ein \texttt{cond} mit fünf Zweigen:
%
\begin{verbatim}
(define improve-points
  (lambda (p a)
    (cond
      (... ...)
      (... ...)
      (... ...)
      (... ...)
      (... ...))))
\end{verbatim}
%
Jetzt müssen wir Tests erfinden, die den Kategorien entsprechen:
%
\begin{verbatim}
(define improve-points
  (lambda (p a)
    (cond
      ((<= p 3) ...)
      ((and (>= p 4) (<= p 8)) ...)
      ((and (>= p 9) (<= p 13)) ...)
      ((and (>= p 14) (<= p 17)) ...)
      ((>= p 18) ...))))
\end{verbatim}
%
Wir fangen mal mit den einfachsten Fällen an~-- unten und oben in der
Punkteskala, wo sich nichts bewegt:
%
\begin{verbatim}
(define improve-points
  (lambda (p a)
    (cond
      ((<= p 3) p)
      ((and (>= p 4) (<= p 8)) ...)
      ((and (>= p 9) (<= p 13)) ...)
      ((and (>= p 14) (<= p 17)) ...)
      ((>= p 18) p))))
\end{verbatim}
%
Im zweiten Zweig~-- zwischen vier und acht Punkten~-- zählt nur ein
Aufbauseminar, alle anderen Maßnahmen bringen nichts.  Darum ist hier
eine binäre Verzweigung angemessen:
%
\begin{verbatim}
(define improve-points
  (lambda (p a)
    (cond
      ...
      ((and (>= p 4) (<= p 8))
       (if (string=? a "Aufbauseminar")
           (- p 4)
           p))
      ...)))
\end{verbatim}
%
Entsprechend funktionieren auch der dritte und der vierte Zweig:
%
\begin{verbatim}
(define improve-points
  (lambda (p a)
    (cond
      ((<= p 3) p)
      ((and (>= p 4) (<= p 8))
       (if (string=? a "Aufbauseminar")
           (- p 4)
           p))
      ((and (>= p 9) (<= p 13))
       (if (string=? a "Aufbauseminar")
           (- p 2)
           p))
      ((and (>= p 14) (<= p 17))
       (if (string=? a "verkehrspsychologische Beratung")
           (- p 2)
           p))
      ((>= p 18) p))))
\end{verbatim}
%
Fertig!  Es gibt trotzdem noch einen Wermutstropfen: Die Abdeckung ist
trotz der vielen Tests immer noch nicht vollständig~-- siehe
Aufgabe~\ref{aufgabe:improve-points-coverage}

\section{Unsinnige Daten abfangen}
\label{sec:nonsensical-data}

Noch einmal zurück zum Parkplatzproblem, das wir auf
Seite~\ref{page:parking-lot-cars} programmiert hatten.  In
Abschnitt~\ref{sec:nonsensical-data-prequel} auf
Seite~\pageref{sec:nonsensical-data-prequel} hatten wir bereits
bemerkt, daß die Funktion
\texttt{parking-lot-cars}\index{parking-lot-cars@\texttt{parking-lot-cars}}
auch für unsinnige Daten fröhlich ebenso unsinnige Ergebnisse
ermittelt.

Auf Seite~\ref{page:parking-lot-problem} wurde bereits eine Bedingung
für sinnvolle Daten formuliert: Wenn $n$ die Anzahl der Fahrzeuge und
$m$ die Anzahl der Räder ist, dann muß $m$ gerade sein sowie $2n\leq
m\leq 4n$ gelten.  Wir können das in einer binären Verzweigung zum
Ausdruck bringen:
%
\begin{verbatim}
(define parking-lot-cars
  (lambda (vehicle-count wheel-count)
    (if (and (even? wheel-count)
             (<= (* 2 vehicle-count) wheel-count)
             (<= wheel-count (* 4 vehicle-count)))
        (/ (- wheel-count (* 2 vehicle-count))
           2)
        ...)))
\end{verbatim}
%
Die eingebaute Funktion \texttt{even?} akzeptiert eine ganze Zahl und
liefert \verb|#t|, falls die Zahl gerade ist und \verb|#f|, falls sie
ungerade ist~-- solche und viele andere nützliche Funktionen finden
Sie in der Dokumentation im Hilfezentrum unter "`Sprachebenen und
Material zu \textit{Schreibe Dein Programm!}"' im Abschnitt
"`Primitive Operationen"'.

Nur~-- was tun im Fehlerfall?  Dazu gibt eine eingebaute Funktion
\texttt{violation}, die eine Fehlermeldung als Zeichenkette akzeptiert
und, wenn sie aufgerufen wird, das Programm abbricht und die
Fehlermeldung ausdruckt.  \texttt{Parking-lot-cars} sieht dann
vollständig so aus:
%
\begin{verbatim}
(define parking-lot-cars
  (lambda (vehicle-count wheel-count)
    (if (and (even? wheel-count)
             (<= (* 2 vehicle-count) wheel-count)
             (<= wheel-count (* 4 vehicle-count)))
        (/ (- wheel-count (* 2 vehicle-count))
           2)
        (violation "unsinnige Daten"))))
\end{verbatim}
%
Natürlich sollten wir auch den Fehlerfall testen~-- das geht nicht mit
\texttt{check-expect}, das ja erwartet, daß ein Testausdruck einen
ordnungsgemäßen Wert liefert.  Für Fehlerfälle gibt es
\texttt{check-error}, das Testfälle erzeugt, die dann bestanden sind,
wenn die Auswertung einen Fehler liefert:
%
\begin{verbatim}
(check-error (parking-lot-cars 10 10)) ; zu wenige Räder
(check-error (parking-lot-cars 3 9))   ; ungerade Räderzahl
(check-error (parking-lot-cars 2 10))  ; zu viele Räder
\end{verbatim}

\section*{Aufgaben}

\begin{aufgabe}
  Schreiben Sie eine Funktion \texttt{card-type}, die den Umsatz einer
  Kreditkarte konsumiert und die eine entsprechende Kategorie als
  Zeichenkette zurückgibt.  Verwenden Sie die Konstruktionsanleitung:
  Schreiben Sie die Kurzbeschreibung auf, führen Sie eine
  Datenanalyse durch und schreiben Sie die Signatur auf. Erstellen Sie
  dann das Gerüst und die Testfälle.  Vervollständigen Sie danach den
  Rumpf der Funktion und vergewissern Sie sich, dass die Tests
  erfolgreich laufen. \\

  \begin{tabular}{crlcrll}
    &        & Umsatz & $<$ & $15.000$   & $\Longrightarrow$ & Weiß \\
    $15.000$  & $\leq$ & Umsatz & $<$ & $50.000 $  & $\Longrightarrow$ & Gold \\
    $50.000$  & $\leq$ & Umsatz & $\leq$ & $150.000 $ 
    & $\Longrightarrow$ & Platin \\
    $150.000$ & $<$ & Umsatz &     &            &  $\Longrightarrow$ & Schwarz \\
  \end{tabular} \\
\end{aufgabe}

\begin{aufgabe}

  \begin{enumerate}

  \item Schreiben Sie eine Funktion \texttt{min-2}, die als Argumente zwei
    Zahlen nimmt und die kleinere der beiden Zahlen zurückgibt.  Schreiben
    Sie außerdem eine Funktion \texttt{min-3}, die als Argumente drei
    Zahlen nimmt und die kleinste der drei Zahlen zurückgibt.  Verwenden
    Sie die Konstruktionsanleitung: Schreiben Sie
    explizit Kurzbeschreibung und Signatur auf, erstellen Sie dann das
    Gerüst und die Testfälle.  Vervollständigen Sie danach den Rumpf der
    Funktion und vergewissern Sie sich, dass die Tests erfolgreich laufen.
    
  \item Schreiben Sie analog eine Funktion \texttt{max-2} und \texttt{max-3}.
    
  \end{enumerate}
\end{aufgabe}

\begin{aufgabe}
  Schreiben Sie die folgenden Funktionen:
  \begin{enumerate}
  \item \texttt{min-of-two}, welche die kleinste von zwei
    gegebenen Zahlen ausgibt
  \item \texttt{min-of-three}, welche die kleinste von drei
    gegebenen Zahlen ausgibt
  \item \texttt{is-min-of-three?}, die überprüft ob die erste
    von drei gegebenen Zahlen das Minimum ist
  \item \texttt{valid-value?}, die überprüft ob die erste von
    drei gegebenen Zahlen zwischen den beiden anderen liegt; gehen Sie
    davon aus, dass der Aufruf immer \texttt{(valid-value? value min max)}
    lautet 
  \item \texttt{clamp}, die wie folgt definiert ist:
    
    \[\text{clamp}(x,\ min,\ max)=
    \begin{cases} 
      x & min \leq x \leq max\\ 
      min & x < min \\ 
      max & x > max 
    \end{cases}
    \]
    
  \end{enumerate}
\end{aufgabe}

\begin{aufgabe}
  Beim Fußball lässt die Rückennummer eines Spielers
  häufig Rückschlüsse auf seine Position zu. Wir machen dabei folgende
  Annahmen:
  \begin{itemize}
  \item Ein \emph{Torwart} hat die Rückennummer 1.
  \item Ein \emph{Abwehrspieler} hat die Rückennummer 2, 3, 4 oder 5.
  \item Ein \emph{Mittelfeldspieler} hat die Rückennummer 6, 7, 8 oder 10.
  \item Ein \emph{Stürmer} hat die Rückennummer 9 oder 11.
  \item Ein \emph{Ersatzspieler} hat eine Rückennummer zwischen 12 und 99.
  \item Alle anderen Rückennummern sind ungültig.
  \end{itemize}
 
  Schreiben Sie nun eine Funktion mit folgender Signatur:
  
  {\small
\begin{verbatim}
 (: nummer->position
    (number ->
      (one-of "Torwart" "Abwehr" "Mittelfeld" "Sturm" "Ersatz" "Ungültig")))
\end{verbatim}
  }

  Die Funktion soll dabei zu einer gegebenen Rückennummer die
  zugehörige Position berechnen.

  Verwenden Sie beim Schreiben der Funktion die
  Konstruktionsanleitungen für Funktionen und für
  Fallunterscheidungen.  Testen Sie die Funktion
  \texttt{nummer->position} mit mindestens sechs Testfällen, so dass
  alle Fälle abgedeckt sind.
\end{aufgabe}


\begin{aufgabe}
  Schreiben Sie ein Programm, mit dem Bußgelder
  automatisch bestimmt werden.
  
  \begin{enumerate}
  \item Programmieren Sie eine Funktion \texttt{zu-langes-parken}
    für die Bewertung von zu langem Parken auf einem kostenpflichtigen
    Parkplatz. Diese bekommt eine Zeitspanne übergeben und gibt das 
    entsprechende Verwarngeld zurück.
    
    Diese Verwarnungen sind wie folgt festgelegt:
    \begin{itemize}
    \item Überschreitung der Höchstparkdauer bis 30 Minuten: \euro{5}
    \item bis zu einer Stunde: \euro{10}
    \item bis zu zwei Stunden: \euro{15}
    \item bis zu drei Stunden: \euro{20}
    \item länger als drei Stunden:  \euro{25}
    \end{itemize}
    
  \item Das Überfahren einer roten Ampel kostet je nach
    Gefährdungslage mehr, gibt Punkte und Fahrverbote. Schreiben Sie
    zwei Funktionen, eine für das Bußgeld \texttt{rote-ampel-bußgeld}, 
    eine für die Punkte in Flensburg \texttt{rote-ampel-punkte} 
    und eine für das Fahrverbot \texttt{rote-ampel-fahrverbot}, 
    welche ausgibt, ob ein Fahrverbot erteilt wird. Übergeben
    Sie den Funktionen, wie lange die Ampel schon rot war und ob eine
    Gefährdung oder Sachbeschädigung vorlag.
    
    Die Bußgelder sind wie folgend definiert:
    \begin{itemize}
    \item Bei Rot über die Ampel innerhalb der ersten Sekunde			
      \euro{50} und 3 Punkte.
    \item Bei Rot über die Ampel innerhalb der ersten Sekunde mit
      Gefährdung oder Sachbeschädigung \euro{125}, 4 		
      Punkte und 1 Monat Fahrverbot.
    \item Bei Rot über die Ampel nach der ersten Sekunde \euro{125},
      4 Punkte und 1 Monat Fahrverbot.
    \item Bei Rot über die Ampel nach der ersten Sekunde mit
      Gefährdung oder Sachbeschädigung \euro{200}, 4
      Punkte und 1 Monat Fahrverbot.
    \end{itemize}
    
    
  \end{enumerate}
\end{aufgabe}


%%% Local Variables: 
%%% mode: latex
%%% TeX-master: "i1"
%%% End: 
