% Diese Datei ist Teil des Buchs "Schreibe Dein Programm!"
% Das Buch ist lizensiert unter der Creative-Commons-Lizenz
% "Namensnennung - Weitergabe unter gleichen Bedingungen 4.0 International (CC BY-SA 4.0)"
% https://creativecommons.org/licenses/by-sa/4.0/deed.de

\chapter{Zeitabhängige Modelle}
\label{cha:representation-and-state}

TBD

\section{Das Teachpack \texttt{image2.ss}}

Für die Grafikprogrammierung mit \drscheme{} ist es notwendig, ein
sogenanntes \textit{Teachpack\index{Teachpack}} zu laden~-- ein
kleiner Sprachzusatz, in diesem Fall mit einer Reihe von Funktionen zur Erzeugung von
Bildern.  Dazu muß im Menü \texttt{Sprache} (oder \texttt{Language} in
der englischen Ausgabe) der Punkt \texttt{Teachpack hinzufügen}
(\texttt{Add teachpack}) angewählt werden, und im dann erscheinenden
Auswahl-Dialog im Verzeichnis \texttt{deinprogramm} die Datei
\texttt{image22.ss\index{image2.ss@\texttt{image2.ss}}}.

Im Teachpack \texttt{image2.ss} erzeugen verschiedene Funktionen
einfache Bilder.  So hat z.B.\ die Funktion \texttt{rectangle}
folgende Signatur:\index{rectangle@\texttt{rectangle}}
%
\begin{alltt}
(: rectangle (natural natural mode color -> image))
\end{alltt}
%
Dabei sind die ersten beiden Argumente Breite und Höhe eines Rechtecks
in Pixeln.
Das Argument von der Sorte \texttt{mode}\index{mode@\texttt{mode}} ist eine Zeichenkette, die
entweder \verb|"solid"| oder \verb|"outline"| sein muß. Sie bestimmt,
ob das Rechteck als durchgängiger Klotz oder nur als Umriß gezeichnet
wird.  Das Argument von der Sorte \texttt{color}\index{color@\texttt{color}} ist eine
Zeichenkette, die eine Farbe (auf Englisch) bezeichnet, 
zum Beispiel \verb|"red"|, \verb|"blue"|, \verb|"yellow"|,
\verb|"black"|, \verb|"white"| oder \verb|"gray"|.  Als Ergebnis
liefert \texttt{rectangle} ein Bild, das von der \drscheme{}-REPL
entsprechend angezeigt wird wie andere Werte auch.

\begin{figure}[tb!]
  \centering
  TBD
  \caption{Teachpack \texttt{image2.ss}}
  \label{fig:image-ss}
\end{figure}

Es gibt es noch weitere Funktionen, die 
geometrische Figuren zeichnen:\index{circle@\texttt{circle}}
%
\begin{alltt}
(: circle (natural mode color -> image))
\end{alltt}
%
Die \texttt{circle}-Funktion liefert einen Kreis, wobei das erste
Argument den Radius angibt.  Die \texttt{mode}- und
\texttt{color}-Argumente sind wie bei \texttt{rectangle}.
%
\begin{alltt}
(: ellipse (natural natural mode color -> image))
\end{alltt}
%
\index{ellipse@\texttt{ellipse}}Diese Funktion liefert eine Ellipse,
wobei das erste Argument die Breite und das zweite die Höhe angibt.
%
\begin{alltt}
(: triangle (natural mode color -> image))
\end{alltt}
%
\index{triangle@\texttt{triangle}}Diese Funktion liefert ein nach oben
zeigendes gleichseitiges Dreieck, wobei das erste Argument die
Seitenlänge angibt.
%
\begin{alltt}
(: line (natural natural real real real real color -> image))
\end{alltt}
%
\index{line@\texttt{line}}zeichnet eine Linie.  Der Aufruf
\texttt{(line $w$ $h$ $x_1$ $y_1$ $x_2$ $y_2$ $c$)} liefert ein Bild
mit Breite $w$ und Höhe $h$, in dem eine Linie von $(x_1, y_1)$ nach
$(x_2, y_2)$ läuft.  Der Ursprung $(0,0)$ ist links \emph{oben},
also nicht, wie in der Mathematik üblich, links unten.

Da diese geometrischen Formen für sich genommen langweilig sind, können
mehrere Bilder miteinander kombiniert werden.

Zum Aufeinanderlegen gibt es die Funktion \texttt{overlay\index{overlay@\texttt{overlay}}}:
%
\begin{alltt}
(: overlay (image image h-place v-place -> image))
\end{alltt}
%
Dabei sind die ersten beiden Argumente die Bilder, die
aufeinandergelegt werden~-- das zweite auf das erste.
Die beiden anderen Argumente geben an, wie
die beiden Bilder zueinander positioniert werden.  Die Signatur
von \texttt{h-place}, das die horizontale Positionierung festlegt,
ist:\index{h-place@\texttt{h-place}}
%
\begin{alltt}
(define h-place
   (signature
      (mixed natural
             (enum "left"
                     "right"
                     "center"))))
\end{alltt}
%
Im ersten Fall, wenn es sich um eine Zahl $x$ handelt, wird das zweite
Bild $x$ Pixel vom linken Rand auf das erste gelegt.  Die drei
Fälle mit Zeichenketten sagen, daß die Bilder am linken Rand bzw.\ am
rechten Rand bündig plaziert werden, bzw.\ das zweite Bild horizontal
in die Mitte des ersten gesetzt wird.
Dementsprechend ist
\texttt{v-place}, das die vertikale Positionierung festlegt,
wie folgt definiert:\index{v-place@\texttt{v-place}}
%
\begin{alltt}
(define h-place
   (signature
      (mixed natural
             (enum "top"
                     "bottom"
                     "center"))))
\end{alltt}
%
Im ersten Fall, wenn es sich um eine Zahl $y$ handelt, wird das zweite
Bild $y$ Pixel vom oberen Rand auf das erste gelegt.  Die drei
Fälle mit Zeichenketten sagen, daß die Bilder am oberen Rand bzw.\ am
unteren Rand bündig plaziert werden, bzw.\ das zweite Bild vertikal
in die Mitte des ersten gesetzt wird.

Das Bild, das bei \texttt{overlay} herauskommt, ist groß genug, daß
beide Eingabebilder genau hineinpassen.

Die folgenden Hilfsfunktionen sind Spezialfälle von \texttt{overlay}:
%
\begin{alltt}
(: above  (image image h-mode -> image))
(: beside (image image v-mode -> image))
\end{alltt}
%
Die Funktion \texttt{above\index{above@\texttt{above}}} ordnet zwei
Bilder übereinander an, \texttt{beside\index{beside@\texttt{beside}}}
nebeneinenander.  Dabei ist \texttt{h-mode} eine der Zeichenketten
\verb|"left"|, \verb|"right"| und \verb|"center"|, die angibt, ob die
Bilder bei \texttt{above} an der linken oder rechten Kante oder der
Mitte ausgerichtet werden.  Entsprechend ist \texttt{v-mode} eine der
Zeichenketten \verb|"top"|, \verb|"bottom"| und \verb|"center"|, die
angibt, ob die Bilder bei \texttt{beside} oben, unten oder an der
Mitte ausgerichtet werden.

Die Funktionen \texttt{clip} und \texttt{pad} beschneiden bzw.\
erweitern ein Bild:\index{clip@\texttt{clip}}\index{pad@\texttt{pad}}
%
\begin{alltt}
(: clip (image natural natural natural natural -> image))
(: pad  (image natural natural natural natural -> image))
\end{alltt}
%
Ein Aufruf \texttt{(clip $i$ $x$ $y$ $w$ $h$)} liefert 
das Teilrechteck des Bildes $i$ mit Ecke bei $(x, y)$, Breite $w$ und
Höhe $h$.  Der Aufruf \texttt{(pad $i$ $l$ $r$ $t$ $b$)}
fügt an den Seiten von $i$ noch transparenten Leerraum an: $l$ Pixel
links, $r$ Pixel rechts, $t$ Pixel oben und $b$ Pixel unten.

Abbildung~\ref{fig:image-ss} zeigt, wie sich die einige der
\texttt{image.ss}-Funktionen in der \drscheme{}-REPL verhalten.

\begin{figure}
  \centering
  TBD
  \caption{Eingefügte Bilder in der \drscheme{}-REPL}
  \label{fig:image-insert}
\end{figure}
%
Es ist auch möglich, externe Bilder-Dateien in
\texttt{image2.ss}-Bilder zu verwandeln.  Dazu dient der Menüpunkt
\texttt{Bild einfügen} im \texttt{Spezial}-Menü:  \drscheme{} fragt nach dem
Namen einer Bilddatei, die dann in den Programmtext da eingefügt wird,
wo der Cursor steht.  Die eingefügten Bilder dienen dann als
Literale für Bild-Objekte.  Abbildung~\ref{fig:image-insert} zeigt ein
Beispiel.

Die folgenden Funktionen ermitteln Breite und Höhe
eines Bildes:\index{image-width@\texttt{image-width}}\index{image-height@\texttt{image-height}}
%
\begin{alltt}
(: image-width  (image -> natural))
(: image-height (image -> natural))
\end{alltt}

\section{Zwischenergebnisse benennen}
\label{sec:let}

\index{lokale Variable}\index{Variable!lokal} Im nächsten Abschnitt
geht es um ein etwas umfangreicheres Programm mit vielen
\textit{Zwischenergebnissen}\index{Zwischenergebnis}.  Die
\texttt{let}-Form erlaubt, Zwischenergebnisse zu benennen und beliebig
oft zu verwenden.  Abbildung~\ref{scheme:let} erläutert die
Funktionsweise.
%
\begin{feature}{Lokale Variablen mit \texttt{let}}{scheme:let}
\texttt{Let}\index{let@\texttt{let}} ist für das Anlegen \textit{lokaler
  Variablen}\index{lokale Variable}\index{Variable!lokal} zuständig.  Ein \texttt{let}-Ausdruck hat die folgende
allgemeine Form:
%
\begin{alltt}
(let ((\(v\sb{1}\) \(e\sb{1}\)) \(\ldots\) (\(v\sb{n}\) \(e\sb{n}\))) \(b\))
\end{alltt}
%
Dabei müssen die $v_i$ Variablen sein und die $e_i$ und
$b$ (der \textit{Rumpf})\index{Rumpf} beliebige Ausdrücke.  Bei der Auswertung
eines solchen \texttt{let}-Ausdrucks werden zunächst alle $e_i$
ausgewertet.  Dann werden deren Werte für die Variablen $v_i$ im Rumpf
eingesetzt; dessen Wert wird dann zum Wert des \texttt{let}-Ausdrucks.

Ein \texttt{let}-Ausdruck hat die gleiche Bedeutung wie folgende
Kombination aus Lambda-Ausdruck und Applikation:
%
\begin{alltt}
(let ((\(v\sb{1}\) \(e\sb{1}\)) \(\ldots\) (\(v\sb{n}\) \(e\sb{n}\))) \(b\))
\(\mapsto\) ((lambda (\(v\sb{1}\) \(\ldots\) \(v\sb{n}\)) \(b\)) \(e\sb{1}\) \(\ldots\) \(e\sb{n}\))
\end{alltt}
%
\end{feature}
%
\texttt{Let} ist selbst dann nützlich, wenn ein Zwischenergebnis nicht
mehrfach verwendet wird.  Es kann die Lesbarkeit des Programmtextes
erhöhen, besonders wenn ein aussagekräftiger Name verwendet wird.
Zum Beispiel berechnet die folgende Funktion das Materialvolumen eines
Rohrs, von dem Außenradius, Dicke und Höhe angegeben sind:\index{pipe-volume@\texttt{pipe-volume}}
%
\begin{alltt}
; Materialvolumen eines Rohrs berechnen
(: pipe-volume (number number number -> number))
(define pipe-volume
  (lambda (outer-radius thickness height)
    (let ((inner-radius (- outer-radius thickness)))
      (- (cylinder-volume outer-radius height)
         (cylinder-volume inner-radius height)))))
\end{alltt}
%
In diesem Beispiel wird eine einzelne lokale Variable namens
\texttt{inner-radius} eingeführt, die für den Wert von \texttt{(-
  outer-radius thickness)} steht.

Da die Variablen, die durch \texttt{let} und \texttt{lambda} gebunden
werden, nur jeweils im Rumpf des \texttt{let} bzw.\ \texttt{lambda}
gelten, heißen sie auch \textit{lokale Variablen}.  Die durch
\texttt{define} gebundenen Variablen heißen dementsprechend~-- da sie überall
gelten~-- \textit{globale Variablen\index{Variable!global}\index{globale Variable}}.

\texttt{Let} kann auch mehrere lokale Variablen
gleichzeitig einführen, wie im folgenden Beispiel:
%
\begin{alltt}
(let ((a 1)
      (b 2)
      (c 3))
  (list a b c))
\evalsto{} #<list 1 2 3>
\end{alltt}
%
Bei der Benutzung von \texttt{let} ist zu beachten, daß die Ausdrücke,
deren Werte an die Variablen gebunden werden, allesamt
\emph{außerhalb} des Einzugsbereich des \texttt{let} ausgewertet
werden.  Folgender Ausdruck führt also bei der Auswertung zu einer
Fehlermeldung:
%
\begin{alltt}
(let ((a 1)
      (b (+ a 1)))
  b)
\evalsto{} reference to an identifier before its definition: a
\end{alltt}
%

\begin{mantra}[lokale Variablen]\label{mantra:local-variables}
    Benenne Zwischenergebnisse mit lokalen Variablen.

\end{mantra}

\section{Modelle und Ansichten}


TBD

\section{Bewegung und Zustand}

TBD  Dafür ist ein weiteres Teachpack namens
\texttt{universe.ss}\index{universe.ss@\texttt{universe.ss}} zuständig.  Es
kann in \drscheme{} genauso wie bei \texttt{image2.ss}
geladen werden.  

Alle Definitionen von
\texttt{image2.ss} sind auch in \texttt{universe.ss} verfügbar.

In der Terminologie von \texttt{universe.ss} ist ein Modell eine
\textit{world}, auf deutsch eine \textit{Welt}: Die Idee dahinter
ist, daß ein Bild eine Ansicht einer kleinen Welt ist.  Damit das
funktioniert, muß bei \texttt{universe.ss} eine erste Welt angemeldet
werden, zusammen mit Angaben, wie groß die Ansicht wird.  Dazu gibt es
die Funktion \texttt{big-bang}\index{big-bang@\texttt{big-bang}}:
%
\begin{alltt}
(: big-bang (natural natural number world -> #t))
\end{alltt}
%
("<Big Bang"> heißt zu deutsch "<Urknall">.)
Die ersten beiden Argumente geben Breite und Höhe der Ansicht an.  Das
dritte Argument gibt die Dauer (in Sekunden) zwischen Ticks der Uhr
an, die für die Animation benötigt wird.  Das vierte Argument gibt
schließlich die erste Welt an.  (Der Rückgabewert, immer
\verb|#t|, ist ohne Bedeutung.)  Für den Himmel mit
Sonne sieht der Aufruf von \texttt{big-bang} folgendermaßen aus:
%
\begin{alltt}
(big-bang sky-width sky-height 0.1 0)
\end{alltt}
%
Dieser Aufruf erzeugt ein Fenster mit Breite und Höhe des Himmels,
startet die Uhr, die jede Sekunde zehnmal tickt, und legt als erste
Welt "<0">, also den Anfang der Zeit fest.  (Eine zehntel Sekunde
reicht etwa aus, damit die Animation dem menschlichen Auge als
"<Bewegung"> erscheint.)

Damit das Teachpack die Welt in eine Ansicht umwandeln kann, muß eine
entsprechende Ansicht angemeldet werden.  Dafür ist die Funktion
\texttt{on-redraw\index{on-redraw@\texttt{on-redraw}}} zuständig:
%
\begin{alltt}
(: on-redraw ((world -> image) -> #t))
\end{alltt}
%
Als Argument akzeptiert \texttt{on-redraw} also eine Funktion, die aus
einer Welt ein Bild macht.  TBD

Auch diese Funktion muß noch beim Teachpack angemeldet werden.  Dafür
die Teachpack-Funktion
\texttt{on-tick-event\index{on-tick-event@\texttt{on-tick-event}}}
zuständig:
%
\begin{alltt}
(: on-tick-event ((world -> world) -> #t))
\end{alltt}
%
Die \texttt{on-tick-event}-Funktion akzeptiert eine Funktion, die bei
jedem Uhren-Tick aufgerufen wird, um aus der "<alten"> Welt eine neue
zu machen.  Auf diese Beschreibung und auch auf die Signatur
paßt aber \texttt{next-time}.  Der Aufruf kann also so aussehen:
%
\begin{alltt}
(on-tick-event next-time)
\end{alltt}
%
Wenn das Programm
beendet werden soll, muß \texttt{on-tick-event} die Funktion
\texttt{end-of-time\index{end-of-time@\texttt{end-of-time}}} des
Teachpacks aufrufen, die folgende Signatur hat:
%
\begin{alltt}
(: end-of-time (string -> world))
\end{alltt}
%

\section{Andere Welten}

Eine kleine (wenn auch nicht besonders sinnvolle) Erweiterung zeigt,
wie die Animation auf Benutzereingaben reagieren kann.  Dazu muß sie
noch eine weitere Funktion anmelden, und zwar mittels
\texttt{on-key-event\index{on-key-event@\texttt{on-key-event}}}, das
ähnlich funktioniert wie \texttt{on-tick-event}:
%
\begin{alltt}
(: on-key-event ((world string -> world) -> #t))
\end{alltt}
%
Die Funktion, die mit \texttt{on-key-event} angemeldet wird, wird
immer aufgerufen, wenn der Benutzer eine Taste drückt.  Welche Taste
gedrückt wurde, gibt das zweite Argument 
an.  Wenn der Benutzer eine reguläre Zeichen-Taste drückt (also keine
Cursor-Taste o.ä.), ist dieses Argument eine Zeichenkette bestehend
aus diesem einen Zeichen.

TBD

\section*{Aufgaben}

TBD

\begin{aufgabe}
  Schreiben Sie ein kleines Telespiel Ihrer Wahl.
\end{aufgabe}

%%% Local Variables: 
%%% mode: latex
%%% TeX-master: "i1"
%%% End: 

