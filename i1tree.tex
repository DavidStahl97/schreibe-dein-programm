% Diese Datei ist Teil des Buchs "Schreibe Dein Programm!"
% Das Buch ist lizensiert unter der Creative-Commons-Lizenz
% "Namensnennung 4.0 International (CC BY 4.0)"
% http://creativecommons.org/licenses/by/4.0/deed.de

\chapter{Bäume}
\label{cha:trees}

\index{Baum}Bäume sind induktive Datenstrukturen,
die viele Anwendungen in der praktischen
Programmierung haben.  Sie sind beispielsweise die Grundlage
für schnelle Suchverfahren in Datenbanken und einige
Datenkompressionsverfahren.  Dieses
Kapitel demonstriert die Programmierung mit Binärbäumen anhand von
Suchbäumen.

\section{Binärbäume}
\label{sec:trees}

Viele Anwendungen von Bäumen bauen auf einem
Spezialfall  auf, den \textit{binären Bäumen\index{binärer
    Baum}\index{Binärbaum}\index{Baum!binär}} oder \textit{Binärbäumen}.  

Die Binärbaume bilden eine induktiv definierte Menge.  Ein Baum ist
eins der folgenden:
%
\begin{itemize}
\item Der \textit{leere Binärbaum\index{leerer Binärbaum}} ist ein Binärbaum.
\item Ein \textit{Knoten\index{Knoten}}, bestehend seinerseits aus zwei Binärbäumen,
  genannt \textit{linker} und \textit{rechter
    Teilbaum}\index{Teilbaum} des Knotens
  und einer beliebigen \textit{Markierung\index{Markierung}} ist ebenfalls ein
  Binärbaum.
\end{itemize}
%
\begin{figure}[h!]
\begin{pspdf}
\begin{center}
  \pstree[levelsep=4ex,treesep=3em,nodesep=2pt]{Axl} { \pstree{\Tr{Slash} {
      \pstree{\Tr{Buckethead}} { \Tdot \pstree{\Tr{Dizzy}} { \Tdot \Tdot } } \Tdot}
    \pstree{\Tr{Duff}} { \pstree{\Tr{Brain}} { \Tdot \Tdot } \Tdot
      } }
\end{center}
\end{pspdf}
  \caption{Binärbaum}
  \label{fig:bintree1}
\end{figure}
Die Markierung an den Knoten kann beliebige Daten aufnehmen, je nach
Verwendungszweck des Baums.

Binärbäume haben eine einleuchtende grafische Darstellung, die in
Abbildung~\ref{fig:bintree1} vorgestellt wird.  Die Punkte unten
am Baum stehen für leere Binärbäume.  Ein Bild der Form:

\begin{pspdf}
\begin{center}
  \pstree[levelsep=4ex,treesep=3em,labelsep=3pt,nodesep=2pt]{\Tr{$L$}}
  {
    \Tr{\raisebox{-1ex}{\ldots}}
    \Tr{\raisebox{-1ex}{\ldots}}
    }
\end{center}
\end{pspdf}

steht für einen Knoten mit Markierung $L$, unter dem sich ein linker und
ein rechter Teilbaum befinden.  Die Teilbäume werden auch
\textit{Zweige}\index{Zweig}  genannt.  Knoten, die als linken und rechten
Teilbaum jeweils den leeren Baum haben, heißen auch
\textit{Blätter\index{Blatt}}.

Die Datendefinition für Bäume zeigt klar, daß es sich um gemischte
Daten handelt.  Für die leeren Bäume kommt ein eigener Record-Typ zum
Einsatz:\index{empty-tree@\texttt{empty-tree}}
%
\begin{alltt}
;  leerer Baum
(define-record-procedures empty-tree
  make-empty-tree empty-tree?
  ())
\end{alltt}
%
Auf den ersten Blick scheint hier ein Mißbrauch vorzuliegen~--
immerhin handelt es sich bei leeren Bäumen eindeutig nicht um
zusammengesetzte Daten: Der Record-Typ
hat kein einziges Feld.  Record-Typen haben aber noch die 
Funktion, neue Datensorten einzuführen, und sind darum auch dann das
Mittel der Wahl für die Realisierung gemischter Daten,
wenn es sich nicht wirklich um zusammengesetzte Daten
handelt.  In diesem Fall reicht es, nur einen leeren
Baum zu haben, genauso wie es nur eine leere Liste gibt:\index{the-empty-tree@\texttt{the-empty-tree}}
%
\begin{alltt}
(define the-empty-tree (make-empty-tree))
\end{alltt}
%
Knoten hingegen sind tatsächlich zusammengesetzte Daten: Ein Knoten
besteht aus seiner Markierung sowie linkem und rechtem
Teilbaum.  Da es Bäume über verschiedenen Sorten von Markierungen
gibt, ist die Signatur für Knoten parametrisch:\index{node@\texttt{node}}\index{node@\texttt{node}}
%
\begin{alltt}
; Ein Knoten besteht aus:
; - Markierung
; - linker Teilbaum
; - rechter Teilbaum
(define-record-procedures-parametric node node-of*
 make-node node?
 (node-label
  node-left-branch node-right-branch))

(define node-of
 (lambda (x)
   (signature
    (node-of* x (tree-of x) (tree-of x)))))
\end{alltt}
%
Hier ist die Datendefinition für Bäume im allgemeinen:
%
\begin{alltt}
; Ein Binärbaum ist eins der folgenden:
; - ein leerer Baum
; - ein Knoten
(define tree-of
 (lambda (x)
   (signature (mixed empty-tree (node-of x)))))
\end{alltt}
%
Damit kann ein Baum wie der in Abbildung~\ref{fig:bintree1} durch
folgenden Scheme-Ausdruck konstruiert werden:
%
\begin{alltt}
(: t (tree-of string))
(define t
  (make-node
   "A"
   (make-node
    "B"
    (make-node
     "C"
     the-empty-tree
     (make-node "D" the-empty-tree the-empty-tree))
    the-empty-tree)
   (make-node
    "E"
    (make-node "F" the-empty-tree the-empty-tree)
    the-empty-tree)))
\end{alltt}
%
Hier sind zwei weitere Beispielbäume über ganzen Zahlen:
% HK: t1 modifiziert, damit node-count ein anderes Ergebnis liefert als depth
\begin{alltt}
(: t1 (tree-of integer))
(define t1
  (make-node 3
    (make-node 4
      the-empty-tree
      (make-node 7 the-empty-tree the-empty-tree))
    (make-node 8 the-empty-tree the-empty-tree)))

(: t2 (tree-of integer))
(define t2 (make-node 17 (make-node 3 the-empty-tree t1) the-empty-tree))
\end{alltt}
Als Beispiel für das Programmieren mit Bäumen dient die
\textit{Tiefe}\index{Tiefe!eines Baumes}
eines Binärbaums, also die maximale Anzahl von "`Ebenen"' im Bild des
Binärbaums.  Hier sind Kurzbeschreibung, Signatur, Testfälle und
Gerüst:\index{depth@\texttt{depth}}
%
\begin{alltt}
; Tiefe eines Baums berechnen
(: depth ((tree-of %a) -> natural))

(check-expect (depth t1) 3)
(check-expect (depth t2) 5)

(define depth
  (lambda (tree)
    ...))
\end{alltt}
%
Es geht weiter strikt nach Anleitung: Es handelt sich um gemischte
Daten, also kommt eine Verzweigung zum Einsatz.  Da es zwei
verschiedene Sorten Bäume gibt, hat die Verzweigung zwei Zweige:
%
\begin{alltt}
(define depth
  (lambda (t)
    (cond
      ((empty-tree? t)
       ...)
      ((node? t)
       ...))))
\end{alltt}
%
Der erste Fall ist einfach: Der leere Baum hat die Tiefe 0.  Im
zweiten Fall geht es um Knoten, die wiederum Bäume enthalten.  Genau
wie bei Listen gibt es also Selbstreferenzen und damit Rekursion:
%
\begin{alltt}
(define depth
  (lambda (t)
    (cond
      ((empty-tree? t)
       0)
      ((node? t)
       ... (node-label t) ...
       ... (depth (node-left-branch t)) ...
       ... (depth (node-right-branch t)) ...))))
\end{alltt}
%
Die Markierung spielt keine Rolle für die Tiefe, kann also wegfallen.
Bei den Teilbäumen spielt für die Tiefe des Knotens nur der tiefere
der beiden eine Rolle.  Der Knoten ist um eins tiefer als das Maximum
der Tiefen der Teilbäume:
%
\begin{alltt}
(define depth
  (lambda (t)
    (cond
      ((empty-tree? t)
       0)
      ((node? t)
       (+ 1
          (max (depth (node-left-branch t))
               (depth (node-right-branch t))))))))
\end{alltt}
%
(\texttt{Max}\index{max@\texttt{max}} ist eine eingebaute Funktion in
Scheme, die das Maximum ihrer Argumente ermittelt.)

Auch \texttt{depth} folgt einer 
Schablone, die für viele Prozeduren auf Bäumen
gilt; Aufgabe~\ref{ex:recipe-trees} beschäftigt sich damit.
Ihr folgt auch die Funktion \texttt{node-count}, welche die Anzahl der
Knoten in einem Binärbaum liefert:\index{node-count@\texttt{node-count}}
%
\begin{alltt}
; Knoten in Baum zählen
(: node-count ((tree-of %a) -> natural))

(check-expect (node-count t1) 4)
(check-expect (node-count t2) 6)

(define node-count
  (lambda (t)
    (cond
      ((empty-tree? t)
       0)
      ((node? t)
       (+ 1
          (node-count (node-left-branch t))
          (node-count (node-right-branch t)))))))
\end{alltt}
%

\section{Suchbäume}
\label{sec:search-trees}

Viele Programme benötigen irgendeine Form von Suchfunktionalität in
einer Menge von Daten: Es könnte sich um die Suche nach einer
Telefonnummer, einer Primzahl oder einer schon geleisteten Aufgabe
handeln.  Eine effiziente Realisierung für eine Suchfunktionalität
sind \textit{Suchbäume\index{Suchbaum}}.  Ein Suchbaum besteht aus
"`Elementen"': Neue Elemente können eingefügt werden, und für ein
gegebenes Element kann festgestellt werden, ob es im Suchbaum
vorhanden ist.

Suchbäume setzen eine Gleichheitsoperation und eine totale Ordnung auf
den Elementen (siehe Definition~\ref{def:total-order}) voraus.

\begin{figure}[tb]
\begin{pspdf}
\begin{center}
  \pstree[levelsep=4ex,treesep=3em,nodesep=2pt]{\Tr{\texttt{N}}} {
    \pstree{\Tr{\texttt{B}}} { \pstree{\Tr{\texttt{A}}}{\Tdot \Tdot} \pstree{\Tr{\texttt{D}}}{\Tdot \Tdot }}
    \pstree{\Tr{\texttt{Q}}} { \pstree{\Tr{\texttt{P}}}{\Tdot \Tdot} \pstree{\Tr{\texttt{R}}}{\Tdot \Tdot} } }
\end{center}
\end{pspdf}
  \caption{Ein Suchbaum über Buchstaben}
  \label{fig:searchtree}
\end{figure}

\index{Suchbaum}\index{Baum!Such-} Sei also $S$ eine total geordnete
Menge.  Dann ist ein \textit{Suchbaum} über $S$ ein Binärbaum, so daß bei
jedem Knoten alle Markierungen in seinem linken Teilbaum kleiner und
alle in seinem rechten Teilbaum größer sind als die Markierung des Knotens
selbst.  Diese Eigenschaft des Baums heißt auch
\textit{Suchbaumeigenschaft} (bezüglich der gewählten totalen
Ordnung).
Abbildung~\ref{fig:searchtree} zeigt einen Suchbaum über Buchstaben,
die alphabetisch geordnet sind.

Die Markierung eines Knotens bestimmt,
in welchem Teilbaum
des Knotens eine gesuchte Markierung stecken muß, wenn diese nicht sowieso schon die gesuchte ist: Ist die gesuchte
Markierung kleiner als die des Knotens, so muß sie (wenn überhaupt) im
linken Teilbaum stecken; wenn sie größer ist, im rechten.
Insbesondere ist es nicht nötig, im jeweils anderen Teilbaum nach der
Markierung zu suchen.

Für Suchbäume wird ein neuer Record-Typ
definiert.\index{search-tree@\texttt{search-tree}} Zu einem Suchbaum
gehören neben dem Baum selbst auch noch Operationen für
Gleichheit und die "`Kleiner-als"'-Relation auf den Markierungen, beide repräsentiert durch
Prädikate (die zum Binärbaum und zueinander passen müssen).  Genau wie
Bäume sind auch Suchbäume parametrisch:
%
\begin{alltt}
; Ein Suchbaum besteht aus
; - einer Funktion, die zwei Markierungen auf Gleichheit testet,
; - einer Funktion, die vergleicht, ob eine Markierung kleiner als die andere ist
; - einem Binärbaum
(: make-search-tree ((%a %a -> boolean) 
                     (%a %a -> boolean) 
                     (tree-of %a) 
                         -> (search-tree-of %a)))
(: search-tree? (any -> boolean))
(: search-tree-label-equal-proc ((search-tree-of %a) -> (%a %a -> boolean)))
(: search-tree-label-less-than-proc ((search-tree-of %a) -> (%a %a -> boolean)))
(: search-tree-tree ((search-tree-of %a) -> (tree-of %a)))

(define-record-procedures-parametric search-tree search-tree-of*
  make-search-tree search-tree?
  (search-tree-label-equal-proc
   search-tree-label-less-than-proc
   search-tree-tree))

(define search-tree-of
  (lambda (x)
    (signature
     (search-tree-of* (x x -> boolean) (x x -> boolean) (tree-of x)))))
\end{alltt}
%
Alle Suchbäume fangen
beim leeren Suchbaum an:\index{make-empty-search-tree@\texttt{make-empty-search-tree}}
%
\begin{alltt}
; leeren Suchbaum konstruieren
(: make-empty-search-tree
   ((%a %a -> boolean) (%a %a -> boolean) -> (search-tree-of %a)))
(define make-empty-search-tree
  (lambda (label-equal-proc label-less-than-proc)
    (make-search-tree label-equal-proc label-less-than-proc
                      the-empty-tree)))
\end{alltt}
%
Ohne weiterführende Prozeduren gibt es hier noch nichts zu testen. Hier kommt
aber schon ein Beispiel zu Testzwecken:

\begin{alltt}
(define s1
  (make-search-tree
   = <
   (make-node 5
              (make-node 3 the-empty-tree the-empty-tree)
              (make-node 17
                         (make-node 10 
                                    the-empty-tree 
                                    (make-node 12 the-empty-tree the-empty-tree))
                         the-empty-tree))))
\end{alltt}
%
Die nachfolgende Funktion \texttt{search"=tree"=member?} 
stellt fest, ob ein Knoten mit Markierung \texttt{l} in einem
Suchbaum \texttt{s} vorhanden ist. Die eigentliche Arbeit
macht die lokale Hilfsprozedur \texttt{member?},\index{search-tree-member?@\texttt{search-tree-member?}}
die auf dem zugrundeliegenden Binärbaum operiert.  Da \texttt{member?} rekursiv
ist, wird sie mit \texttt{letrec} (siehe
Abbildung~\ref{scheme:letrec}) gebunden.
%
\begin{alltt}
; feststellen, ob Element in Suchbaum vorhanden ist
(: search-tree-member? (%a (search-tree-of %a) -> boolean))

(check-expect (search-tree-member? 3 s1) #t)
(check-expect (search-tree-member? 5 s1) #t)
(check-expect (search-tree-member? 9 s1) #f)
(check-expect (search-tree-member? 10 s1) #t)
(check-expect (search-tree-member? 17 s1) #t)

(define search-tree-member?
  (lambda (l s)
    (let ((label-equal? (search-tree-label-equal-proc s))
          (label-less-than? (search-tree-label-less-than-proc s)))
      (letrec
          ;; member? : tree -> bool
          ((member?
            (lambda (t)
              (cond
               ((empty-tree? t) #f)
               ((node? t)
                (cond                 
                  ((label-equal? (node-label t) l)
                   #t)
                  ((label-less-than? l (node-label t))
                   (member? (node-left-branch t)))
                  (else
                   (member? (node-right-branch t)))))))))
        (member? (search-tree-tree s))))))
\end{alltt}
%
\texttt{Search-tree-member?} packt zunächst die beiden
Vergleichsoperationen \texttt{label-equal?} und
\texttt{label-less-than?} aus dem Suchbaum aus.
Dann wird die Hilfsprozedur \texttt{member?} aufgerufen.

Da es zwei Arten Binärbäume gibt, folgt 
\texttt{member?} zunächst der Konstruktionsanleitung für
gemischte Daten.  Im Zweig für den leeren Baum ist die Antwort
klar.  Im Zweig für einen Knoten vergleicht \texttt{member?} die
gesuchte Markierung mit der des Knotens.  Dabei gibt es drei
Möglichkeiten, also auch drei Zweige: Bei Gleichheit ist die Markierung
gefunden.  Ansonsten wird \texttt{member?} entweder auf den linken
oder den rechten Teilbaum angewendet, je nachdem, in welchem Teilbaum
die Markierung stehen muß.

\texttt{Search-tree-member?} kann nur richtig funktionieren, wenn das
Argument \texttt{s} tatsächlich die Suchbaumeigenschaft
erfüllt.  Rein prinzipiell ist es möglich, durch Mißbrauch von
\texttt{make-search-tree} einen Wert vom Typ \texttt{search-tree} zu
erzeugen, der nicht die Suchbaumeigenschaft erfüllt, wie etwa
\texttt{s2} hier:
%
\begin{alltt}
(define s2
  (make-search-tree
   = <
   (make-node 5
              (make-node 17 the-empty-tree the-empty-tree)
              (make-node 3 the-empty-tree the-empty-tree))))
\end{alltt}
%
Zu \texttt{s2} paßt das folgende Bild:
%
\begin{pspdf}
\begin{center}
  \pstree[levelsep=4ex,treesep=3em,nodesep=2pt]{\Tr{5}}
  {
    \pstree{\Tr{17}}{\Tdot\Tdot}
    \pstree{\Tr{3}}{\Tdot\Tdot}
    }
\end{center}
\end{pspdf}
%
In diesem "`Suchbaum"' findet \texttt{search-tree-member?} zwar
die 5, nicht aber die anderen beiden Elemente:\label{label:non-search-tree}
%
\begin{alltt}
(search-tree-member? 5 s2)
\evalsto{} #t
(search-tree-member? 17 s2)
\evalsto{} #f
(search-tree-member? 3 s2)
\evalsto{} #f
\end{alltt}
%
Aus diesem Grund sollte \texttt{make-search-tree} nur "`intern"'
verwendet werden.   Ansonsten sollten nur die Funktionen \texttt{make"=empty"=search"=tree} und eine
neue Funktion \texttt{search"=tree"=insert} verwendet werden, die
ein neues Element in den
Suchbaum einfügt und dabei die Suchbaumeigenschaft
erhält.\index{search-tree-insert@\texttt{search-tree-insert}}
Hier Kurzbeschreibung und Signatur:
%
\begin{alltt}
; neues Element in Suchbaum einfügen
(: search-tree-insert (%a (search-tree-of %a) -> (search-tree-of %a)))
\end{alltt}
%
Für die Testfälle wird ein Suchbaum \texttt{s3} wie folgt definiert:
%
\begin{alltt}
(define s3
  (search-tree-insert
   5
   (search-tree-insert
    17
    (search-tree-insert
     3
     (make-empty-search-tree = <)))))
\end{alltt}
%
Die Testfälle werden dann wie zuvor mit Hilfe von
\texttt{search"=tree"=member?} formuliert:
%
\begin{alltt}
(check-expect (search-tree-member? 5 s3) #t)
(check-expect (search-tree-member? 17 s3) #t)
(check-expect (search-tree-member? 3 s3) #t)
(check-expect (search-tree-member? 13 s3) #f)
(check-expect (search-tree-member? -1 s3) #f)
\end{alltt}
%
Zu beachten ist, daß die Definition von \texttt{s3} im Programm
\emph{hinter} die Definition von \texttt{search-tree-insert} gestellt
wird, da diese Prozedur für die Auswertung der rechten Seite benötigt wird.
%
\begin{alltt}
(define search-tree-insert
  (lambda (l s)
    (let ((label-equal? (search-tree-label-equal-proc s))
          (label-less-than? (search-tree-label-less-than-proc s)))
      (letrec
          ; (: insert (tree-of %a) -> (tree-of %a))
          ((insert
            (lambda (t)
              (cond
               ((empty-tree? t)
                (make-node l the-empty-tree the-empty-tree))
               ((node? t)
                (cond
                  ((label-equal? l (node-label t))
                   t)
                  ((label-less-than? l (node-label t))
                   (make-node (node-label t)
                              (insert (node-left-branch t))
                              (node-right-branch t)))
                  (else
                   (make-node (node-label t)
                              (node-left-branch t)
                              (insert (node-right-branch t))))))))))
        (make-search-tree
         label-equal? label-less-than?
         (insert (search-tree-tree s)))))))
\end{alltt}
%
Im Herzen von \texttt{search-tree-insert} erledigt die rekursive
Hilfsprozedur \texttt{insert} die eigentliche Arbeit: Soll
\texttt{l} in den leeren Baum eingefügt werden, so gibt
\texttt{insert} einen trivialen Baum der Form
%
\begin{pspdf}
\begin{center}
    \pstree[levelsep=4ex,treesep=3em,nodesep=2pt]{\Tr{\texttt{l}}}
    {\Tdot\Tdot}
\end{center}
\end{pspdf}
% 
zurück.  Wenn \texttt{t} ein Knoten ist, gibt es wieder drei Fälle:
Wenn \texttt{l} mit der
Knotenmarkierung übereinstimmt, so ist es bereits im alten Baum
vorhanden~-- \texttt{insert} kann \texttt{t} unverändert
zurückgeben.  Ansonsten muß
\texttt{l} im linken oder rechten Teilbaum eingefügt werden,
und \texttt{insert} bastelt aus dem neuen Teilbaum und dem anderen, alten Teilbaum
einen neuen Baum zusammen.

Das Resultat des Aufrufs von \texttt{insert} am Ende der Funktion wird
schließlich wieder in einen \texttt{search-tree}-Wert eingepackt,
mit denselben \texttt{label-equal}- und \texttt{label"=less"=than}"=Operationen wie vorher.

Die Prozedur \texttt{search-tree-member?} muß für den Suchbaum in
Abbildung~\ref{fig:searchtree} nicht alle Elemente nach dem gesuchten
durchforsten; \texttt{search-tree-member?} sucht auf direktem Weg
von der Wurzel des Suchbaums nach unten zum gesuchten Element.  Da pro
weiterer "`Ebene"' eines Binärbaums jeweils doppelt soviele Elemente
Platz finden als in der vorhergehenden, wächst die Anzahl der Ebenen
des Baums~-- die Tiefe also~-- nur mit dem Zweierlogarithmus der
Anzahl der Elemente, also viel langsamer als zum Beispiel die Länge
einer Liste, die alle Elemente aufnehmen müßte.

\begin{figure}[tb]
  \begin{pspdf}
  \begin{center}
    \pstree[nodesep=2pt,levelsep=4ex,treesep=3em]{\Tr{\texttt{A}}}{\Tdot
      \pstree{\Tr{\texttt{B}}}{\Tdot
        \pstree{\Tr{\texttt{D}}}{\Tdot
          \pstree{\Tr{\texttt{N}}}{\Tdot
            \pstree{\Tr{\texttt{P}}}{\Tdot
              \pstree{\Tr{\texttt{Q}}}{\Tdot
                \pstree{\Tr{\texttt{R}}}{\Tdot \Tdot}}}}}}}
    \caption{Ein entarteter Suchbaum}
    \label{fig:stupid-searchtree}
  \end{center}
\end{pspdf}
\end{figure}

Leider ist nicht jeder Suchbaum so angenehm organisiert wie der in
Abbildung~\ref{fig:searchtree}.  Abbildung~\ref{fig:stupid-searchtree}
zeigt einen Binärbaum, der zwar die Suchbaumeigenschaft erfüllt, aber
\textit{entartet\index{entarteter Suchbaum}\index{Suchbaum!entartet}}
ist:\label{label:entartet}
In diesem Suchbaum dauert die Suche genauso
lang wie in einer Liste.  Welche Form der Suchbaum hat und ob er
entartet wird, hängt von der
Reihenfolge der Aufrufe von \texttt{search"=tree"=insert}
ab, mit denen er konstruiert wird.  Es gibt allerdings Varianten von Suchbäumen, die bei
  \texttt{search-tree-insert} die Entartung vermeiden und den
  Suchbaum \textit{balancieren}.


\section{Eigenschaften der Suchbaum-Operationen}

\texttt{Search-tree-insert} ist eine der komplizierteren Funktionen in
diesem Buch; es ist alles andere als offensichtlich, daß sie korrekt
ist.  Die Testfälle mögen zwar punktuell auf die Korrektheit von
\texttt{search-tree-insert} und \texttt{search-tree-member?}
hindeuten, Sicherheit liefern sie jedoch nicht.  Der erste Schritt, um
mehr Vertrauen in die Korrektheit zu gewinnen, ist die Formulierung
von Eigenschaften und deren Überprüfung mit \texttt{check-expect}.
Der zweite Schritt ist ein Beweis der Suchbaumeigenschaft.

Um interessante Eigenschaften zu formulieren, müssen
\texttt{search-tree-insert} und \texttt{search"=tree"=member} gemeinsam
betrachtet werden: \texttt{search-tree-insert} allein kann nichts
sinnvolles anstellen, wenn nach den eingefügten Elementen nicht auch
gesucht werden kann.  Die wichtigsten Eigenschaften sind folgende:
%
\begin{enumerate}
\item Ein mit \texttt{search-tree-insert} eingefügtes Element wird
  stets von \texttt{search"=tree"=member?} wieder gefunden.
\item Wenn ein Element \emph{nicht} mit in den Suchbaum eingefügt
  wurde, wird es von \texttt{search"=tree"=member?} \emph{nicht}
  gefunden.
\end{enumerate}
%
Die erste Eigenschaft ergibt ohne die zweite wenig Sinn: Sie wäre auch
erfüllt, wenn \texttt{search-tree-member?} immer \verb|#t| liefern
würde.

Für einen Test mit \texttt{for-all} müssen beliebige Suchbäume
betrachtet werden.  Allerdings funktioniert der Ansatz
%
\begin{alltt}
(for-all ((st (search-tree-of %a)))
  ...)
\end{alltt}
%
nicht, schon weil \texttt{search-tree-of} eine  parametrisierte Signatur ist und
damit nicht direkt in \texttt{for-all} verwendet kann.  Außerdem
sollen die Suchbäume, die betrachtet werden, ja gerade mit
\texttt{search-tree-insert} konstruiert werden.  

Im folgenden legen wir uns auf eine bestimmte Parametrisierung der Suchbäume
fest, wohl wissend, dass es für die Suchbaumeigenschaft auf die
Parametrisierung im Grunde nicht ankommt. Der nächste Versuch
könnte deshalb so aussehen:
%
\begin{alltt}
(for-all ((el natural))
  ... (search-tree-insert el (make-empty-search-tree = <) ...)
\end{alltt}
%
Eine solche Eigenschaft würde allerdings nur Suchbäume mit einem
einzigen Element einbeziehen, also eher uninteressante Vertreter ihrer
Spezies.  Für substantielle Tests ist es notwendig, Suchbäume zu
betrachten, die aus unterschiedlichen (insbesondere unterschiedlich
langen) Folgen von \texttt{search-tree-insert}-Operationen entstanden
sind.  Da für die Repräsentation von Folgen in Scheme Listen zuständig
sind, bietet sich eine Eigenschaft folgender Form an:
%
\begin{alltt}
(for-all ((els (list-of natural)))
  ...)
\end{alltt}
%
Damit es funktioniert, muß nur die Liste \texttt{els} noch in einen
Suchbaum umgewandelt werden, der gerade ihre Elemente enthält.  Das
kann eine Hilfsprozedur namens \texttt{list->search"=tree} leisten.
Hier sind Kurzbeschreibung und Signatur:
%
\begin{alltt}
; aus allen Zahlen einer Liste einen Suchbaum machen
(: list->search-tree ((%a %a -> boolean)
                      (%a %a -> boolean) (list-of %a) -> (search-tree-of %a)))
\end{alltt}
%
Da \texttt{list->search-tree} nur sukzessive für alle Elemente der
Liste \texttt{search-tree-insert} aufruft, wird sie am einfachsten mit
\texttt{fold} programmiert:
%
\begin{alltt}
(define list->search-tree
  (lambda (= < els)
    (fold (make-empty-search-tree = <)
          search-tree-insert
          els)))
\end{alltt}
%
Die erste Eigenschaft~-- ob also jedes mit \texttt{search-tree-insert}
in einen Suchbaum eingefügte Element auch von
\texttt{search-tree-member?} gefunden wird, läßt sich jetzt mit Hilfe
der Funktion \texttt{every?} aus Abschnitt~\ref{page:every}
formulieren.  (Zur Erinnerung: \texttt{every?} wendet ein Prädikat auf
alle Elemente einer Liste an und gibt \verb|#t| zurück, wenn das
Prädikat für jedes Element \verb|#t| liefert, sonst \verb|#f|.)
%
\begin{alltt}
(check-property
 (for-all ((els (list-of natural)))
   (let ((st (list->search-tree = < els)))
     (every? (lambda (el)
               (search-tree-member? el st))
             els))))
\end{alltt}
%
Es fehlt noch die zweite Eigenschaft: Für ein Element, das nicht im
Suchbaum vorhanden ist, darf \texttt{search-tree-member?} auch nicht
\verb|#t| liefern.  Zum Test der Eigenschaft gehört also wie schon bei
der ersten Eigenschaft ein beliebiger Suchbaum sowie ein einzelnes Element:
%
\begin{alltt}
(for-all ((els (list-of natural))
          (el natural))
  ... (list->search-tree = < els) ...)
\end{alltt}
%
Der Test ist nur sinnvoll, wenn \texttt{el} \emph{nicht} Element der Liste
\texttt{els} ist: Das muß erst einmal überprüft werden, und zwar durch
eine Funktion, die testet, ob ein Wert Element einer Liste ist.  Hier
sind Kurzbeschreibung und Signatur:
%
\begin{alltt}
; ist Wert Element einer Liste?
(: member? ((%a %a -> boolean) %a (list-of %a) -> boolean))
\end{alltt}
%
Das erste Argument ist ein Gleichheitsprädikat, welches den gesuchten
Wert mit den Listenelementen vergleicht.  Hier sind einige Tests,
Gerüst und Schablone für die Prozedur, die eine Liste akzeptiert:
%
\begin{alltt}
(check-expect (member? = 5 empty) #f)
(check-expect (member? = 5 (list 1 2 3)) #f)
(check-expect (member? = 1 (list 1 2 3)) #t)
(check-expect (member? = 2 (list 1 2 3)) #t)
(check-expect (member? = 3 (list 1 2 3)) #t)

(define member?
  (lambda (= el lis)
    (cond
      ((empty? lis) ...)
      ((pair? lis)
       ... (first lis)
       ... (member? = el (rest lis)) ...))))
\end{alltt}
%
Im \texttt{empty?}-Zweig ist die Liste leer, das Ergebnis also
\verb|#f|.  Im anderen Fall muß die Prozedur feststellen, ob
\texttt{(first lis)} gerade das gesuchte Element \texttt{el} ist.
Vollständig sieht die Prozedur so aus:
%
\begin{alltt}
(define member?
  (lambda (= el lis)
    (cond
      ((empty? lis) #f)
      ((pair? lis)
       (if (= el (first lis))
           #t
           (member? = el (rest lis)))))))
\end{alltt}
%
Zurück zur Eigenschaft von \texttt{search-tree-member?}: Immer dann,
wenn \texttt{el} \emph{nicht} Element von \texttt{els} ist, darf auch
\texttt{search-tree-member?} \texttt{el} nicht finden.  Diese
Implikation wird mit \texttt{==>} formuliert:
%
\begin{alltt}
(check-property
 (for-all ((els (list-of natural))
           (el natural))
   (==> (not (member? = el els))
        (not (search-tree-member? el (list->search-tree = < els)))))) 
\end{alltt}
%
Die beiden \texttt{check-property}-Tests stärken also das Vertrauen in
die korrekte Funktionsweise von \texttt{search-tree-insert} und
\texttt{search-tree-member?}.  Aber auch hier ist Kontrolle über einen
Beweis der Korrektheit noch besser:
Es lohnt sich, etwas formaler über die
Korrektheit von \texttt{search-tree-insert} nachzudenken.  Zunächst
einmal ist es wichtig, zu formulieren, was der Begriff "`Korrektheit"'
im Zusammenhang mit \texttt{search-tree-insert} überhaupt bedeutet:
%
\begin{satz}\label{satz:suchbaum}
  \texttt{Search-tree-insert} \emph{erhält die Suchbaumeigenschaft}.
  Oder mit anderen Worten: Wenn das \texttt{search-tree}-Argument von
  \texttt{search"=tree"=insert} die Suchbaumeigenschaft erfüllt, so
  erfüllt auch der zurückgegebene Baum die Suchbaumeigenschaft\index{Suchbaumeigenschaft}.
\end{satz}
%
\begin{beweis}
  Die Korrektheit ist an der Hilfsprozedur \texttt{insert}
  festgemacht: Wenn das Argument von \texttt{insert} die
  Suchbaumeigenschaft erfüllt, so muß auch der Rückgabewert sie
  erfüllen.  Der Beweis funktioniert über strukturelle Induktion über
  den Wert \texttt{t}, der an den Baum $\tau$ gebunden sei.  Im
  Beweis gibt es vier Fälle, die den Zweigen der \texttt{cond}-Formen
  entsprechen:
%
\begin{itemize}
\item $\tau$ ist der leere Baum.  Der dann zurückgegebene Baum
  der Form
\begin{pspdf}
  \begin{center}
    \pstree[levelsep=4ex,treesep=3em,nodesep=2pt]{\Tr{$l$}}
    {\Tdot\Tdot}
  \end{center}
\end{pspdf}
erfüllt offensichtlich die Suchbaumeigenschaft.
\item $\tau$ ist ein Knoten, dessen Markierung mit
  $l$ übereinstimmt.  Dann gibt \texttt{insert} 
  $\tau$ zurück.  Da $\tau$ nach Voraussetzung die
  Suchbaumeigenschaft erfüllt, ist auch hier die
  Suchbaumeigenschaft erhalten.
\item $\tau$ ist ein Knoten, dessen Markierung \emph{größer}
  ist als $l$, sieht also so aus:
\begin{pspdf}
  \begin{center}
    \pstree[levelsep=4ex,treesep=3em,dotsize=0pt 0.1]{\Tr{$m$}}
    {
      \pstree{\Tdot}{\Ttri{\raisebox{-1ex}[0pt][1.8ex]{$a$}}}
      \pstree{\Tdot}{\Ttri{\raisebox{-1ex}[0pt][1.8ex]{$b$}}}
      }
  \end{center}
\end{pspdf}
  wobei sowohl $a$ als auch $b$ selbst die Suchbaumeigenschaft
  erfüllen. In diesem
  Fall sieht der entstehende Baum folgendermaßen aus:
  %
  \begin{pspdf}
  \begin{center}
    \pstree[levelsep=4ex,treesep=3em,dotsize=0pt 0.1]{\Tr{$m$}} {
      \pstree{\Tdot}{\Ttri{\raisebox{-1ex}[0pt][1.8ex]{
          \texttt{(insert \valof{a})}
          }}}
      \pstree{\Tdot}{\Ttri{\raisebox{-1ex}[0pt][1.8ex]{$b$}}} }
  \end{center}
  \end{pspdf}
% 
  Per Induktionsannahme erfüllt \texttt{(insert \valof{a})} die
  Suchbaumeigenschaft.  Da $b$ auch die Suchbaumeigenschaft
  erfüllt, muß nur noch gezeigt werden, daß alle Markierungen in
  \texttt{(insert \valof{a})} kleiner sind als $m$.
  Es gibt in \texttt{insert} drei Aufrufe von \texttt{make-node},
  die neue Knoten erzeugen können.  Alle fügen höchstens
  \texttt{l} zu der Menge der Markierungen des Baumes hinzu.
  Alle anderen Markierungen sind nach Voraussetzung kleiner als $m$,
  ebenso wie \texttt{l}.  Das Resultat erfüllt also ebenfalls
  die Suchbaumeigenschaft.
\item Im vierten Fall ist $\tau$ ein Knoten, dessen Markierung
  \emph{kleiner} ist als $l$.  Dieser Fall geht analog zum
  dritten Fall.
\end{itemize}
\end{beweis}
%
\section*{Aufgaben}

\begin{aufgabe}
  \label{ex:recipe-trees}
  Formulieren Sie eine spezielle Schablone für Prozeduren, die
  Binärbäume akzeptieren!
\end{aufgabe}

\begin{aufgabe}
  Schreiben Sie eine Prozedur, die einen Binärbaum akzeptiert
  und eine Liste aller Markierungen in dem Baum zurückgibt.
\end{aufgabe}

\begin{aufgabe}
  Wie muß \texttt{search-tree-insert} aufgerufen werden, um den
  Suchbaum in Abbildung~\ref{fig:searchtree} zu erzeugen?  Wie muß
  \texttt{search"=tree"=insert} aufgerufen werden, um den Suchbaum in
  Abbildung~\ref{fig:stupid-searchtree} zu erzeugen?
\end{aufgabe}

\begin{aufgabe}
  Schreiben Sie eine Funktion \texttt{search-tree-delete}, die ein
  Element aus einem Suchbaum entfernt.  Beweisen Sie, daß die Prozedur die
  Suchbaumeigenschaft erhält.
\end{aufgabe}

\begin{aufgabe}
  Die Implementierung von Suchbäumen ist für viele Suchprobleme nicht
  mächtig genug, da \texttt{search-tree-member?} nur überprüft,
  \emph{ob} ein Element in einem Suchbaum vorhanden ist.  Das hilft
  nicht viel z.B.\ beim Suchen von Telefonnummern zu gegebenen Namen.
  Erweitern Sie die Implementierung so, daß sie auch z.B.\ zum Suchen
  von Telefonnummern verwendet werden kann.  Realisieren Sie
  exemplarisch das Suchen nach Telefonnummern!

  Hinweis: Benutzen Sie als Markierungen im Suchbaum sogenannte
  \textit{Einträge}, die aus eine \textit{Schlüssel} (z.B.\ dem Namen)
  und dem \textit{Wert} bestehen.  Schreiben Sie dazu parametrische Daten-,
  Record- und Signaturdefinitionen. Ändern Sie \texttt{search-tree-insert}
  dahingehend, daß es Schlüssel und Element akzeptiert.  Schreiben Sie eine
  Prozedur \texttt{search-tree-find}, die zu einem Schlüssel den
  zugehörigen Wert findet.
\end{aufgabe}

\begin{aufgabe}
  Beweisen Sie die Korrektheit von \texttt{search-tree-member?}.
  Formulieren Sie zunächst eine geeignete Korrektheitseigenschaft und
  beweisen Sie diese mit Hilfe von Induktion!
\end{aufgabe}

%%% Local Variables: 
%%% mode: latex
%%% TeX-master: "i1"
%%% End: 
